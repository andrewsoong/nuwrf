\documentclass{article}

\usepackage{amsmath}
\usepackage{array}
\usepackage{color}

%\usepackage{draftwatermark} % COMMENT OUT FOR PUBLIC RELEASE

\usepackage{eso-pic}
\usepackage{fancyhdr}
\usepackage{fancyvrb}
\usepackage{graphicx}
\usepackage{hyperref}
\usepackage{ifthen}
\usepackage{longtable}
\usepackage{natbib}
\usepackage{pict2e}
\usepackage{soul}
\usepackage[nottoc]{tocbibind}
\usepackage[normalem]{ulem}

\pagestyle{fancy}
\rhead{}
\cfoot{} 

% Begin DRAFT water mark at center header and footer
%\chead{DRAFT} % COMMENT OUT FOR PUBLIC RELEASE
%\cfoot{DRAFT} % COMMENT OUT FOR PUBLIC RELEASE
% End DRAFT water mark at center header and footer

\rfoot{\thepage}

% Begin DRAFT water mark in center of page
%\SetWatermarkText{DRAFT} % COMMENT OUT FOR PUBLIC RELEASE
%\SetWatermarkScale{4}    % COMMENT OUT FOR PUBLIC RELEASE
%\SetWatermarkAngle{45}   % COMMENT OUT FOR PUBLIC RELEASE
% End DRAFT water mark in center of page

%------------------------------------------------------------------------------
\begin{document}

\pagenumbering{arabic}

\title{User Guide for Diagnosing Surface Bareness from NDVI for NU-WRF}
\author{Eric Kemp \\ Science Systems and Applications, Inc and \\
        NASA Goddard Space Flight Center}
\date{16 September 2016}

\maketitle

\section{Introduction}
\label{sec:Introduction}

The NASA Unified-Weather Research and Forecasting (NU-WRF) modeling system
\citep{ref:PetersLidardEtAl2014} includes a NASA-developed dynamic dust 
emissions scheme described briefly by~\cite{ref:ErodUserGuide}.  Two special
input fields are required by the emissions scheme:

\begin{enumerate}
\item A \emph{topographic depression} field ($S$), indicating geographic
regions where loose soils are likely to collect; and 
\item A \emph{surface bareness} field ($L$), indicating absence (presence)
of vegetation to permit (hinder) lofting of loose soils by high winds.
\end{enumerate}

The $S$ fields are static and are available from the NU-WRF team (the data
values are based on the 30$''$ GTOPO terrain distributed with the community 
WRF). In contrast, $L$ must be generated from Normalized Vegetation Difference
Index (NDVI) data, preferably valid at the NU-WRF simulation start time.  This 
document describes software for deriving $L$ from NDVI data from on of two 
possible sources:  the NASA Short-term Prediction Research and Transition 
Center (SPoRT; \url{http://weather.msfc.nasa.gov/sport/}), and the NASA Global 
Inventory Modeling and Mapping Studies (GIMMS) group
(\url{http://gimms.gsfc.nasa.gov/MODIS/}).

\section{Software}
\label{sec:Software}

\subsection{ndviBareness4Wrf}
\label{subsec:ndviBareness4Wrf}

A Fortran program (\textit{ndviBareness4Wrf}) has been developed to read 
quality-controlled NDVI to diagnose $L$ based on user-defined NDVI thresholds,
and to output the $L$ (also called BARE\_DYN) and NDVI fields in WPS 
intermediate binary format [see Chapter 2 of \cite{ref:ArwUserGuide}].  
For simplicity, the data are marked as valid at 00 UTC.  These files are 
subsequently read by the METGRID preprocessor to horizontally interpolate the 
fields to the user-specified WRF grid.  The interpolated fields are then 
passed to REAL for insertion into the initial condition files read by WRF.  

Currently two data formats are supported by the software:

\begin{itemize}
\item \textbf{NASA SPoRT MODIS NDVI composite data in GrADS binary format}.  
These NDVI data are in the cylindrical equidistant (latitude-longitude) 
projection at $0.01^{\circ}$ resolution; cover the contiguous United States, 
southern Canada, and northern Mexico; and are updated daily based on 
available Moderate Resolution Imaging Spectroradiometer (MODIS) data
\citep{ref:CaseEtAl2011}.  The \textit{ndviBareness4Wrf} executable will 
process the entire SPoRT domain for a specified day.  Per discussions with 
SPoRT personnel (J. Case, 2014, personnel communication), pixels with NDVI 
values between zero and 0.05 are reset to 0.05, while negative NDVI values
are reset to a missing value (-9999). The output file prefix can be set by 
the user.  If no file is processed, the program will terminate with an error.

\item \textbf{NASA GIMMS MODIS NDVI data in GeoTIFF format}.  These data are 
in the cylindrical equidistant (latitude-longitude) projection, and are
organized in $9^{\circ} \times 9^{\circ}$ regional tiles at $0.00225^{\circ}$ 
resolution every 8 days (see 
\url{http://gimms.gsfc.nasa.gov/MODIS/README.txt}).  Separate NDVI products 
are generated from the NASA \textit{Aqua} and \textit{Terra} satellites. Due 
to the large amount of NDVI data available via GIMMS, 
\textit{ndviBareness4Wrf} will automatically select only those tiles that 
include data contained in a user-specified latitude/longitude box.  In 
addition, each tile will be processed separately and have their own unique WPS
output file.  The output file prefixes for these data will always be 
\textit{\textbf{SAT}\_X\textbf{MM}Y\textbf{NN}}, where 
\textit{\textbf{SAT}} is either \textit{AQUA} or \textit{TERRA}, and 
\textit{\textbf{MM}} and \textit{\textbf{NN}} are tile number identifiers that
match the GIMMS tile index convention.  Masked pixel values (e.g., for water, 
snow, or bare land) are reset to a missing value (-9999). The NDVI pixels are 
also `flipped' to transform from the GIMMS y-axis convention (data progressing
north to south) to the WRF y-axis convention (data progressing from south to 
north). If a GeoTIFF tile file is missing, \textit{ndviBareness4Wrf} will 
print a warning message and continue (this behavior is due to gaps in tile 
coverage that exclude the poles and ocean regions far from land).  However, if
no tiles are processed at all, the program will terminate with an error.

\end{itemize}

The Fortran executable will expect a \textit{namelist.ndviBareness4Wrf} file
to specify run-time settings.  A single namelist block is used in this file,
and is documented below.

\begin{tabular}{|l|l|} \hline
Variable Names & Description \\ \hline
\&settings & \\ \hline
ndviFormat & Integer, specified NDVI file format. \\
           & SPoRT MODIS GrADS = 1 \\
           & GIMMS MODIS GeoTIFF AQUA = 10 \\ 
           & GIMMS MODIS GeoTIFF TERRA = 11 \\ \hline
lowerLeftLat & Real, specifies southwestern latitude limit of required \\
             & data. Only used for GIMMS GeoTIFF files. \\ \hline
lowerLeftLon & Real, specifies southwestern longitude limit of required \\
             & data. Only used for GIMMS GeoTIFF files. \\ \hline
upperRightLat & Real, specifies northeastern latitude limit of required \\
              & data. Only used for GIMMS GeoTIFF files. \\ \hline
upperRightLon & Real, specifies northeastern longitude limit of required \\
              & data. Only used for GIMMS GeoTIFF files. \\ \hline
ndviBarenessThreshold1 & Real, specifies lower threshold of NDVI to \\
                       & diagnose bare soil.  NDVI values below this number \\
                       & are assumed to indicate water. \\ \hline
ndviBarenessThreshold2 & Real, specifies upper threshold of NDVI to \\
                       & diagnose bare soil.  NDVI values above this number \\
                       & are assumed to indicate vegetation. \\ \hline
inputDirectory & String, specifies top directory with NDVI input files. \\
               & For SPoRT, files are assumed to be in subdirectories for \\
               & year and month (e.g., 2016/07). \\
               & For GIMMS, files are assumed to be in subdirectories for \\
               & year and day-of-year (e.g., 2011/185). \\ \hline
year & Integer, specifies valid year of NDVI data. \\ \hline
month & Integer, specifies valid month of NDVI data. \\ \hline
dayOfMonth & Integer, specifies valid day-of-month of NDVI data. \\ \hline
outputDirectory & String, specifies directory to write output files. \\ \hline
outputFilePrefix & String, specifies prefix of output file names.  Ignored \\
                 & if processing GIMMS MODIS GeoTIFF files. \\ \hline
\end{tabular} \\

\subsection{proc\_sport\_modis\_ndvi.py}
\label{subsec:procSportModisNdviPy}

To assist users in processing SPoRT files, the Python script 
\textit{proc\_sport\_modis\_ndvi.py} has been written.  This script will 
use pre-staged SPoRT GrADS NDVI files.  For each day in the specified date 
range, the script will generate a customized 
\textit{namelist.ndviBareness4Wrf} file and run the \textit{ndviBareness4Wrf}
program to generate WPS intermediate binary files.  The entire SPoRT domain
will be processed if the file is found; if the file for a particular date is
missing, the script will write a warning message and continue; and if no
files are found the script will fail with an error. 

The script uses a configuration file \textit{sport\_modis.cfg} for most
settings, but will also recognize several optional command line arguments to
modifiy behavior.  These are described below.

\subsubsection{sport\_modis.cfg}
\label{subsubsec:sportModisCfg}

This configuration file is designed for use with the Python standard library
module \textit{ConfigParser} (see 
\url{https://docs.python.org/2/library/configparser.html}).  It is made up
of a single `section' of data: \\

\begin{tabular}{|l|l|} \hline
Variable Names          & Description \\ \hline
[SPORT\_MODIS]          & \\ \hline
start\_year:            & Integer, year of first NDVI file to \\
                        & process. \\ \hline
start\_month:           & Integer, month of first NDVI file to \\
                        & process. \\ \hline
start\_day:             & Integer, day-of-month of first NDVI file to \\
                        & process. \\ \hline
end\_year:              & Integer, year of final NDVI file to \\ 
                        & process. \\ \hline
end\_month:             & Integer, month of final NDVI file to \\
                        & process. \\ \hline
end\_day:               & Integer, day-of-month of final NDVI file to \\
                        & process. \\ \hline
ndvi\_bareness\_4\_wrf\_path: & String, path to \textit{ndviBareness4Wrf} \\
                              & executable. \\ \hline
top\_work\_dir: & String, path to top level work directory where files \\
                & will be processed. \\ \hline
top\_local\_archive\_dir: & String, top-level directory containing \\
                          & pre-fetched NDVI files.  \\ \hline
ndvi\_bareness\_threshold\_1: & Real, lower threshold of NDVI to diagnose \\
                              & bare soil.  NDVI values below this number \\
                              & are assumed to indicate water. \\ \hline
ndvi\_bareness\_threshold\_2: & Real, upper threshold of NDVI to diagnose \\
                              & bare soil.  NDVI values above this number \\
                              & are assumed to indicate vegetation. \\ \hline
output\_file\_prefix: & String, will be used as prefix for WPS output \\
                      & files. \\ \hline
\end{tabular} \\

\subsection{proc\_gimms\_modis\_ndvi.py}
\label{subsec:procGimmsModisNdviPy}

Similar to \textit{proc\_sport\_modis\_ndvi.py}, the
\textit{proc\_gimms\_sport\_ndvi.py} script has been written to simplify
the process of collecting and processing GIMMS NDVI tiles. This
script will either download science quality Aqua or Terra NDVI tiles from the 
GIMMS server (\url{ftp://gimms.gsfc.nasa.gov/MODIS}) or alternatively use 
files already staged on local disk.  For each day in the specified date range,
the script will generate a customized \textit{namelist.ndviBareness4Wrf} file 
and run the \textit{ndviBareness4Wrf} executable to generate WPS intermediate 
binary files.  If downloading, the script will only fetch those tiles that 
have data in the user-specified latitude-longitude box; if a tile file is 
missing, the script will print a warning message and continue.  If no NDVI 
files can be found for processing, the script will terminate with an error. If
the script downloads files, it will optionally delete them after processing to
conserve disk space.

The Python script uses a configuration file \textit{gimms\_modis.cfg} for most 
settings and recognizes several optional command line flags.

\subsubsection{gimms\_modis.cfg}
\label{gimmsModisCfg}

This configuration file is similar to \textit{sport\_modis.cfg} but is
specific to the GIMMS GeoTIFF files.  The options are summarized below:

\begin{tabular}{|l|l|} \hline
Variable Names          & Description \\ \hline
[GIMMS\_MODIS]          & \\ \hline
satellite:              & String, must be either Aqua or Terra. \\ \hline
start\_year:            & Integer, year of first NDVI file to \\
                        & process. \\ \hline
start\_month:           & Integer, month of first NDVI file to \\
                        & process. \\ \hline
start\_day:             & Integer, day-of-month of first NDVI file to \\
                        & process. \\ \hline
end\_year:              & Integer, year of final NDVI file to \\ 
                        & process. \\ \hline
end\_month:             & Integer, month of final NDVI file to \\
                        & process. \\ \hline
end\_day:               & Integer, day-of-month of final NDVI file to \\
                        & process. \\ \hline
ndvi\_bareness\_4\_wrf\_path: & String, path to \textit{ndviBareness4Wrf} \\
                              & executable. \\ \hline
top\_work\_dir: & String, path to top level work directory where files \\
                & will be processed. \\ \hline
top\_local\_archive\_dir: & String, top-level directory containing \\
                          & pre-fetched NDVI files.  Only used if \\
                          & appropriate command-line flag is invoked; \\
                          & otherwise, files will be downloaded into work \\
                          & directory. \\ \hline
lower\_left\_lat: & Real, southwestern latitude limit of required \\
                  & data. \\ \hline
lower\_left\_lon: & Real, southwestern longitude limit of required \\
                  & data. \\ \hline
upper\_right\_lat: & Real, northeastern latitude limit of required \\
                   & data. \\ \hline
upper\_right\_lon: & Real, northeastern longitude limit of required \\
                   & data. \\ \hline
ndvi\_bareness\_threshold\_1: & Real, lower threshold of NDVI to diagnose \\
                              & bare soil.  NDVI values below this number \\
                              & are assumed to indicate water. \\ \hline
ndvi\_bareness\_threshold\_2: & Real, upper threshold of NDVI to diagnose \\
                              & bare soil.  NDVI values above this number \\
                              & are assumed to indicate vegetation. \\ \hline
\end{tabular} \\

\subsubsection{Command line flags}
\label{subsubsec:GimmsCmdLineFlags}

The following flags are recognized on the command line:

\begin{itemize}

\item \textbf{-H} and \textbf{-{}-help} will cause the script to print 
a summary of all the command line flag options and then exit.

\item \textbf{-D} and \textbf{-{}-download} instructs the script to
download any files from the SPoRT server via FTP.  By default, the script will
process pre-staged files.

\item \textbf{-S} and \textbf{-{}-save} instructs the script not to delete
any files that have been downloaded.  By default, the downloaded files
will be deleted after processing to save disk space.

\end{itemize}

\subsection{link\_gimms\_wps\_files.py}
\label{subsec:linkGimmsWpsFilesPy}

The nominal 8-day resolution of the GIMMS NDVI products presents a challenge,
since WRF's preprocessors expect all time varying input data to have the same
valid times and intervals.  Fortunately, it is possible to `trick' the 
METGRID preprocessor into ingesting the same data multiple times by using
symbolic links.  To aid the user in creating these, a Python script called
\textit{link\_gimms\_wps\_files.py} has been developed.  This script will
loop through a series of date/times, identify when a particular date/time
has GIMMS WPS files available, record the names of those files (in particular,
the tile numbers), and for dates/times without data produce symbolic links to 
the nearest preceding files.  Note that the script will fail if no GIMMS WPS 
files exist for the start date/time.

The benefit of using this script instead of 
\textit{link\_wps\_intermediate\_files.py} (see 
section \ref{subsec:linkWpsIntermediateFilesPy}) is that the present script
will search for and properly link multiple tiles of Aqua or Terra data.

The script uses a configure file \textit{link\_gimms.cfg} for all user-defined
settings.  This file contains a single `section' of data: \\

\begin{tabular}{|l|l|} \hline
Variable Names          & Description \\ \hline
[LINK\_GIMMS]           & \\ \hline
satellite:      & String, either Aqua or Terra. \\ \hline
top\_work\_dir: & String, path to top level work directory where files \\
                & will be linked. \\ \hline
start\_year:            & Integer, year of first NDVI file to \\
                        & link to. \\ \hline
start\_month:           & Integer, month of first NDVI file to \\
                        & link to. \\ \hline
start\_day:             & Integer, day-of-month of first NDVI file to \\
                        & link to. \\ \hline
start\_hour:            & Integer, hour (UTC) of first NDVI file to \\
                        & link to. \\ \hline
end\_year:              & Integer, final year for linking \\ 
                        & files. \\ \hline
end\_month:             & Integer, final month for linking \\
                        & files. \\ \hline
end\_day:               & Integer, final day-of-month for linking \\
                        & files. \\ \hline
end\_hour:              & Integer, final hour (UTC) for linking \\
                        & files. \\ \hline
link\_interval\_in\_hours: & Integer, interval for creating symbolic \\
                        & links. \\ \hline
\end{tabular} \\

\subsection{link\_wps\_intermediate\_files.py}
\label{subsec:linkWpsIntermediateFilesPy}

A generic version of \textit{link\_gimms\_wps\_files.py} is available for
SPoRT MODIS data or other WPS files with a time interval other
than the atmospheric lateral boundary conditions.  This script ---
\textit{link\_wps\_intermediate\_files.py} --- reads a 
\textit{link\_wps\_int.cfg} file for configuration settings.  However, the 
script also optionally accepts the WPS file name prefix from the command line 
with an appropriate flag (see section\ref{subsubsec:LinkWpsCmdLineFlags}). 
This can be useful if WPS files from multiple sources but identical time 
characteristics need to be linked, and it would be inconvenient for the user
to change \textit{link\_wps\_int.cfg} for each data source.

\subsubsection{link\_wps\_int.cfg}
\label{subsubsec:LinkWpsIntCfg}

The \emph{link\_wps\_int.cfg} file has a single `section' of data:

\begin{tabular}{|l|l|} \hline
Variable Names          & Description \\ \hline
[LINK\_WPS\_INT]        & \\ \hline
prefix:         & String, the exact prefix of the WPS intermediate files.\\
                & Ignored if valid command line flag is used. \\ \hline
top\_work\_dir: & String, path to top level work directory where files \\
                & will be linked. \\ \hline
start\_year:            & Integer, year of first NDVI file to \\
                        & link to. \\ \hline
start\_month:           & Integer, month of first NDVI file to \\
                        & link to. \\ \hline
start\_day:             & Integer, day-of-month of first NDVI file to \\
                        & link to. \\ \hline
start\_hour:            & Integer, hour (UTC) of first NDVI file to \\
                        & link to. \\ \hline
end\_year:              & Integer, final year for linking \\ 
                        & files. \\ \hline
end\_month:             & Integer, final month for linking \\
                        & files. \\ \hline
end\_day:               & Integer, final day-of-month for linking \\
                        & files. \\ \hline
end\_hour:              & Integer, final hour (UTC) for linking \\
                        & files. \\ \hline
link\_interval\_in\_hours: & Integer, interval for creating symbolic \\
                        & links. \\ \hline
\end{tabular} \\

\subsubsection{Command line flags}
\label{subsubsec:LinkWpsCmdLineFlags}

The following flags are recognized on the command line:

\begin{itemize}

\item \textbf{-H} and \textbf{-{}-help} will cause the script to print 
a summary of all the command line flag options and then exit.

\item \textbf{-P PREFIX} and \textbf{-{}-prefix=PREFIX} instructs the script to
accept the WPS file prefix name from the command line.  In this case, the 
prefix setting from the \textit{link\_wps\_int\_.cfg} file will be ignored.

\end{itemize}

\bibliographystyle{ametsoc}
\settocbibname{References}
\bibliography{references}

\end{document}

%-------------------------------------------------------------------------------------------------------------------
\begin{frame}
%  \titlepage
\maketitle
\end{frame}

%-------------------------------------------------------------------------------------------------------------------
\section{Introduction}
%-------------------------------------------------------------------------------------------------------------------

%-------------------------------------------------------------------------------------------------------------------
\begin{frame}{What is in this tutorial?}

\footnotesize{
This tutorial  describes the process of downloading, compiling, and running NU-WRF (Charney release, patch 1) for four different and representative workflows:
\begin{itemize}
     \item Basic: Based on WRF ARW.
     \item WRF-LIS: The recommended non-chemistry NU-WRF workflow (LIS MERRA forcing).
     \item Chemistry: The recommended chemistry (non-LIS) NU-WRF workflow.
     \item SCM: The Single Column Model NU-WRF workflow (with LIS coupling).
\end{itemize}
Each workflow consists of a series of steps starting with the acquisition of data files followed by the execution of several NU-WRF components. Successful completion of any workflow requires that \textbf{all} steps be completed in the order presented.
\mbox{}\\
For any additional information please consult the \href{https://nuwrf.gsfc.nasa.gov/sites/default/files/docs/nuwrf_userguide.pdf}{NU-WRF user's guide} at: https://nuwrf.gsfc.nasa.gov/sites/default/files/docs/nuwrf\_userguide.pdf
}

\end{frame}

%-------------------------------------------------------------------------------------------------------------------
\begin{frame}{Special instructions}

\footnotesize{
\begin{itemize}
\item All workflows are designed with pre-defined settings\footnote{That is, compiler/MPI combination, external libraries and run-time settings.}, including datasets. \emph{Workflows are not guaranteed to work if you deviate from the prescribed instructions}. So, please refrain from editing the inputs unless you know what you are doing. 
\item It is assumed that users are running on Discover or Pleiades using the bash shell. However, the steps described here \emph{should} work under other shells.
\end{itemize}
}
\textbf{IMPORTANT}:Henceforth all lines starting with '$>$' are Unix commands to be executed by the users.\\
\textbf{Note on paths}: On Pleiades \emph{most} paths are identical to the ones on Discover. So, in most cases just removing  "/discover" from the path should be enough for the setup instructions to work on both systems.\\
\textbf{Note on workfows}: Only the basic and default workflows are available on Pleiades.\\

\end{frame}


%-------------------------------------------------------------------------------------------------------------------
\begin{frame}{Useful links}

\footnotesize{
\begin{itemize}
\item \href{http://www2.mmm.ucar.edu/wrf/users/docs/user_guide_V3.5/contents.html}{Community ARW User Guide} http://www2.mmm.ucar.edu/wrf/users/docs/user\_guide\_V3.5/contents.html
\item \href{http://www2.mmm.ucar.edu/wrf/OnLineTutorial/index.htm}{Community WRF ARW Online Tutorial} http://www2.mmm.ucar.edu/wrf/OnLineTutorial/index.htm
\item \href{http://www.nccs.nasa.gov/primer/}{Documentation for using Discover and other NCCS systems} http://www.nccs.nasa.gov/primer/
\item \href{http://www.nas.nasa.gov/hecc/support/kb//}{Documentation for using Pleiades and other NAS systems} http://www.nas.nasa.gov/hecc/support/kb/
\end{itemize}
\textbf{NOTE:} The details for building the software on Discover and Pleiades are automatically handled by 
NUWRF's build scripts!
}

\end{frame}


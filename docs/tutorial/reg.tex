%-------------------------------------------------------------------------------------------------------------------
\section{Regression testing}
\label{sec:reg_testing}
%-------------------------------------------------------------------------------------------------------------------

%-------------------------------------------------------------------------------------------------------------------
\begin{frame}[fragile]\frametitle{Definition}

From \href{https://en.wikipedia.org/wiki/Regression_testing}{Wikipedia}:
\footnotesize{
\begin{verbatim}
Regression testing is a type of software testing that verifies that
software previously developed and tested still performs correctly
even after it was changed or interfaced with other software. 
Changes may include software enhancements, patches, 
configuration changes, etc. During regression  testing, new 
software bugs or regressions may be uncovered. 
\end{verbatim}
}
\end{frame}

%-------------------------------------------------------------------------------------------------------------------
\begin{frame}[fragile]\frametitle{Testing types}

Software testing can be roughly divided into three categories:
\begin{itemize}
\item Automated regression  tests.   These tests compile and run a very large number of model configurations and may perform various checks.
\item Manual/user testing.   These tests are intended to be performed by users that wish to verify their changes prior to pushing to a central repository.   In detail, these tests are similar to the automated tests, but are more easily executed from the command line. 
\item Unit tests.  These tests execute extremely quickly and provide a finer-grained verification than the other regression tests, albeit with far less coverage of the source code. \emph{Unit tests do not form part of the NU-WRF code base at this time}. 
\end{itemize}
In this section we discuss the available NU-WRF regression testing infrastructure and how it can be used to perform automated and manual tests.
\end{frame}

%-------------------------------------------------------------------------------------------------------------------
\begin{frame}[fragile]\frametitle{NU-WRF regression testing scripts}


The NU-WRF regression scripts are a set of python scripts and configuration
files designed to run tests on the NU-WRF code base.
The scripts' functionality is all driven by the settings specified in a single
configuration file.\\
\mbox{}\\
The configuration file is a text file with a particular structure readable by
the python scripts and easily understood by humans too! The files are organized
into sections, and each section contains name-value pairs for configuration data.\\

\end{frame}

%-------------------------------------------------------------------------------------------------------------------
\begin{frame}[fragile]\frametitle{NU-WRF regression testing scripts}

There is one configuration file required by the scripts - specified as a command
line argument. The file can have any name but must have the '.cfg' extension.
A sample configuration file (sample.cfg) is included \$NUWRFDIR/scripts/python/regression
that can be modified and used for "personalized" testing. \\
\mbox{}\\
Note:\\
\mbox{}\\
There are three additional files that are used specifically for NU-WRF testing:
master.cfg, develop.cfg and comp.cfg. These files should not be modified or used for anything
other than as reference to build your own customized configuration.

\end{frame}


%-------------------------------------------------------------------------------------------------------------------
\begin{frame}[fragile]\frametitle{Workflows revisited}

\footnotesize{
\begin{verbatim}
In this document we have discussed 4 workflows: basic, default, 
chemistry and scm. For the purposes of regression testing 
workflows are categorized as follows:

   wrf: default WRF-ARW runs (without LIS) (basic)
   wrflis: like wrf but with LIS-coupling (default and scm)
   chem: like wrf with chemistry (chemistry)
   kpp: like wrf with chemistry - uses KPP

This distinction is necessary to specify test cases in 
the configuration file. For a list of all supported NU-WRF
configurations see: 

$NUWRFDIR/scripts/python/regression/master.cfg
\end{verbatim}
}

\end{frame}


%-------------------------------------------------------------------------------------------------------------------
\begin{frame}[fragile]\frametitle{Python scripts}

The python scripts are the following:
\begin{itemize}
\item reg : Main driver.
\item RegPool.py    : A class with functionality to parallelize tasks.
\item regression.py : A driver script to execute the NU-WRF component tasks.
\item RegRuns.py    : A class that defines regression test run instances.
\item RegTest.py    : A class that defines regression test instances.
\item reg\_tools.py  : Various tools used by the main drivers.
\item reg\_utils.py  : Various utilities used by all the scripts.
\end{itemize}
The scripts can be used to perform both automated regression tests as well as "manual testing".
\end{frame}

%-------------------------------------------------------------------------------------------------------------------
\begin{frame}[fragile]\frametitle{Manual testing}

Before running the regression testing scripts make sure you duplicate the sample.cfg file 
and customize it to your needs. When ready to run the scripts, specify the configuration file 
name as an argument to reg and wait for the results:

\footnotesize{
\begin{verbatim}

$ cd $NUWRFDIR/scripts/python/regression 

To run the scripts, specify the configuration file name as an 
argument to reg:

$ reg <cfg_file> &

Note that the configuration file name extension should 
not be specified.
\end{verbatim}
}

\end{frame}

%-------------------------------------------------------------------------------------------------------------------
\begin{frame}[fragile]\frametitle{Manual testing}

\footnotesize{
\begin{verbatim}
Upon execution you will see some messages echoed to STDOUT 
but all the output will be logged to a file named <cfg_file>.LOG.

There will be other "log" files, one for each  <workflow> and one for 
each <experiment>, that will be generated for each 'reg' run.

Workflow builds, a maximum of four, will be generated in 
	<scratch_dir>/builds
	
Each build will have its own make.log as well as <workflow>.out 
and <workflow>.err  files that will be useful to diagnose build /run 
errors.
\end{verbatim}
}

\end{frame}

%-------------------------------------------------------------------------------------------------------------------
\begin{frame}[fragile]\frametitle{Manual testing}

\footnotesize{
\begin{verbatim}
Run logs will be generated in each run directory under 
<scratch_dir>/results.
These will have the name <experiment>-regression.log 
and should also be very useful to diagnose test-case-specific 
run-time issues.

At the end of a "reg" an email test report will be emailed to 
the recipient specified in the mail_to field in USERCONFIG.

Note that all 'reg' runs are tagged with a unique time stamp 
corresponding to the date/time the run was executed. 
\end{verbatim}
}

\end{frame}

%-------------------------------------------------------------------------------------------------------------------
\begin{frame}[fragile]\frametitle{Manual testing within interactive queue}

One can run NU-WRF test cases interactively. To do so make sure you set use\_batch=no in 
your configuration file. Also, if running interactively, one can only use one compiler option; i.e.
in the 'compilers' setting you can only use intel-sgimpt or gnu-openmpi. With those caveats in mind:

\footnotesize{
\begin{verbatim}
Request an interactive queue on DISCOVER, for example:

salloc --ntasks=84 --time=1:00:00 --account=s0492 --constraint=hasw

Once you get the prompt

$ cd $NUWRFDIR/scripts/python/regression 

$ reg <cfg_file> &

Note that the extension should not be specified.
\end{verbatim}
}

\end{frame}

%-------------------------------------------------------------------------------------------------------------------
\begin{frame}

As you may have figured out, all the workflows described 
earlier in this tutorial can be recreated with the appropriate
configuration file and the testing scripts.\\
\mbox{}\\
For more information about regression testing or any workflow described in this
tutorial please feel free to contact the NU-WRF software integration group.


\end{frame}




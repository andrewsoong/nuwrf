\comment{
# This is the full lis.config file.  It contains all the user-configurable
# options plus documentation.
#
# Please add any updates to the LIS code regarding configuration options
# to this file -- including documentation.
#
# Documentation must be placed in between 
#    #latex: BEGIN_DESCRIPTION
#    #latex: END_DESCRIPTION
# markers.
#
# Lines in the lis.config file must be placed in between 
#    #latex: BEGIN_CONFIG
#    #latex: END_CONFIG
# markers.
#
# Documentation for development-only configuration options should be placed
# in between
#    #latex: BEGIN_DEVELOPMENT_ONLY
#    #latex: END_DEVELOPMENT_ONLY
# markers.
#
# All documentation must the marked up with "#latex: " tags.
#
# Actual lines of the lis.config file should not be marked up.
#
# To include this file in the User's Guide:
# 1) Checkout the latest copy of this file from the repository.
# 2) Place it with the source for the User's Guide.
# 3) Rename it lis.config.tex
# 4) Edit the lis.config.tex file:
#    Remove the string "#latex:"
#    Replace the string "BEGIN_CONFIG" with the string "\begin{Verbatim}"
#    Replace the string "END_CONFIG" with the string "\end{Verbatim}"
#    Replace the string "BEGIN_DESCRIPTION" with the empty string
#    Replace the string "END_DESCRIPTION" with the empty string
#    For the developer's version:
#       Replace the string "BEGIN_DEVELOPMENT_ONLY" with the empty string
#       Replace the string "END_DEVELOPMENT_ONLY" with the empty string
#    For the public version:
#       Delete the lines between the "BEGIN_DEVELOPMENT_ONLY" and 
#       "END_DEVELOPMENT_ONLY" strings
#
# These are the commands (vi) that I use to process this file for the
# User's Guide.
#
#:%s/#latex://
#:%s/BEGIN_DESCRIPTION//
#:%s/END_DESCRIPTION//
#:%s/BEGIN_CONFIG/\\begin{Verbatim}[frame=single]/
#:%s/END_CONFIG/\\end{Verbatim}/
#
# For developer's version
#:%s/BEGIN_DEVELOPMENT_ONLY//
#:%s/END_DEVELOPMENT_ONLY//
#
# For public version, 1) replace <CR> with the single keycode ^M,
# 2) copy this into register c, and 3) execute it
#/BEGIN_DEVELOPMENT_ONLY<CR>ma/END_DEVELOPMENT_ONLY<CR>mb:'a,'bdelete<CR>@c
#
}

 
 \section{LIS config File} \label{sec:lisconfigfile}
 This section describes the options in the \file{lis.config} file.

 

 
 \subsection{Overall driver options} \label{ssec:driveropts}
 

 
 \var{Running mode:} specifies the running mode used in LIS.
 Acceptable values are:
 \begin{tabular}{ll}
 Value                    & Description                \\
 ``retrospective''        & retrospective mode         \\
 ``WRF coupling''         & Coupled WRF mode           \\
 ``ensemble smoother''    & ensemble smoother mode     \\
 \end{tabular}
 

 \begin{Verbatim}[frame=single]
Running mode: retrospective
 \end{Verbatim}


 
 \var{Number of nests:} specifies the number of nests used for the run. 
 Values 1 or higher are acceptable. The maximum number of nests is
 limited by the amount of available memory on the system. 
 The specifications for different nests are done using white spaces 
 as the delimiter. Please see below for further explanations. Note 
 that all nested domains should run on the same projection and same 
 land surface model.
 

 \begin{Verbatim}[frame=single]
Number of nests: 1
 \end{Verbatim}

 
 \var{Number of surface model types:} specifies the number of surface
 model types used for the run. 
 Values of 1 through \var{LIS\_rc\%max\_model\_types}
 (currently equal to 3)
 are acceptable.
 

 \begin{Verbatim}[frame=single]
Number of surface model types: 1
 \end{Verbatim}

 
 \var{Surface model types:} specifies the surface model types
 used for the run. 
 Acceptable values are:

 \begin{tabular}{ll}
 Value   &  Description        \\
 LSM     & land surface model  \\
 Lake    & lake model          \\
 Glacier & glacier model       \\
 \end{tabular}
 

 \begin{Verbatim}[frame=single]
Surface model types: LSM
 \end{Verbatim}

 
 \var{Surface model output interval:} specifies the surface model
 output interval.

 See Section \ref{ssec:timeinterval} for a description
 of how to specify a time interval.
 

 \begin{Verbatim}[frame=single]
Surface model output interval: 3hr
 \end{Verbatim}

 
 \var{Land surface model:} specifies the land surface model to run.
 Acceptable values are:

 \begin{tabular}{ll}
 Value       & Description           \\
 none        & template lsm          \\
 Noah.2.7.1  & Noah version 2.7.1    \\
 Noah.3.2    & Noah version 3.2      \\
 Noah.3.3    & Noah version 3.3      \\
 Noah.3.6    & Noah version 3.6      \\
 NoahMP.3.6  & NoahMP version 3.6    \\
 CLM.2       & CLM version 2.0       \\
 VIC.4.1.1   & VIC version 4.1.1     \\
 VIC.4.1.2   & VIC version 4.1.2.l   \\
 Mosaic      & Mosaic                \\
 HySSiB      & HySSiB                \\
 GeoWRSI.2   & GeoWRSI version 2.0   \\
 CABLE.1.4b  & CABLE version 1.4b    \\
 ``CLSM F2.5''  & Catchment Fortuna-2\_5 \\
 RDHM.3.5.6  & RDHM 3.5.6 (SacHTET and Snow17) \\
 \end{tabular}
 

 \begin{Verbatim}[frame=single]
Land surface model: Noah.2.7.1
 \end{Verbatim}




 
 \var{Number of met forcing sources:} specifies the
 number of met forcing datasets to be used. Acceptable 
 values are 0 or higher. 
 

 \begin{Verbatim}[frame=single]
Number of met forcing sources: 1
 \end{Verbatim}

 
 \var{Met forcing chosen ensemble member:} specifies the desired
 ensemble member from a given forcing data source to be assigned
 across all LIS ensemble members.  This option is enabled only if
 the met forcing data source contains its own ensembles.
 

 \begin{Verbatim}[frame=single]
Met forcing chosen ensemble member:
 \end{Verbatim}

 
 \var{Blending method for forcings:} specifies the
 blending method to combine forcings if more than one 
 forcing dataset is used. 
 Acceptable values are:

 \begin{tabular}{ll}
 Value    & Description                                    \\
 overlay  & datasets are overlaid on top of each other     \\
          & in the order they are specified                \\
 ensemble & each forcing dataset is assigned to a separate \\
          & ensemble member.                               \\
 \end{tabular}
 

 \begin{Verbatim}[frame=single]
Blending method for forcings: overlay
 \end{Verbatim}

 
 \var{Met forcing sources:} specifies the met forcing 
 data sources for the run. The values should be specified in a column 
 format.
 Acceptable values for the sources are:

 \begin{tabular}{ll}
 Value                         & Description                        \\
 ``none''                      & None                               \\
 ``GDAS''                      & GDAS                               \\
 ``GEOS''                      & GEOS                               \\
 ``GEOS5 forecast''            & GEOS5 Forecast                     \\
 ``ECMWF''                     & ECMWF                              \\
 ``GSWP1''                     & GSWP1                              \\
 ``GSWP2''                     & GSWP2                              \\
 ``ECMWF reanalysis''          & ECMWF Reanalysis                   \\
 ``PRINCETON''                 & Princeton                          \\
 ``NLDAS1''                    & NLDAS1                             \\
 ``NLDAS2''                    & NLDAS2                             \\
 ``GLDAS''                     & GLDAS                              \\
 ``GFS''                       & GFS                                \\
 ``MERRA-Land''                & MERRA-Land                         \\
 ``MERRA2''                    & MERRA2                             \\
 ``CMAP''                      & CMAP                               \\
 ``TRMM 3B42RT''               & TRMM 3B42RT                        \\
 ``TRMM 3B42RTV7''             & TRMM 3B42RTV7                      \\
 ``TRMM 3B42V6''               & TRMM 3B42V6                        \\
 ``TRMM 3B42V7''               & TRMM 3B42V7                        \\
 ``CPC CMORPH''                & CMORPH from CPC                    \\
 ``CPC STAGEII''               & STAGEII from CPC                   \\
 ``CPC STAGEIV''               & STAGEIV from CPC                   \\
 ``NARR''                      & North American Regional Reanalysis \\
 ``RFE2(daily)''               & Daily rainfall estimator           \\
 ``RFE2(GDAS bias-corrected)'' & RFE2 data bias corrected to GDAS   \\
 ``CEOP''                      & CEOP                               \\
 ``SCAN''                      & SCAN                               \\
 ``GDAS(LSWG)''                & GDAS data for LSWG project         \\
 ``AGRMET radiation''          & AGRMET radiation                   \\
 ``Bondville''                 & Bondville site data                \\
 ``SNOTEL''                    & SNOTEL data                        \\
 ``COOP''                      & COOP data                          \\
 ``Rhone AGG''                 & Rhone AGG forcing data             \\
 ``VIC processed forcing''     & VIC processed forcing              \\
 ``PALS station forcing''      & PALS station forcing               \\
 ``PILDAS''                    & PILDAS                             \\
 ``PET USGS''                  & USGS PET 1.0 deg                   \\
 ``NAM242''                    & NAM 242 AWIPS Grid -- Over Alaska  \\
 ``WRFout''                    & WRF output                         \\
 ``RDHM.3.5.6''                & RDHM 3.5.6 (SACHTET and Snow17)    \\
 ``LDT-generated''             & LDT-generated forcing files        \\
 \end{tabular}
 

 \begin{Verbatim}[frame=single]
Met forcing sources: GDAS
 \end{Verbatim}

 
 \var{Topographic correction method (met forcing):} specifies whether 
 to use elevation correction for base forcing.
 Acceptable values are:

 \begin{tabular}{ll}
 Value          & Description                                       \\
 ``none''         & Do not apply topographic correction for forcing \\
 ``lapse-rate''   & Use lapse rate correction for forcing           \\
 ``slope-aspect'' & Apply slope-aspect correction for forcing       \\
 \end{tabular}
 

 \begin{Verbatim}[frame=single]
Topographic correction method (met forcing):  "lapse-rate"
 \end{Verbatim}

 
 \var{Enable spatial downscaling of precipitation:} specifies whether
 to use spatial downscaling of precipitaion.
 Acceptable values are:

 \begin{tabular}{ll}
 Value & Description                       \\
 0     & Do not enable spatial downscaling \\
 1     & Enable spatial downscaling        \\
 \end{tabular}
 

 \begin{Verbatim}[frame=single]
Enable spatial downscaling of precipitation: 0
 \end{Verbatim}

 
 \var{Spatial interpolation method (met forcing):}
 specifies the type of interpolation scheme to
 apply to the met forcing data.
 Acceptable values are:

 \begin{tabular}{ll}
 Value               & Description                              \\
 ``bilinear''        & bilinear scheme                          \\
 ``budget-bilinear'' & conservative scheme                      \\
 ``neighbor''        & neighbour scheme                         \\
 \end{tabular}

 Bilinear interpolation uses 4 neighboring points to compute the
 interpolation weights. The conservative approach uses 25 neighboring
 points. If the conservative
 option is turned on, it is used to interpolate the precip field only (to
 conserve water). Other fields will still be interpolated with the
 bilinear option.
 

 \begin{Verbatim}[frame=single]
Spatial interpolation method (met forcing): bilinear
 \end{Verbatim}

 
 \var{Spatial upscaling method (met forcing):}
 specifies the type of upscaling scheme to
 apply to the met forcing data.
 Acceptable values are:

 \begin{tabular}{ll}
 Value               & Description                              \\
 ``average''         & averaging scheme                         \\
 \end{tabular}

 Please note that not all met forcing readers support upscaling
 of the met forcing data.
 

 \begin{Verbatim}[frame=single]
Spatial upscaling method (met forcing): average
 \end{Verbatim}

 
 \var{Temporal interpolation method (met forcing):} 
 specifies the type of temporal interpolation scheme to 
 apply to the met forcing data.
 Acceptable values are:

 \begin{tabular}{ll}
 Value     & Description                      \\
 linear    & linear scheme                    \\
 trilinear & uber next scheme                 \\
 \end{tabular}

 The linear temporal interpolation method computes the temporal weights
 based on two points. Ubernext computes weights based on three points.
 Currently the ubernext option is implemented only for the GSWP forcing.
 

 \begin{Verbatim}[frame=single]
Temporal interpolation method (met forcing): linear
 \end{Verbatim}

 
 \subsection{Runtime options} \label{ssec:runtimeopts}
 

 
 \var{Forcing variables list file:} specifies the file containing
 the list of forcing variables to be used. Please refer to the 
 sample forcing\_variables.txt (Section~\ref{sec:forcingvars})
 file for a complete specification description. 
 

 \begin{Verbatim}[frame=single]
Forcing variables list file:     ./input/forcing_variables.txt
 \end{Verbatim}

 
 \var{Output methodology:} specifies whether to write output as a
 1-D array containing only land points or as a 2-D array containing
 both land and water points. 1-d tile space includes the subgrid 
 tiles and ensembles. 1-d grid space includes a vectorized, land-only
 grid-averaged set of values. 
 Acceptable values are:

 \begin{tabular}{ll}
 Value          & Description                         \\
 ``none''         & Do not write output               \\
 ``1d tilespace'' & Write output in a 1-D tile domain \\
 ``2d gridspace'' & Write output in a 2-D grid domain \\
 ``1d gridspace'' & Write output in a 1-D grid domain \\
 \end{tabular}
 

 \begin{Verbatim}[frame=single]
Output methodology: "2d gridspace"
 \end{Verbatim}

 
 \var{Output model restart files:} specifies whether to write
 model restart files.
 Acceptable values are:

 \begin{tabular}{ll}
 Value & Description                \\
 0     & Do not write restart files \\
 1     & Write restart files        \\
 \end{tabular}
 

 \begin{Verbatim}[frame=single]
Output model restart files:   1
 \end{Verbatim}

 
 \var{Output data format:} specifies the format of the model output data.
 Acceptable values are:

 \begin{tabular}{ll}
 Value      & Description                       \\
 ``binary'' & Write output in binary format     \\
 ``grib1''  & Write output in GRIB-1 format     \\
 ``netcdf'' & Write output in netCDF format     \\
 \end{tabular}
 

 \begin{Verbatim}[frame=single]
Output data format: netcdf
 \end{Verbatim}

 
 \var{Output naming style:} specifies the style of the model output
 names and their organization.
 Acceptable values are:

 \begin{tabular}{ll}
 Value                 & Description                       \\
 ``2 level hierarchy'' & 2 levels of hierarchy             \\
 ``3 level hierarchy'' & 3 levels of hierarchy             \\
 ``4 level hierarchy'' & 4 levels of hierarchy             \\
 ``WMO convention''    & WMO convention for weather codes  \\
 \end{tabular}
 

 \begin{Verbatim}[frame=single]
Output naming style: "3 level hierarchy"
 \end{Verbatim}

 
 \var{Output GRIB Table Version:} specifies GRIB table version.

 \var{Output GRIB Center Id:} specifies GRIB center id.

 \var{Output GRIB Subcenter Id:} specifies GRIB sub-center id.

 \var{Output GRIB Grid Id:} specifies GRIB grid id.

 \var{Output GRIB Process Id:} specifies GRIB process id.

 

 \begin{Verbatim}[frame=single]
Output GRIB Table Version: 130
Output GRIB Center Id:     173
Output GRIB Subcenter Id:    4
Output GRIB Grid Id:        11
Output GRIB Process Id:      1
 \end{Verbatim}


 
 \var{Start mode:} specifies if a restart mode is being used. 
 Acceptable values are:

 \begin{tabular}{ll}
 Value     & Description                                           \\
 restart   & A restart mode is being used                          \\
 coldstart & A cold start mode is being used, no restart file read \\
 \end{tabular}

 When the cold start option is specified, the program is initialized
 using the LSM-specific initial conditions (typically assumed uniform
 for all tiles). When a restart mode is used, it is assumed that a 
 corresponding restart file is provided depending upon which LSM is 
 used. The user also needs to make sure that the ending time of the
 simulation is greater than model time when the restart file was 
 written.  
 

 \begin{Verbatim}[frame=single]
Start mode: coldstart
 \end{Verbatim}

 
 The start time is specified in the following format: 

 \begin{tabular}{lll}
 Variable & Value & Description                      \\
 \var{Starting year:} & integer 2001 -- present & 
                        specifying starting year     \\
 \var{Starting month:} & integer 1 -- 12 & 
                        specifying starting month    \\
 \var{Starting day:} & integer 1 -- 31 & 
                       specifying starting day       \\
 \var{Starting hour:} & integer 0 -- 23 &
                        specifying starting hour     \\
 \var{Starting minute:} & integer 0 -- 59 &
                          specifying starting minute \\
 \var{Starting second:} & integer 0 -- 59 &
                          specifying starting second \\
 \end{tabular}
 

 \begin{Verbatim}[frame=single]
Starting year:                             2002
Starting month:                            10
Starting day:                              29
Starting hour:                             1
Starting minute:                           0
Starting second:                           0
 \end{Verbatim}

 
 The end time is specified in the following format: 

 \begin{tabular}{lll}
 Variable & Value & Description                  \\
 \var{Ending year:} & integer 2001 -- present & 
                      specifying ending year     \\
 \var{Ending month:} & integer 1 -- 12 & 
                       specifying ending month   \\
 \var{Ending day:} & integer 1 -- 31 & 
                     specifying ending day       \\
 \var{Ending hour:} & integer 0 -- 23 &
                      specifying ending hour     \\
 \var{Ending minute:} & integer 0 -- 59 &
                        specifying ending minute \\
 \var{Ending second:} & integer 0 -- 59 &
                        specifying ending second \\
 \end{tabular}
 

 \begin{Verbatim}[frame=single]
Ending year:                               2002
Ending month:                              10
Ending day:                                31
Ending hour:                               1
Ending minute:                             0
Ending second:                             0
 \end{Verbatim}

 
 \var{LIS time window interval:} specifies the interval at which the
 LIS run loop cycles, used in the ``ensemble smoother'' running mode.
 

 \begin{Verbatim}[frame=single]
LIS time window interval:
 \end{Verbatim}

 
 \var{Undefined value:} specifies the undefined value.
 The default is set to -9999.
 

 \begin{Verbatim}[frame=single]
Undefined value: -9999
 \end{Verbatim}

 
 \var{Output directory:} specifies the name of the top-level output
 directory.
 Acceptable values are any 40 character string.
 The default value is set to OUTPUT.
 For simplicity, throughout the rest of this document, this top-level
 output directory shall be referred to by its default name,
 \file{\$WORKING/LIS/OUTPUT}.
 

 \begin{Verbatim}[frame=single]
Output directory: OUTPUT
 \end{Verbatim}

 
 \var{Diagnostic output file:} specifies the name of run time
 diagnostic file. 
 Acceptable values are any 40 character string.
 

 \begin{Verbatim}[frame=single]
Diagnostic output file: lislog
 \end{Verbatim}

 
 \var{Number of ensembles per tile:} specifies the number of 
 ensembles of tiles to be used. The value should be greater than
 or equal to 1. 
 

 \begin{Verbatim}[frame=single]
Number of ensembles per tile: 1
 \end{Verbatim}

 
 The following options are used for subgrid tiling based on vegetation

 \var{Maximum number of surface type tiles per grid:} defines the
 maximum surface type tiles per grid (this can be as many as the total
 number of vegetation types). 
 

 \begin{Verbatim}[frame=single]
Maximum number of surface type tiles per grid: 1
 \end{Verbatim}

 
 \var{Minimum cutoff percentage (surface type tiles):} defines the
 smallest percentage of a cell for which to create a tile.
 The percentage value is expressed as a fraction.
 

 \begin{Verbatim}[frame=single]
Minimum cutoff percentage (surface type tiles): 0.05
 \end{Verbatim}

 
 \var{Maximum number of soil texture tiles per grid:} defines the
 maximum soil texture tiles per grid (this can be as many as the total
 number of soil texture types). 
 

 \begin{Verbatim}[frame=single]
Maximum number of soil texture tiles per grid: 1
 \end{Verbatim}

 
 \var{Minimum cutoff percentage (soil texture tiles):} defines the
 smallest percentage of a cell for which to create a tile.
 The percentage value is expressed as a fraction.
 

 \begin{Verbatim}[frame=single]
Minimum cutoff percentage (soil texture tiles): 0.05
 \end{Verbatim}

 
 \var{Maximum number of soil fraction tiles per grid:} defines the
 maximum soil fraction tiles per grid (this can be as many as the total
 number of soil fraction types). 
 

 \begin{Verbatim}[frame=single]
Maximum number of soil fraction tiles per grid: 1
 \end{Verbatim}

 
 \var{Minimum cutoff percentage (soil fraction tiles):} defines the
 smallest percentage of a cell for which to create a tile.
 The percentage value is expressed as a fraction.
 

 \begin{Verbatim}[frame=single]
Minimum cutoff percentage (soil fraction tiles): 0.05
 \end{Verbatim}

 
 \var{Maximum number of elevation bands per grid:} defines the
 maximum elevation bands per grid (this can be as many as the total
 number of elevation band types). 
 

 \begin{Verbatim}[frame=single]
Maximum number of elevation bands per grid: 1
 \end{Verbatim}

 
 \var{Minimum cutoff percentage (elevation bands):} defines the
 smallest percentage of a cell for which to create a tile.
 The percentage value is expressed as a fraction.
 

 \begin{Verbatim}[frame=single]
Minimum cutoff percentage (elevation bands): 0.05
 \end{Verbatim}

 
 \var{Maximum number of slope bands per grid:} defines the
 maximum slope bands per grid (this can be as many as the total
 number of slope band types). 
 

 \begin{Verbatim}[frame=single]
Maximum number of slope bands per grid: 1
 \end{Verbatim}

 
 \var{Minimum cutoff percentage (slope bands):} defines the
 smallest percentage of a cell for which to create a tile.
 The percentage value is expressed as a fraction.
 

 \begin{Verbatim}[frame=single]
Minimum cutoff percentage (slope bands): 0.05
 \end{Verbatim}

 
 \var{Maximum number of aspect bands per grid:} defines the
 maximum aspect bands per grid (this can be as many as the total
 number of aspect band types). 
 

 \begin{Verbatim}[frame=single]
Maximum number of aspect bands per grid: 1
 \end{Verbatim}

 
 \var{Minimum cutoff percentage (aspect bands):} defines the
 smallest percentage of a cell for which to create a tile.
 The percentage value is expressed as a fraction.
 

 \begin{Verbatim}[frame=single]
Minimum cutoff percentage (aspect bands): 0.05
 \end{Verbatim}

 
 This section specifies the 2-d layout of the processors in a 
 parallel processing environment. The user can specify the number of 
 processors along the east-west dimension and north-south dimension
 using \var{Number of processors along x:} and \var{Number of processors
 along y:}, respectively. Note that the layout of processors should 
 match the total number of processors used. For example, if 8 
 processors are used, the layout can be specified as 1x8, 2x4, 4x2, or
 8x1. Further, this section also allows the specification of halos 
 around the domains on each processor using \var{Halo size along x:} 
 and \var{Halo size along y:}.
 

 \begin{Verbatim}[frame=single]
Number of processors along x:    2
Number of processors along y:    2
Halo size along x: 0
Halo size along y: 0
 \end{Verbatim}

 
 \var{Routing model:} specifies the routing model to run.
 Acceptable values are:

 \begin{tabular}{ll}
 Value            & Description                \\
 none             & do not use a routing model \\
 ``NLDAS router'' & use the NLDAS router       \\
 ``HYMAP router'' & use the HYMAP router       \\
 \end{tabular}

 \var{Number of application models:} specifies the number
 of application models to run.
 

 \begin{Verbatim}[frame=single]
Routing model: none
Radiative transfer model: none
Number of application models: 0
 \end{Verbatim}

 
 \subsection{Data assimilation} \label{ssec:dataassim}
 This section specifies the choice of data assimilation options. 
 

 
 \var{Number of data assimilation instances:} specifies the 
 number of data assimilation instances. Valid values are 
 0 (no assimilation) or higher.
 

 \begin{Verbatim}[frame=single]
Number of data assimilation instances: 0
 \end{Verbatim}

 
 \var{Data assimilation algorithm:} specifies the choice of data
 assimilation algorithms. 
 Acceptable values are:

 \begin{tabular}{ll}
 Value                & Description                                    \\
 ``none''             & None                                           \\
 ``Direct insertion'' & Direct Insertion                               \\
 ``EnKF''             & GMAO EnKF data assimilation                    \\
 ``EnKS''             & GRACE ensemble Kalman filter data assimilation \\
 \end{tabular}
 

 \begin{Verbatim}[frame=single]
Data assimilation algorithm: none
 \end{Verbatim}

 
 \var{Data assimilation set:} specifies the ``assimilation set'', 
 which is the instance related to the assimilation 
 of a particular observation. 
 Acceptable values are:

 \begin{tabular}{ll}
 Value                          & Description               \\
 ``none''                       & none                      \\
 ``AMSR-E(NASA) soil moisture'' & AMSRE L3 soil moisture daily gridded data (HDF format) \\
 ``AMSR-E(LPRM) soil moisture'' & AMSRE L3 soil moisture daily gridded data (HDF format) \\
 ``ECV soil moisture''          & ECV soil moisture         \\
 ``Windsat''                    & Windsat                   \\
 ``ANSA SCF''                   & ANSA SCF                  \\
 ``PMW snow''                   & PMW-based SWE or snow depth  \\
 ``MODIS SCF''                  & MODIS SCF                 \\
 ``GRACE TWS''                  & GRACE TWS                 \\
 ``SMOPS soil moisture''        & SMOPS soil moisture    \\
 \end{tabular}
 

 \begin{Verbatim}[frame=single]
Data assimilation set: none
 \end{Verbatim}

 

 \var{Data assimilation exclude analysis increments:} specifies whether
 the analysis increments
 are to be skipped. This option is typically used along with the dynamic
 bias estimation algorithm. The user can choose to apply only the bias
 increments or both the bias increments and analysis increments. 
 Acceptable values are:

 \begin{tabular}{ll}
 Value & Description                                       \\
 0     &  Apply analysis increments                        \\
 1     &  Do not apply analysis increments                 \\
 \end{tabular}
 

 \begin{Verbatim}[frame=single]
Data assimilation exclude analysis increments:     0
 \end{Verbatim}

 
 \var{Data assimilation output interval for diagnostics:} specifies
 the output diagnostics interval.

 See Section \ref{ssec:timeinterval} for a description
 of how to specify a time interval.
 

 \begin{Verbatim}[frame=single]
Data assimilation output interval for diagnostics: 1da
 \end{Verbatim}

 
 \var{Data assimilation number of observation types:} specifies the
 number of opervation species/types used in the assimilation.
 

 \begin{Verbatim}[frame=single]
Data assimilation number of observation types: 0
 \end{Verbatim}

 
 \var{Data assimilation output ensemble members:} specifies whether
 to output the ensemble members.
 Acceptable values are:

 \begin{tabular}{ll}
 Value & Description                         \\
 0     &  Do not output the ensemble members \\
 1     &  Output the ensemble members        \\
 \end{tabular}
 

 \begin{Verbatim}[frame=single]
Data assimilation output ensemble members: 0
 \end{Verbatim}

 
 \var{Data assimilation output processed observations:} specifies
 whether the processed, quality-controlled
 observations are to be written (Note that a corresponding observation
 plugin routine needs to be implemented).
 Acceptable values are:

 \begin{tabular}{ll}
 Value & Description                               \\
 0     &  Do not output the processed observations \\
 1     &  Output the processed observations        \\
 \end{tabular}
 

 \begin{Verbatim}[frame=single]
Data assimilation output processed observations: 0
 \end{Verbatim}

 
 \var{Data assimilation output innovations:} specifies whether
 a binary output of the normalized innovations is to be written. 
 Acceptable values are:

 \begin{tabular}{ll}
 Value & Description                    \\
 0     &  Do not output the innovations \\
 1     &  Output the innovations        \\
 \end{tabular}
 

 \begin{Verbatim}[frame=single]
Data assimilation output innovations: 0
 \end{Verbatim}

 
 \var{Data assimilation use a trained forward model:} specifies whether
 to use a trained forward model.
 Acceptable values are:

 \begin{tabular}{ll}
 Value & Description                         \\
 0     &  Do not use a trained forward model \\
 1     &  Use a trained forward model        \\
 \end{tabular}

 \var{Data assimilation trained forward model output file:} specifies
 the name of the output file for the trained forward model.
 The training is done by LDT, and thus, this file is produced by LDT.
 

 \begin{Verbatim}[frame=single]
Data assimilation use a trained forward model: 0
Data assimilation trained forward model output file: none
 \end{Verbatim}

 
 Bias estimation

 \var{Bias estimation algorithm:} specifies the dynamic bias estimation
 algorithm to use.
 Acceptable values are:

 \begin{tabular}{ll}
 Value                        & Description                       \\
 ``none''                     & No dynamic bias estimation        \\
 ``Adaptive bias correction'' & NASA GMAO dynamic bias estimation \\
 \end{tabular}
 

 \begin{Verbatim}[frame=single]
Bias estimation algorithm: none
 \end{Verbatim}

 
 \var{Bias estimation attributes file:} ASCII file that
 specifies the attributes of the bias estimation. A
 sample file is shown below, which lists the variable
 name first. This is followed by the nparam value 
 (0-no bias correction, 1- constant bias correction without
 diurnal cycle, 3- diurnal sine/cosine bias correction, 
 5 - semi-diurnal sine/cosine bias correction, 
 2-``time of day'' bias correction with 2 separate bias
 estimates per day, 4 - ``time of day'' bias correction with 
 4 separate estimates per day, 8 - ``time of day'' bias 
 correction with 8 separate bias estimates per day), 
 tconst (which describes the time scale relative to the 
 temporal spacing of the observations), and trelax
 (which specifies temporal relaxation parameter, in seconds)

 \#nparam  tconst trelax \\
 \indent Soil Temperature        \\
 \indent 1.0    0.05    86400.0
 

 \begin{Verbatim}[frame=single]
Bias estimation attributes file:
 \end{Verbatim}

 
 \var{Bias estimation restart output frequency:} Specifies the frequency
 of bias restart files.

 See Section \ref{ssec:timeinterval} for a description
 of how to specify a time interval.
 

 \begin{Verbatim}[frame=single]
Bias estimation restart output frequency: 1da
 \end{Verbatim}

 
 \var{Bias estimation start mode:} This option specifies whether the
 bias parameters are to be read from a checkpoint file. 
 Acceptable values are:

 \begin{tabular}{ll}
 Value & Description                    \\
 none  & Do not use a bias restart file \\
 read  & Use a bias restart file        \\
 \end{tabular}
 

 \begin{Verbatim}[frame=single]
Bias estimation start mode: none
 \end{Verbatim}

 
 \var{Bias estimation restart file:} Specifies the restart file to be
 used for initializing bias parameters 
 

 \begin{Verbatim}[frame=single]
Bias estimation restart file: none
 \end{Verbatim}

 
 \var{Perturbations start mode:} specifies if the perturbations settings
 should be read from a restart file.
 Acceptable values are: 

 \begin{tabular}{ll}
 Value     & Description       \\
 coldstart & None (cold start) \\
 restart   & Use restart file  \\
 \end{tabular}
 

 \begin{Verbatim}[frame=single]
Perturbations start mode: coldstart
 \end{Verbatim}

 
 \var{Apply perturbation bias correction:} specifies whether
 to apply the Ryu et al. algorithm, (JHM 2009), to forcing and
 model states to avoid undesirable biases resulting from perturbations.
 Acceptable values are: 

 \begin{tabular}{ll}
 Value & Description  \\
 0     & Do not apply \\
 1     & Apply        \\
 \end{tabular}
 

 \begin{Verbatim}[frame=single]
Apply perturbation bias correction:
 \end{Verbatim}

 
 \var{Perturbations restart output interval:} specifies the
 perturbations restart output writing interval.

 See Section \ref{ssec:timeinterval} for a description
 of how to specify a time interval.
 

 \begin{Verbatim}[frame=single]
Perturbations restart output interval: 1da
 \end{Verbatim}

 
 \var{Perturbations restart filename:} specifies the name of the 
 restart file, which is used to initialize perturbation settings
 if a cold start option is not employed. 
 

 \begin{Verbatim}[frame=single]
Perturbations restart filename: none
 \end{Verbatim}

 \var{Forcing perturbation algorithm:} specifies the algorithm for
 perturbing the forcing variables.
 Acceptable values are: 

 \begin{tabular}{ll}
 Value           & Description                 \\
 ``none''        & None                        \\
 ``GMAO scheme'' & GMAO perturbation algorithm \\
 \end{tabular}
 

 \begin{Verbatim}[frame=single]
Forcing perturbation algorithm: none
 \end{Verbatim}

 
 \var{Forcing perturbation frequency:} specifies the forcing
 perturbation interval.

 See Section \ref{ssec:timeinterval} for a description
 of how to specify a time interval.
 

 \begin{Verbatim}[frame=single]
Forcing perturbation frequency: 1hr
 \end{Verbatim}

 
 \var{Forcing attributes file:} ASCII file that
 specifies the attributes of the forcing (for perturbations)
 A sample file is shown below, which lists 3 forcing 
 variables. For each variable, the name of the variable is 
 specified first, followed by the min and max values in the 
 next line. This is repeated for each additional variable.  

   \#varmin  varmax                       \\
   \indent Incident Shortwave Radiation Level 001 \\
   \indent 0.0      1300.0                        \\
   \indent Incident Longwave Radiation Level 001  \\
   \indent -50.0    800.0                         \\
   \indent Rainfall Rate Level 001                \\
   \indent 0.0      0.001
 

 \begin{Verbatim}[frame=single]
Forcing attributes file: none
 \end{Verbatim}

 
 \var{Forcing perturbation attributes file:} ASCII file that
 specifies the attributes of the forcing perturbations. 
 A sample file is shown below, which lists 3 forcing 
 variables. There are three lines of specifications for 
 each variable. The first line specifies the name of the 
 variable. The second line specifies the perturbation type
 (0-additive, 1-multiplicative) and the perturbation type 
 for standard deviation (0-additive, 1-multiplicative). The 
 third line specifies the following values in that order: 
 standard deviation of perturbations, coefficient of 
 standard deviation (if perturbation type for standard 
 deviation is 1),standard normal max, whether to enable
 zero mean in perturbations, temporal correlation scale 
 (in seconds), x and y -correlations and finally the cross
 correlations with other variables. 

 \#ptype   std    std\_max   zeromean  tcorr  xcorr ycorr ccorr \\
 \indent Incident Shortwave Radiation Level 001                 \\
 \indent 1  0                                                   \\
 \indent 0.50     2.5     1     86400     0     0     1.0  -0.5  -0.8 \\
 \indent Incident Longwave Radiation Level 001                        \\
 \indent 0  1                                                         \\
 \indent 50.0     0.2     2.5   1   86400  0    0    -0.5   1.0  0.5  \\
 \indent Rainfall Rate Level 001                                      \\
 \indent 1  0                                                         \\
 \indent 0.50     2.5     1       86400  0     0     0.8   0.5  1.0
 

 \begin{Verbatim}[frame=single]
Forcing perturbation attributes file: none
 \end{Verbatim}

 
 \var{State perturbation algorithm:} specifies the algorithm for
 perturbing the state prognostic variables.
 Acceptable values are: 

 \begin{tabular}{ll}
 Value           & Description                 \\
 ``none''        & None                        \\
 ``GMAO scheme'' & GMAO perturbation algorithm \\
 \end{tabular}
 

 \begin{Verbatim}[frame=single]
State perturbation algorithm: none
 \end{Verbatim}

 
 \var{State perturbation frequency:} specifies the prognostic variable
 perturbation interval.

 See Section \ref{ssec:timeinterval} for a description
 of how to specify a time interval.
 

 \begin{Verbatim}[frame=single]
State perturbation frequency: 1hr
 \end{Verbatim}

 
 \var{State attributes file:} ASCII file that specifies
 the attributes of the prognostic variables. 
 A sample file is shown below, which lists 2 model state
 variables. For each variable, the name of the variable is 
 specified first, followed by the min and max values in the 
 next line. This is repeated for each additional variable.  

 \#name  varmin  varmax \\
 \indent SWE                    \\
 \indent 0.0   100.0            \\
 \indent Snowdepth              \\
 \indent 0.0   100.0
 

 \begin{Verbatim}[frame=single]
State attributes file: none
 \end{Verbatim}

 
 \var{State perturbation attributes file:} ASCII file that specifies
 the attributes of the prognostic variable perturbations. 
 A sample file is provided below, which follows the same format 
 as that of the forcing perturbations attributes file: 

 \#perttype  std    std\_max   zeromean  tcorr  xcorr ycorr ccorr \\
 \indent SWE                                                      \\
 \indent 1    0                                                   \\
 \indent 0.01   2.5       1        10800   0    0    1.0  0.9     \\
 \indent Snowdepth                                                \\
 \indent 1    0                                                   \\
 \indent 0.02    2.5       1        10800   0    0    0.9  1.0
 

 \begin{Verbatim}[frame=single]
State perturbation attributes file: none
 \end{Verbatim}

 
 \var{Observation perturbation algorithm:} specifies the algorithm
 for perturbing the observations.
 Acceptable values are: 

 \begin{tabular}{ll}
 Value           & Description                 \\
 ``none''        & None                        \\
 ``GMAO scheme'' & GMAO perturbation algorithm \\
 \end{tabular}
 

 \begin{Verbatim}[frame=single]
Observation perturbation algorithm: none
 \end{Verbatim}

 
 \var{Observation perturbation frequency:} specifies the observation
 perturbation interval.

 See Section \ref{ssec:timeinterval} for a description
 of how to specify a time interval.
 

 \begin{Verbatim}[frame=single]
Observation perturbation frequency: 1hr
 \end{Verbatim}

 
 \var{Observation attributes file:} ASCII file that
 specifies the attributes of the observation variables. 
 A sample file is provided below, which follows the same format 
 as that of the forcing attributes file and state attributes file. 

 \#error rate varmin  varmax \\ 
 \indent ANSA SWE                    \\
 \indent 10.0   0.01   500 
  
 

 \begin{Verbatim}[frame=single]
Observation attributes file: none
 \end{Verbatim}

 
 \var{Observation perturbation attributes file:} ASCII file that
 specifies the attributes of the observation variable perturbations. 
 A sample file is provided below, which follows the same format 
 as that of the forcing perturbations attributes file: 

 \#perttype  std    std\_max   zeromean  tcorr  xcorr ycorr ccorr \\
 \indent ANSA SWE                                                 \\
 \indent 0     10      2.5        1        10800       0    0    1 
 

 \begin{Verbatim}[frame=single]
Observation perturbation attributes file: none
 \end{Verbatim}

 
 \var{IMS data directory:} specifes the location of the IMS data.
 

 \begin{Verbatim}[frame=single]
IMS data directory:
 \end{Verbatim}







 
 \subsubsection{AMSR-E (NASA) soil moisture assimilation}
 \label{sssec:nasaamsreda}
 

 
 \var{NASA AMSR-E soil moisture data directory:} specifies the directory
 for the AMSR-E (NASA/NSIDC) soil moisture data.

 \var{NASA AMSR-E soil moisture scale observations:} specifies if the
 observations are to be rescaled (using CDF matching).

 \var{NASA AMSR-E soil moisture model CDF file:} specifies the
 name of the model CDF file (observations will be scaled into this
 climatology).

 \var{NASA AMSR-E soil moisture observation CDF file:} specifies the
 name of the observatoin CDF file.

 \var{NASA AMSR-E soil moisture number of bins in the CDF:} specifies the
 number of bins in the CDF.
 

 \begin{Verbatim}[frame=single]
NASA AMSR-E soil moisture data directory:       'input'
NASA AMSR-E soil moisture scale observations:   1
NASA AMSR-E soil moisture model CDF file:       lsm_cdf.nc
NASA AMSR-E soil moisture observation CDF file: obs_cdf.nc
NASA AMSR-E soil moisture number of bins in the CDF: 100 
 \end{Verbatim}

 
 \subsubsection{AMSR-E (LPRM) soil moisture assimilation}
 \label{sssec:lprmamsreda}
 

 
 \var{AMSR-E(LPRM) soil moisture data directory:} specifies the
 directory for the AMSR-E (LPRM) soil moisture data.

 \var{AMSR-E(LPRM) soil moisture use raw data:} specifies if the 
 the raw fields (in wetness units) or scaled fields
 (in volumetric units) are to be used.

 \var{AMSR-E(LPRM) scale observations:} specifies if the
 observations are to be rescaled (using CDF matching).

 \var{AMSR-E(LPRM) use scaled standard deviation model:} specifies if
 the observation error standard deviation is to be scaled using
 model and observation standard deviation.

 \var{AMSR-E(LPRM) model CDF file:} specifies the
 name of the model CDF file (observations will be scaled into this
 climatology).

 \var{AMSR-E(LPRM) observation CDF file:} specifies the
 name of the observatoin CDF file.

 \var{AMSR-E(LPRM) soil moisture number of bins in the CDF:}
 specifies the number of bins in the CDF.
 

 \begin{Verbatim}[frame=single]
AMSR-E(LPRM) soil moisture data directory:       'input'
AMSR-E(LPRM) soil moisture use raw data:          0 
AMSR-E(LPRM) scale observations:                  1
AMSR-E(LPRM) use scaled standard deviation model: 1
AMSR-E(LPRM) model CDF file:                      lsm_cdf.nc
AMSR-E(LPRM) observation CDF file:                obs_cdf.nc
AMSR-E(LPRM) soil moisture number of bins in the CDF: 100 
 \end{Verbatim}

 
 \subsubsection{ECV soil moisture assimilation}
 \label{sssec:ecvsmda}
 

 
 \var{ECV soil moisture data directory:} specifies the directory
 for the ECV soil moisture data.

 \var{ECV scale observations:} specifies if the
 observations are to be rescaled (using CDF matching).

 \var{ECV use scaled standard deviation model:} specifies if
 the observation error standard deviation is to be scaled using
 model and observation standard deviation.

 \var{ECV model CDF file:} specifies the
 name of the model CDF file (observations will be scaled into this
 climatology).

 \var{ECV observation CDF file:} specifies the
 name of the observatoin CDF file.

 \var{ECV soil moisture number of bins in the CDF:} specifies the
 number of bins in the CDF.
 

 \begin{Verbatim}[frame=single]
ECV soil moisture data directory:       'input'
ECV scale observations:                  1
ECV use scaled standard deviation model: 1
ECV model CDF file:                      lsm_cdf.nc
ECV observation CDF file:                obs_cdf.nc
ECV soil moisture number of bins in the CDF: 100 
 \end{Verbatim}

 
 \subsubsection{WindSat soil moisture assimilation}
 \label{sssec:windsatsmda}
 

 
 \var{WindSat soil moisture data directory:} specifies the directory
 for the WindSat soil moisture data.

 \var{WindSat scale observations:} specifies if the
 observations are to be rescaled (using CDF matching).

 \var{WindSat model CDF file:} specifies the
 name of the model CDF file (observations will be scaled into this
 climatology).

 \var{WindSat observation CDF file:} specifies the
 name of the observatoin CDF file.

 \var{WindSat number of bins in the CDF:} specifies the
 number of bins in the CDF.
 

 \begin{Verbatim}[frame=single]
WindSat soil moisture data directory:       'input'
WindSat scale observations:                  1
WindSat model CDF file:                      lsm_cdf.nc
WindSat observation CDF file:                obs_cdf.nc
WindSat number of bins in the CDF:           100
 \end{Verbatim}




 
 \subsubsection{ANSA Snow Covered Fraction (SCF) Assimilation}
 \label{sssec:ansascfda}
 

 
 \var{ANSA SCF data directory:} specifies the directory for the
 ANSA SCA data.

 \var{ANSA SCF lower left lat:} specifies the lower left latitude
 of the ANSA domain. (cylindrical latitude/longitude projection)

 \var{ANSA SCF lower left lon:} specifies the lower left longitude
 of the ANSA domain. (cylindrical latitude/longitude projection)

 \var{ANSA SCF upper right lat:} specifies the upper right latitude
 of the ANSA domain. (cylindrical latitude/longitude projection)

 \var{ANSA SCF upper right lon:} specifies the upper right longitude
 of the ANSA domain. (cylindrical latitude/longitude projection)

 \var{ANSA SCF resolution (dx):} specifies the resolution of the
 of the ANSA domain along the east-west direction.

 \var{ANSA SCF resolution (dy):} specifies the resolution of the
 of the ANSA domain along the north-south direction.

 \var{ANSA SCF local time for assimilation:} specifies the local time
 for performing the ANSA SCF assimilation; LIS will find the closest
 time depending on model timestep.

 \var{ANSA SCF field name:} specifies the name of the SCF field to be
 assimilated in the ANSA SCF data file.

 \var{ANSA SCF file name convention:} specifies the name convention
 of the ANSA SCF file; currently supported: *YYYYMMDD*, *YYYYDOY*.

 \var{ANSA SCF use triangular-shaped observation error:} specifies
 whether to use a triangular-shaped observation error as follows
 (De Lannoy et al., 2012):
 $std = std*scf\_obs$ if $scf\_obs<=50$;
 $std = std*(100-scf\_obs)$ if $scf\_obs>50$;
 otherwise, $std$ remains to be the same as read in from the observation
 perturbation attributes file.

 \var{ANSA SCF using EnKF with DI:} specifies whether to used rule-based
 direct insertion approach to supplement EnKF when model predicts zero
 or full snow cover for all ensemble memebers. The entries after this
 are needed only if 1 is specified here.

 \var{ANSA SCF direct insertion methodology:} specifies which
 rule to use when model predicts snow and observation says no snow.
 Acceptable values are:

 \begin{tabular}{ll}
 Value           & Description                  \\
  ``standard''   & use Rodell and Houser (2004) \\
  ``customized'' & use Liu et al. (2013)        \\
 \end{tabular}

 \var{ANSA SCF amount of SWE (mm) to add to model:} specifies how much
 swe to add to model when observation sees snow while model predicts
 no snow.

 \var{ANSA SCF maximum SWE melt rate (mm/day):} specifies the swe melt rate
 if ``customized'' is chosen for the direction insertion methodology.

 \var{ANSA SCF threshold of model SWE to be removed at once:} specifies
 the threshold of model SWE to be removed when observation says no snow.

 \var{ANSA SCF length of snowmelt period in days:} specifies the length
 of the typical snowmelt period in the region.

 \var{ANSA SCF threshold of observed SCF for snow presence:} specifies
 the threshold of observed SCF for indicating snow presence.

 \var{ANSA SCF threshold of observed SCF for snow non-presence:}
 specifies the threshold of obsered SCF for indicating snow non-presence.

 \var{ANSA SCF threshold of model SWE(mm) for snow non-presence:}
 specifies the threshold of model SWE for indicating snow absence.

 \var{ANSA SCF threshold of observed SCF for non-full snow cover:}
 specifies the threshold of observed SCF which indicates non-full
 snow cover.
 

 \begin{Verbatim}[frame=single]
ANSA SCF data directory:           ./ANSA_SCF_UCO
ANSA SCF lower left lat:           35.025
ANSA SCF lower left lon:           -112.475
ANSA SCF upper right lat:          43.975
ANSA SCF upper right lon:          -105.525
ANSA SCF resolution (dx):          0.05
ANSA SCF resolution (dy):          0.05
ANSA SCF local time for assimilation:                      10.0
ANSA SCF field name:                                       "/ansa_interpsnow"
ANSA SCF file name convention:                             "ansa_all_YYYYMMDD.h5"
ANSA SCF use triangular-shaped observation error:          1
ANSA SCF using EnKF with DI:                               1
ANSA SCF direct insertion methodology:                     "customized"
ANSA SCF amount of SWE (mm) to add to model:               10
ANSA SCF maximum SWE melt rate (mm/day):                   50
ANSA SCF threshold of model SWE to be removed at once:     20
ANSA SCF length of snowmelt period in days:                15
ANSA SCF threshold of observed SCF for snow presence:      0.4
ANSA SCF threshold of observed SCF for snow non-presence:  0.1
ANSA SCF threshold of model SWE(mm) for snow non-presence: 5
ANSA SCF threshold of observed SCF for non-full snow cover: 0.7
 \end{Verbatim}





 
 \subsubsection{MODIS snow cover fraction assimilation}
 \label{sssec:modisscfda}
 

 
 \var{MODIS SCF data directory:} specifies the directory for
 the MODIS snow cover fraction data.

 \var{MODIS SCF use gap filled product:} specifies whether 
 the gap-filled product is to be used (1-use, 0-do not use).

 \var{MODIS SCF cloud threshold:} Cloud cover threshold to be 
 used for screening observations (in percentage).

 \var{MODIS SCF cloud persistence threshold:} Cloud cover persistence
 threshold to be used for screening observations (in days).
 

 \begin{Verbatim}[frame=single]
MODIS SCF data directory: ./MODIS
MODIS SCF use gap filled product: 1
MODIS SCF cloud threshold: 90
MODIS SCF cloud persistence threshold: 3
 \end{Verbatim}

 
 \subsubsection{PMW snow depth or SWE assimilation}
 \label{sssec:pmwsnowdepthda}
 

 
 \var{PMW snow data directory:} specifies the directory for the
 PMW SWE or snow depth data.

 \var{PMW snow data file format (HDF4, HDF-EOS, HDF5):} specifies 
 the file format of the PMW snow data. Currently, three options
 are supported: HDF4, HDF-EOS, and HDF5

 \var{PMW snow data coordindate system (EASE, LATLON):} specifies 
 the coordinate system of the PMW snow data. Currently two options
 are supported: EASE and LATLON.

 \var{PMW snow data variable (SWE, snow depth):} specifies which variable
 to assimilate: SWE or snow depth

 \var{PMW snow data unit (m, cm, mm, inch):} specifies the unit of
 the snow data; currently only units of m, cm, mm, inch are supported

 \var{PMW snow data use flag (1=yes, 0=no):} specifies whether to use
 the data flags that come along with the PMW snow data in the same file

 \var{PMW snow data flag - number of invalid values:} specifies
 the number of invalid values in the flag field of the PMW snow data

 \var{PMW snow data flag - invalid values:} specifies the invalid values
 of the flag field of the PMW snow data

 \var{PMW snow data - number of additional invalid values:} specifies
 the number of additional invalid values in the actual data field of 
 the PMW snow data

 \var{PMW snow data - additional invalid values:} specifies the invalid
 values of the actual data field of the PMW snow data

 \var{PMW snow data - apply min/max mask:} specifies whether to use
 min/max data values for quality control of the PMW snow data

 \var{PMW snow data minimum valid value:} specifies the minimum valid 
 value of the PMW snow data

 \var{PMW snow data maximum valid value:} specifies the maximum valid 
 value of the PMW snow data

 \var{PMW snow data scale factor:} specifies the scale factor of
 the PMW snow data

 \var{PMW snow data file name convention:} specifies the file name
 convention of the PMW snow data; currently only the following two
 formats are supported:
 *YYYYMMDD*  and *YYYYDOY*
 note that the PMW snow reader assumes that the data files are stored
 in corresponding year directory as follows: datadir/YYYY/*YYYMMDD*

 \var{PMW snow data assimilation local time:} specifies the local time
 in hours to apply the assimilation (usually corresponding to the overpass time)

 \var{PMW snow data - apply mask with GVF (1=yes, 0=no):} specifies 
 whether to use greenness vegetation fraction as mask for assimilation;
 1 is suggested unless confidence is high with the PMW snow data (e.g.,
 those that are bias corrected against station data) in dense vegetation
 area.  If ``1'' is chosen, LIS will not assimilate PMW snow data in
 those areas with gvf \textgreater 0.7.

 \var{PMW snow data - apply mask with landcover type (1=yes, 0=no):}
 specifies whether to use landcover type as mask for assimilation.
 If ``1'' is chosen, LIS will not assimilate PMW snow data in areas
 with forest land cover.
 
 \var{PMW snow data - apply mask with LSM temperature (1=yes, 0=no):}
 specifies whether to use model-based temperatures as mask for
 assimilation. if ``1'' is chosen, LIS will not assimilate PMW snow
 data in areas with a skin temperature or surface soil temperature
 higher than 5 degree C according to the LSM. This mask should be
 used with care if the LSM temperatures are known to be biased.

 The following 8 configuration lines are for HDF5+LANTON datasets only
 
 \var{PMW snow data lower left lat:} specifies the lower left latitude
 of the dataset.
 
 \var{PMW snow data lower left lon:} specifies the lower left longitude
 of the dataset.
 
 \var{PMW snow data upper right lat:} specifies the upper right latitude
 of the dataset.
 
 \var{PMW snow data upper right lon:} specifies the upper right longitude
 of the dataset.
 
 \var{PMW snow data resolution (dx):} specifies horizotal resolution dx
 of the dataset.
 
 \var{PMW snow data resolution (dy):} specifies vertical resolution dy
 of the dataset.
 
 \var{PMW (HDF5) snow data field name:} specifies the name of the snow
 data field in the dataset for assimilation.
 
 \var{PMW (HDF5) snow data flag field name:} specifies the name of the
 snow data
 flag field to use as a mask for assimilation; this must be specified if 
 the \var{PMW snow data use flag (1=yes, 0=no):} option is set to 1.

 The following 4 configuration lines are for HDF4+EASE datasets only
 
 \var{PMW (HDF4) snow data NL SDS index (-1, 0, 1, 2, ...):}
 specifies the index of the SDS of the NL grid in the PMW snow data;
 valid index starts from 0; use -1 if no SDS for the NL grid is to be
 assimilated.

 \var{PMW (HDF4) snow data SL SDS index (-1, 0, 1, 2, ...):}
 specifies the index of the SDS of the SL grid in the PMW snow data;
 valid index starts from 0; use -1 if no SDS for the NL grid is to be
 assimilated.

 \var{PMW (HDF4) snow data flag NL SDS index (-1, 0, 1, 2, ...):}
 specifies the index of the flag SDS of the NL grid in the PMW snow data;
 this must be specified if
 the \var{PMW snow data use flag (1=yes, 0=no):} option is set to 1.

 \var{PMW (HDF4) snow data flag SL SDS index (-1, 0, 1, 2, ...):}
 specifies the index of the flag SDS of the SL grid in the PMW snow data;
 this must be specified if
 the \var{PMW snow data use flag (1=yes, 0=no):} option is set to 1.

 The following 6 configuration lines are for HDF-EOS+EASE datasets only

 \var{PMW (HDF-EOS) NL grid name:} specifies the name of the NL grid.

 \var{PMW (HDF-EOS) SL grid name:} specifies the name of the SL grid.

 \var{PMW (HDF-EOS) NL SDS name:} specifies the name of the SDS in the
 NL grid.

 \var{PMW (HDF-EOS) SL SDS name:} specifies the name of the SDS in the
 SL grid.

 \var{PMW (HDF-EOS) NL snow data flag SDS name:} specifies the name of
 the data
 flag SDS in the NL grid; this must be specified if
 the \var{PMW snow data use flag (1=yes, 0=no):} option is set to 1.

 \var{PMW (HDF-EOS) SL snow data flag SDS name:} specifies the name of
 the data
 flag SDS in the SL grid; this must be specified if
 the \var{PMW snow data use flag (1=yes, 0=no):} option is set to 1.
 

 \begin{Verbatim}[frame=single]
# all datasets
PMW snow data directory:                          "./input/ANSA_OI"
PMW snow data file format (HDF4, HDF-EOS, HDF5):  "HDF5"
PMW snow data coordindate system (EASE, LATLON):  "LATLON"
PMW snow data variable (SWE, snow depth):         "snow depth"
PMW snow data unit (m, cm, mm, inch):             "mm"
PMW snow data use flag (1=yes, 0=no):                1
PMW snow data flag - number of invalid values:       2
PMW snow data flag - invalid values:                 -1  0
PMW snow data - number of additional invalid values: 0
PMW snow data - additional invalid values:           494 496 504 596 508 510
PMW snow data - apply min/max mask:                  1 
PMW snow data minimum valid value:                   0
PMW snow data maximum valid value:                   5000 
PMW snow data scale factor:                          1.0 
PMW snow data file name convention:                  "ansa_all_YYYYMMDD.h5"
PMW snow data assimilation local time:               2.0
PMW snow data - apply mask with GVF (1=yes, 0=no):             0
PMW snow data - apply mask with landcover type (1=yes, 0=no):  0
PMW snow data - apply mask with LSM temperature (1=yes, 0=no): 0

# HDF5 & LATLON datasets only
PMW snow data lower left lat:                     50.025
PMW snow data lower left lon:                    -172.975
PMW snow data upper right lat:                    75.725
PMW snow data upper right lon:                   -130.025
PMW snow data resolution (dx):                   0.05
PMW snow data resolution (dy):                   0.05
PMW (HDF5) snow data field name:                 "ansa_swe_depth" 
PMW (HDF5) snow data flag field name:            "ansa_swe_depth_flag"

# HDF4 & EASE datasets only
PMW (HDF4) snow data NL SDS index (-1, 0, 1, 2, ...):       0
PMW (HDF4) snow data SL SDS index (-1, 0, 1, 2, ...):       -1
PMW (HDF4) snow data flag NL SDS index (-1, 0, 1, 2, ...):  1
PMW (HDF4) snow data flag SL SDS index (-1, 0, 1, 2, ...):  -1

# HDF-EOS and EASE datasets only
PMW (HDF-EOS) NL grid name:                  "Northern Hemisphere"
PMW (HDF-EOS) SL grid name:                  "Southern Hemisphere"
PMW (HDF-EOS) NL SDS name:                   "SWE_NorthernDaily"
PMW (HDF-EOS) SL SDS name:                   "SWE_SouthernDaily"  
PMW (HDF-EOS) NL snow data flag SDS name:    "Flags_NorthernDaily"
PMW (HDF-EOS) SL snow data flag SDS name:    "Flags_SouthernDaily"
 \end{Verbatim}

 
 \subsubsection{GRACE TWS Assimilation}
 \label{sssec:gracetwsda}
 

 
 \var{GRACE data directory:} specifies the directory for the
 GRACE TWS data (processed data from LDT).

 \var{GRACE use reported measurement error values:} specifies
 whether to use the spatially distributed reported measurement
 errors in the GRACE data for specifying observation errors.
 Acceptable values are: 

 \begin{tabular}{ll}
 Value & Description \\
 0     & Do not use  \\
 1     & Use         \\
 \end{tabular}
 

 \begin{Verbatim}[frame=single]
GRACE data directory:                  ./GRACEOBS
GRACE use reported measurement error values:
 \end{Verbatim}

 
 \subsubsection{SMOPS soil moisture assimilation}
 \label{sssec:smopssmda}
 

 
 \var{SMOPS soil moisture data directory:} specifies the directory
 for the SMOPS soil moisture data.

 \var{SMOPS soil moisture use ASCAT data:} specifies if the 
 ASCAT data layer is to be used.

 \var{SMOPS use scaled standard deviation model:} specifies if
 the observation error standard deviation is to be scaled using
 model and observation standard deviation.

 \var{SMOPS model CDF file:} specifies the
 name of the model CDF file (observations will be scaled into this
 climatology).

 \var{SMOPS observation CDF file:} specifies the
 name of the observation CDF file.

 \var{SMOPS soil moisture number of bins in the CDF:} specifies the
 number of bins in the CDF.

 \var{SMOPS use realtime data:} specifies whether to use
 the 6 hour data feed instead of the daily data feed.
 Acceptable values are: 

 \begin{tabular}{ll}
 Value & Description          \\
 0     & Use daily data feed  \\
 1     & Use 6 hour data feed \\
 \end{tabular}

 \var{SMOPS soil moisture use scaled standard deviation
 model:} specifies whether to use scaled standard deviation.
 This generates and uses spatially distributed observation
 errors by scaling the specified observation error standard
 deviation by a factor of the model standard deviation to the
 observation standard deviation.
 $ ( e \mapsto e \times m_\sigma / o_\sigma ) $

 \var{SMOPS naming convention:} specifies the naming convention of the 
 SMOPS soil moisture data.  Used when reading the 6-hour data feed.
 Acceptable values are: 

 \begin{tabular}{ll}
 Value     & Description                            \\
 ``LIS''   & YYYY/NPR\_SMOPS\_CMAP\_DYYYYMMDDHH.gr2 \\
 ``other'' & smops\_dYYYYMMDD\_sHH0000\_cness.gr2   \\
 \end{tabular}
 

 \begin{Verbatim}[frame=single]
SMOPS soil moisture data directory:       'input'
SMOPS soil moisture use ASCAT data:        1
SMOPS use scaled standard deviation model: 1
SMOPS model CDF file:                      lsm_cdf.nc
SMOPS observation CDF file:                obs_cdf.nc
SMOPS soil moisture number of bins in the CDF: 100 
SMOPS use realtime data:
SMOPS soil moisture use scaled standard deviation model:
SMOPS naming convention: LIS
 \end{Verbatim}





 
 \subsection{Radiative Transfer/Forward Models} \label{ssec:rtms}
 This section specifies the choice of radiative transfer or forward
 modeling tools.

 \var{Radiative transfer model:} specifies which RTM is to be used.
 Acceptable values are:

 \begin{tabular}{ll}
 Value         & Description   \\
 CRTM2EM       & CRTM2EM       \\
 CMEM          & CMEM          \\
 \end{tabular}

 \var{RTM invocation frequency:} specifies the invocation frequency
 of the chosen RTM.

 See Section \ref{ssec:timeinterval} for a description
 of how to specify a time interval.

 \var{RTM history output frequency:} specifies the history
 output frequency of the RTM.

 See Section \ref{ssec:timeinterval} for a description
 of how to specify a time interval.
 

 \begin{Verbatim}[frame=single]
Radiative transfer model:     CRTM2EM
RTM invocation frequency:     1hr  
RTM history output frequency: 3hr
 \end{Verbatim}



 \subsubsection{CRTM2EM} \label{ssec:crtm2em}
 This section specifies the specifications to enable a CRTM2EM instance.

 \var{CRTM number of sensors:} specifies the number of sensors
 to be used.

 \var{CRTM number of layers:} specifies the number of atmospheric
 layers.

 \var{CRTM number of absorbers:} specifies the number of absorbers.

 \var{CRTM number of clouds:} specifies the number of cloud types.

 \var{CRTM number of aerosols:} specifies the number of aerosol types.

 \var{CRTM sensor id:} specifies the name of sensors to be simulated.

 \var{CRTM coefficient data path:} specifies the location of the files
 containing the CRTM coefficient data.  These data are part of the
 Subversion checkout that was performed to obtain the CRTM library
 from JCSDA.  The \var{CRTM coefficient data path:} variable should
 either explicitly specify the whole path to or symbolicly link to 
 \file{trunk/fix/TauCoeff/ODPS/Big\_Endian/} found within the
 aforementioned checkout. 

 \var{RTM input soil moisture correction:} specifies whether to
 enable input soil moisture correction.
 Acceptable values are:

 \begin{tabular}{ll}
 Value & Description              \\
 0     & Do not enable correction \\
 1     & Enable correction        \\
 \end{tabular}

 \var{RTM input soil moisture correction src mean file:} specifies
 the RTM input soil moisture correction src mean file.

 \var{RTM input soil moisture correction src sigma file:} specifies
 the RTM input soil moisture correction src sigma file.

 \var{RTM input soil moisture correction dst mean file:} specifies
 the RTM input soil moisture correction dst mean file.

 \var{RTM input soil moisture correction dst sigma file:} specifies
 the RTM input soil moisture correction dst sigma file.
 

 \begin{Verbatim}[frame=single]
CRTM number of sensors:        1
CRTM number of layers:         64
CRTM number of absorbers:      2
CRTM number of clouds:         0 
CRTM number of aerosols:       0 
CRTM sensor id:                amsua_n18  
CRTM coefficient data path:    ./Coefficient_Data/
RTM input soil moisture correction:
RTM input soil moisture correction src mean file:
RTM input soil moisture correction src sigma file:
RTM input soil moisture correction dst mean file:
RTM input soil moisture correction dst sigma file:
 \end{Verbatim}

 
 \subsubsection{CMEM3} \label{ssec:cmem3}
 This section specifies the specifications to enable a CMEM3 instance.
 For more information regarding CMEM3, please see \\
 \hyperref{http://www.ecmwf.int/research/data\_assimilation/land\_surface/cmem/cmem\_index.html}{}{}{http://www.ecmwf.int/research/data\_assimilation/land\_surface/cmem/cmem\_index.html}.

 \var{CMEM3 sensor id:} specifies the name of sensors to be simulated.

 \var{CMEM3 number of frequencies:} specifies the number of
 frequencies.

 \var{CMEM3 frequencies file:} specifies the file
 containing the CMEM3 frequencies data.
 This is an ASCII file containing two columns of data.
 The first column specifies frequency in GHz; the second column
 specifies the incidence angle.  A sample file for AMSR-E:

 \begin{tabular}{lr}
  6.925 & 55.0 \\
  10.65 & 55.0 \\
  18.7  & 55.0 \\
  23.8  & 55.0 \\
  36.5  & 55.0 \\
  89.0  & 55.0 \\
 \end{tabular}

 \var{RTM input soil moisture correction:} specifies whether to
 enable input soil moisture correction.
 Acceptable values are:

 \begin{tabular}{ll}
 Value & Description              \\
 0     & Do not enable correction \\
 1     & Enable correction        \\
 \end{tabular}

 \var{RTM input soil moisture correction src mean file:} specifies
 the RTM input soil moisture correction src mean file.

 \var{RTM input soil moisture correction src sigma file:} specifies
 the RTM input soil moisture correction src sigma file.

 \var{RTM input soil moisture correction dst mean file:} specifies
 the RTM input soil moisture correction dst mean file.

 \var{RTM input soil moisture correction dst sigma file:} specifies
 the RTM input soil moisture correction dst sigma file.
 

 \begin{Verbatim}[frame=single]
CMEM3 sensor id:                  amsre
CMEM3 number of frequencies:      ./amsre-freqs.tx
CMEM3 frequencies file:
RTM input soil moisture correction:
RTM input soil moisture correction src mean file:
RTM input soil moisture correction src sigma file:
RTM input soil moisture correction dst mean file:
RTM input soil moisture correction dst sigma file:
 \end{Verbatim}

 
 \subsection{Optimization and Uncertainty Estimation}
 \label{ssec:optimization}
 This section specifies options for carrying out parameter estimation
 and uncertainty estimation.

 \var{Optimization/Uncertainty Estimation Algorithm:} Specifies which 
 algorithm is to be used for optimization.
 Acceptable values are:

 \begin{tabular}{ll}
 Value                                     & Description         \\
 ``none''                                  & no optimization     \\
 ``Genetic algorithm''                     & Genetic Algorithm   \\
 ``Monte carlo sampling''                  & MCSIM Algorithm     \\
 ``Differential evolution markov chain z'' & DEMCz Algorithm     \\
 \end{tabular}

 \var{Optimization/Uncertainty Estimation Set:} specifies
 the calibration data set,
 which represents the observation source used in the particular
 parameter estimation instance.
 Acceptable values are:

 \begin{tabular}{ll}
 Value                    & Description                               \\
 ``No obs''               & no observations                           \\
 ``AMSRE SR''             & AMSR-E (Colorado State Univ.)             \\
 ``AMSR-E(LPRM) pe soil moisture'' & AMSR-E LPRM soil moisture        \\
 \end{tabular}

 \var{Objective Function Method:} specifies the objective function
 method.
 Acceptable values are:

 \begin{tabular}{ll}
 Value             & Description                         \\
 ``Least squares'' & Least squares                       \\
 ``Likelihood''    & Maximum likelihood                  \\
 ``Probability''   & Maximize probability                \\
 \end{tabular}

 \var{Write PE Observations:} specifies whether to output processed
 observations for parameter estimation
 Acceptable values are:

 \begin{tabular}{ll}
 Value & Description                  \\
 0     & Do not write pe observations \\
 1     & Write pe observations        \\
 \end{tabular}

 \var{Number of model types subject to parameter estimation:}
 specifies the number of model classes used in a parameter estimation
 instance.  E.g.: if LSM and RTM parameters are simultaneously being
 calibrated then this option will be 2.

 \var{Model types subject to parameter estimation:} specifies the
 names of the model types to be used in the parameter estimation
 instance.  E.g.: LSM RTM

 \var{Number of model types with observation predictors for parameter estimation:}
 specifies the number of model types (e.g., LSM, RTM) that will be
 generating predictions of observations for comparison against real
 observations when conducting parameter or uncertainty estimation.
 Acceptable values are either 1 or 2.

 \var{Model types with observation predictors for parameter estimation:}
 specifies the list of model types (e.g., LSM, RTM) that will be
 generating predictions of observations for comparison against real
 observations when conducting parameter or uncertainty estimation.
 Acceptable values are a combination of LSM and/or RTM.

 \var{Initialize decision space with default values:} specifies
 whether to use defaults instead of sampled values at the beginning
 of optimization.
 Acceptable values are:

 \begin{tabular}{ll}
 Value & Description        \\
 0     & Use defaults       \\
 1     & Use sampled values \\
 \end{tabular}
 \newline (Yes, this is backwards from what the label suggests.)

 \var{Calibration period start year:} specifies the starting year
 of the calibration period.

 \var{Calibration period start month:} specifies the starting
 month of the calibration period.

 \var{Calibration period start day:} specifies the starting day
 of the calibration period.

 \var{Calibration period start hour:} specifies the starting hour
 of the calibration period.

 \var{Calibration period start minutes:} specifies the starting
 minutes of the calibration period.

 \var{Calibration period start seconds:} specifies the starting
 seconds of the calibration period.
 

 
 \subsubsection{Least squares} \label{ssec:ls}
 This section provides specifications of the LS objective function instance
 

 
 \var{Least Squares objective function weights file:} specifies the file containing
 the weights to be applied to each objective function

 \var{Least Squares objective function mode:} specifies which 
 least squares aggregation to use.
 Acceptable values are:

 \begin{tabular}{ll}
 Value & Description    \\
 1     & distributed (ie, optimized for each cell independently)  \\
 \end{tabular}

 \var{Least Squares objective function minimum number of obs:} for
 grid cells with fewer obs than specified, least squares parameter
 estimation will not be conducted so as to avoid 'overfitting' model to the data.
 

 
 \subsubsection{Probability} \label{sssec:probability}
 This section provides specifications of the Probability objective function
 instance.
 

 
 \var{Prior distribution attributes file:} specifies the file containing the 
  prior probability distribution over the parameters
 

 
 \subsubsection{Likelihood} \label{sssec:likelihood}
 This section provides specifications of the Likelihood objective function
 instance.  There are no additional specifications needed.  Unlike the Probability
 objective function, Likelihood does not factor in prior probability.
 

 
 \subsubsection{Genetic Algorithm} \label{ssec:ga}
 This section provides specifications of the genetic algorithm instance
 

 
 \var{GA restart file:} specifies the name of the 
 GA restart file.

 \var{GA number of generations:} specifies the 
 number of generations of GA.

 \var{GA number of children per parent:} specifies how many 
 offsprings are produced by two parent solutions (1 or 2).

 \var{GA crossover scheme:} specifies the type of crossover
 scheme.
 Acceptable values are:

 \begin{tabular}{ll}
 Value & Description            \\
 1     & single point crossover \\
 2     & uniform crossover      \\
 \end{tabular}

 \var{GA crossover probability:} threshold to be used for 
 conducting a crossover operation.
 \var{GA mutation scheme:} specifies the type of mutation
 scheme.
 Acceptable values are:

 \begin{tabular}{ll}
 Value & Description    \\
 0     & jump mutation  \\
 1     & creep mutation \\
 \end{tabular}

 \var{GA creep mutation probability:} specifies the 
 creep mutation max threshold.

 \var{GA jump mutation probability:} specifies the 
 jump mutation max threshold.

 \var{GA use elitism:} specifies whether to enable
 elitism in the selection of new solutions.
 Acceptable values are:

 \begin{tabular}{ll}
 Value & Description \\
 0     & do not use  \\
 1     & use         \\
 \end{tabular}

 \var{GA start mode:} specifies the start mode.
 Acceptable values are:

 \begin{tabular}{ll}
 Value     & Description \\
 restart   & restart     \\
 coldstart & cold start  \\
 \end{tabular}
 

 \begin{Verbatim}[frame=single]
GA restart file:                 ./OUTPUT/EXP999/GA/GA.188.GArst
GA number of generations:                      100
GA number of children per parent:              1
GA crossover scheme:                           2
GA crossover probability:                      0.5
GA use creep mutations:                        0 
GA creep mutation probability:                 0.04
GA jump mutation probability:                  0.02
GA use elitism:                                1
GA start mode:                                 coldstart 
 \end{Verbatim}

 
 \subsubsection{Differential Evolution Markov Chain (DEMCz) algorithm} \label{ssec:demcz}
 This section provides specifications of the DEMCz algorithm instance.
 DEMCz is an instance of Bayesian analysis (Reference: Gelman et al. (1995)) conducted via 
 Markov chain Monte Carlo (MCMC) (Reference: Brooks et al. (2011)).
 MCMC enables generation of parameter ensembles for subsequent LIS ensemble runs,
 where the ensembles reflect user-specified probability distributions as updated with
 observational datasets.
 Reference for DEMCz: ter Braak (2006), and ter Braak and Vrugt (2008).
 DEMCz implements DEMC with the 'sampling from the past' of ter Braak and Vrugt (2008)
 

 
 \var{DEMCz restart file:} specifies the name of the 
 DEMCz restart file.

 \var{DEMCz number of iterations:} specifies the 
 number of iterations of DEMCz.

 \var{DEMCz GA restart file:} specifies the GA solution
  that serves as the DEMCz algorithm starting point

 \var{DEMCz perturbation factor:} Applied uniformly to 
  all parameters.  The product of this term and the 
 width of the parameter range (ie, max-min) determines the random-walk-like term ('b') in the DEMCz algorithm 

 \var{DEMCz mode hopping frequency:} At this frequency (f), full jumps between separated regions of high probability 
 may occur (so as to better balance exploration of each region) through the setting of a DEMCz
 control parameter (gamma=1); at frequency 1-f, the settings are optimized for exploration of the local region of high probability (gamma=2.38) 

 \var{DEMCz start mode:} specifies the start mode.
 Acceptable values are:

 \begin{tabular}{ll}
 Value     & Description \\
 restart   & restart     \\
 coldstart & cold start  \\
 \end{tabular}
 

 \begin{Verbatim}[frame=single]
DEMCz restart file:                 ./OUTPUT/DEMCz/DEMCz.188.DEMCzrst
DEMCz number of iterations:         100
DEMCz start mode:                   coldstart 
DEMCz GA restart file:              ./OUTPUT/GA/GA.188.GArst
DEMCz perturbation factor:         0.001
DEMCz mode hopping frequency:      0.10
 \end{Verbatim}

 
 \subsubsection{Monte Carlo simulation} \label{ssec:mcsim}
 This section provides specifications of the MCSIM algorithm instance.
 MCSIM randomly samples from user-specified probability distributions
 to generate parmeter ensembles for subsequent use in  LIS ensemble runs.
 Unlike MCMC algorithms (e.g., DEMCz), the probability distributions being sampled
 are those given by the user, and not as updated with observational datasets.
 Algorithm reference: Morgan and Henrion (1990).
 

 
 
 \var{MCSIM number of iterations:} specifies the 
 number of iterations of MCSIM.  This typically will be
 set to 1.  Only set to values higher than 1 to accumulate more samples
 than can be achieved in a single LIS ensemble run.

 \var{MCSIM start mode:} specifies the start mode.  The restart option, as just noted,
 would only be needed if the number of samples that can be achieved in a single LIS ensemble run is limiting.
 
 Acceptable values are:
 \begin{tabular}{ll}
 Value     & Description \\
 restart   & restart     \\
 coldstart & cold start  \\
 \end{tabular}

 \var{MCSIM restart file:} specifies the name of the 
 MCSIM restart file.
 

 \begin{Verbatim}[frame=single]
MCSIM number of iterations:         1
MCSIM start mode:                   coldstart 
MCSIM restart file:                 none
 \end{Verbatim}

 
 \subsubsection{Observations for Parameter Estimation} \label{ssec:peobs}
 This section of the config file includes the observation specifications
 for paramter estimation
 



 
 \subsubsection{AMSRE\_SR Emissivity}
 \label{sssec:amsresremobs}
 \var{AMSRE SR data directory:} specifies the location
 of the AMSR-E emissivity retrievals data.

 \var{AMSRE\_SR Emissivity observations attributes file:} specifies the location
 of the observation attributes file.

 \var{AMSRE\_SR number of observations threshold:} specifies how many observations
 must be behind emissivity average for cell
 

 \begin{Verbatim}[frame=single]
AMSRE_SR Emissivity Obs data directory: './obs/'
AMSRE_SR Emissivity observations attributes file: './AMSRE_SR_attribs.txt'
AMSRE_SR number of observations threshold: 5
 \end{Verbatim}

 
 \subsubsection{AMSR-E (LPRM) pe soil moisture}
 \label{sssec:amsrelprmsm}
 \var{LPRM AMSRE soil moisture data directory:} specifies the location
 of the AMSR-E LPRM soil moisture data.

 \var{LPRM AMSRE soil moisture observations attributes file:} specifies the location
 of the observation attributes file.
 

 \begin{Verbatim}[frame=single]
LPRM AMSRE soil moisture data directory: './LPRM.v6'
LPRM AMSRE soil moisture observations attributes file: './LPRM_attribs.txt'
 \end{Verbatim}

 
 \subsubsection{No obs}
 \label{sssec:noobs}
 This PE observation option is used when conducting MCSIM as MCSIM
 does not factor in observational datasets
 in the sampling of parameter ensembles.  There are no configuration options.
 

 
 \subsection{Parameters} \label{ssec:parameters}
 

 
 \var{LIS domain and parameter data file:} specifies the
 primary input file that contains LIS parameter data.

 LIS 7 includes a pre-processing system called the
 Land Data Toolkit (LDT).  It reads the raw parameter
 data and processes them to the LIS running domain.
 The \var{LIS domain and parameter data file:} is the
 result of the LDT pre-processing.  Please read the
 ``Land Data Toolkit (LDT) User's Guide'' for more
 information.
 

 \begin{Verbatim}[frame=single]
LIS domain and parameter data file: ./lis_input.d01.nc
 \end{Verbatim}

 
 \subsubsection{Parameter options} \label{ssec:paramopts}
 

 
 \var{Landmask data source:} specifies the usage of landmask data
 in the run. 
 Acceptable values are:

 \begin{tabular}{ll}
 Value & Description                                       \\
 none  & Do not landmask                                   \\
 LDT   & Read landmask from the LDT-generated \var{LIS domain and parameter data file:} \\
 \end{tabular}

 \var{Landcover data source:} specifies the usage of landcover data
 in the run. 
 Acceptable values are:

 \begin{tabular}{ll}
 Value   & Description                       \\
 LDT     & Read landcover data from the LDT-generated \var{LIS domain and parameter data file:}   \\
 \end{tabular}
 

 \begin{Verbatim}[frame=single]
Landmask data source:  LDT
Landcover data source: LDT
 \end{Verbatim}

 
 \var{Soil texture data source:} specifies the usage of soil texture
 data in the run. 
 Acceptable values are:

 \begin{tabular}{ll}
 Value & Description                                           \\
 none  & Do not read soil texture                              \\
 LDT   & Read soil texture data from the LDT-generated \var{LIS domain and parameter data file:}   \\
 \end{tabular}
 

 \begin{Verbatim}[frame=single]
Soil texture data source: LDT
 \end{Verbatim}

 
 \var{Soil fraction data source:} specifies the usage of soil
 fraction parameters in the run. 
 Acceptable values are:

 \begin{tabular}{ll}
 Value & Description                                             \\
 none  & Do not read soil fractions                              \\
 LDT   & Read soil fractions data from the LDT-generated \var{LIS domain and parameter data file:}   \\
 \end{tabular}
 

 \begin{Verbatim}[frame=single]
Soil fraction data source: none
 \end{Verbatim}

 
 \var{Soil color data source:} specifies the usage of soil
 color data in the run. 
 Acceptable values are:

 \begin{tabular}{ll}
 Value & Description                                         \\
 none  & Do not read soil color                              \\
 LDT   & Read soil color data from the LDT-generated \var{LIS domain and parameter data file:}   \\
 \end{tabular}
 

 \begin{Verbatim}[frame=single]
Soil color data source: none
 \end{Verbatim}

 
 \var{Elevation data source:} specifies the usage of topography data
 in the run.
 Acceptable values are:

 \begin{tabular}{ll}
 Value & Description                                         \\
 none  & Do not read elevation                           \\
 LDT   & Read elevation data from the LDT-generated \var{LIS domain and parameter data file:}   \\
 \end{tabular}
 

 \begin{Verbatim}[frame=single]
Elevation data source: LDT
 \end{Verbatim}

 
 \var{Slope data source:} specifies the usage of slope data in the run.
 Acceptable values are:

 \begin{tabular}{ll}
 Value & Description                                    \\
 none  & Do not read slope                              \\
 LDT   & Read slope data from the LDT-generated \var{LIS domain and parameter data file:}   \\
 \end{tabular}
 

 \begin{Verbatim}[frame=single]
Slope data source: none
 \end{Verbatim}

 
 \var{Aspect data source:} specifies the usage of aspect data in the run.
 Acceptable values are:

 \begin{tabular}{ll}
 Value & Description                                     \\
 none  & Do not read aspect                              \\
 LDT   & Read aspect data from the LDT-generated \var{LIS domain and parameter data file:}   \\
 \end{tabular}
 

 \begin{Verbatim}[frame=single]
Aspect data source: none
 \end{Verbatim}

 
 \var{Curvature data source:} specifies the usage of curvature data
 in the run.
 Acceptable values are:

 \begin{tabular}{ll}
 Value & Description                                        \\
 none  & Do not read curvature                              \\
 LDT   & Read curvature data from the LDT-generated \var{LIS domain and parameter data file:}   \\
 \end{tabular}
 

 \begin{Verbatim}[frame=single]
Curvature data source: none
 \end{Verbatim}

 
 \var{LAI data source:} specifies the usage of LAI data in the run.  
 Acceptable values are:

 \begin{tabular}{ll}
 Value    & Description                                  \\
 none     & Do not read LAI                              \\
 LDT      & Read LAI data from the LDT-generated \var{LIS domain and parameter data file:}   \\
 ``MODIS real-time'' & Read MODIS real-time LAI          \\
 \end{tabular}
 

 \begin{Verbatim}[frame=single]
LAI data source: none
 \end{Verbatim}

 
 \var{SAI data source:} specifies the usage of LAI data in the run.  
 Acceptable values are:

 \begin{tabular}{ll}
 Value    & Description                                  \\
 none     & Do not read SAI                              \\
 LDT      & Read SAI data from the LDT-generated \var{LIS domain and parameter data file:}   \\
 ``MODIS real-time'' & Read MODIS real-time SAI          \\
 \end{tabular}
 

 \begin{Verbatim}[frame=single]
SAI data source: none
 \end{Verbatim}

 
 \var{Albedo data source:} specifies the usage of albedo data
 in the run.
 Acceptable values are:

 \begin{tabular}{ll}
 Value & Description                                     \\
 none  & Do not read albedo                              \\
 LDT   & Read albedo data from the LDT-generated \var{LIS domain and parameter data file:}   \\
 \end{tabular}
 

 \begin{Verbatim}[frame=single]
Albedo data source: LDT
 \end{Verbatim}

 
 \var{Max snow albedo data source:} specifies the usage of the
 maximum snow albedo in the run.
 Acceptable values are:

 \begin{tabular}{ll}
 Value & Description                                             \\
 none  & Do not read max snow albedo                             \\
 fixed & Use fixed max snow albedo from the ``lis.config'' file. \\
       &   This option is only available to the Noah-3.x LSMs.   \\
 LDT   & Read max snow albedo data from the LDT-generated \var{LIS domain and parameter data file:}   \\
 \end{tabular}
 

 \begin{Verbatim}[frame=single]
Max snow albedo data source: LDT
 \end{Verbatim}

 
 \var{Greenness data source:} specifies the usage of greenness
 fraction data in the run.
 Acceptable values are:

 \begin{tabular}{ll}
 Value             & Description                           \\
 none              & Do not read greenness fraction        \\
 LDT               & Read greenness data from the LDT-generated \var{LIS domain and parameter data file:}   \\
 ``NESDIS weekly'' & Read NESDIS weekly greenness fraction \\
 ``SPORT''         & Read SPORT greenness fraction         \\
 VIIRS             & Read VIIRS greenness fraction         \\
 \end{tabular}
 

 \begin{Verbatim}[frame=single]
Greenness data source: LDT
 \end{Verbatim}

 
 \var{Roughness data source:} specifies the usage of roughness data
 in the run.
 Acceptable values are:

 \begin{tabular}{ll}
 Value & Description                                        \\
 none  & Do not read roughness                              \\
 LDT   & Read roughness data from the LDT-generated \var{LIS domain and parameter data file:}   \\
 \end{tabular}
 

 \begin{Verbatim}[frame=single]
Roughness data source: none
 \end{Verbatim}

 
 \var{Porosity data source:} specifies the usage of soil porosity data
 in the run.
 Acceptable values are:

 \begin{tabular}{ll}
 Value & Description                                            \\
 none  & Do not read soil porosity                              \\
 LDT   & Read porosity data from the LDT-generated \var{LIS domain and parameter data file:}   \\
 \end{tabular}
 

 \begin{Verbatim}[frame=single]
Porosity data source: none
 \end{Verbatim}

 
 \var{Ksat data source:} specifies the usage of hydraulic conductivity
 data in the run.
 Acceptable values are:

 \begin{tabular}{ll}
 Value & Description                              \\
 none  & Do not read hydraulic conductivity       \\
 LDT   & Read hydraulic conductivity data from the LDT-generated \var{LIS domain and parameter data file:}   \\
 \end{tabular}
 

 \begin{Verbatim}[frame=single]
Ksat data source: none
 \end{Verbatim}

 
 \var{B parameter data source:} specifies the usage of b parameter data
 in the run.
 Acceptable values are:

 \begin{tabular}{ll}
 Value & Description                                          \\
 none  & Do not read b parameter                              \\
 LDT   & Read b parameter data from the LDT-generated \var{LIS domain and parameter data file:}   \\
 \end{tabular}
 

 \begin{Verbatim}[frame=single]
B parameter data source: none
 \end{Verbatim}

 
 \var{Quartz data source:} specifies the usage of quartz data
 in the run.
 Acceptable values are:

 \begin{tabular}{ll}
 Value & Description                                     \\
 none  & Do not read quartz                              \\
 LDT   & Read quartz data from the LDT-generated \var{LIS domain and parameter data file:}   \\
 \end{tabular}
 

 \begin{Verbatim}[frame=single]
Quartz data source: none
 \end{Verbatim}

 
 \var{Emissivity data source:} specifies the usage of emissivity data
 in the run.
 Acceptable values are:

 \begin{tabular}{ll}
 Value & Description                                         \\
 none  & Do not read emissivity                              \\
 LDT   & Read emissivity data from the LDT-generated \var{LIS domain and parameter data file:}   \\
 \end{tabular}
 

 \begin{Verbatim}[frame=single]
Emissivity data source: none
 \end{Verbatim}

 
 \subsubsection{TBOT lag} \label{sssec:tbotlag}
 

 
 \var{TBOT lag skin temperature update option:} specifies whether
 to adjust deep soil temperature as a weighted average of
 previous year's annual mean skin temperature and mean of time
 series of recent daily mean skin temperatures.
 Acceptable values are:

 \begin{tabular}{ll}
 Value & Description                         \\
 0     & Do not adjust deep soil temperature \\
 1     & Adjust deep soil temperature        \\
 \end{tabular}

 \var{TBOT skin temperature lag days:} specifies the number of
 lag days.
 

 \begin{Verbatim}[frame=single]
TBOT lag skin temperature update option: 0
TBOT skin temperature lag days:          0
 \end{Verbatim}

 
 \subsubsection{MODIS real-time LAI}
 \label{sssec:modisrtlai}
 

 
 \var{MODIS LAI data directory:} specifies the location of the
 MODIS real-time LAI files.
 

 \begin{Verbatim}[frame=single]
MODIS LAI data directory:
 \end{Verbatim}

 
 \subsubsection{NESDIS weekly greenness fraction}
 \label{sssec:nesdisgreenness}
 

 
 \var{NESDIS greenness data directory:} specifies the location of
 the NESDIS weekly greenness files.
 

 \begin{Verbatim}[frame=single]
NESDIS greenness data directory:
 \end{Verbatim}

 
 \subsubsection{SPORT greenness fraction}
 \label{sssec:sportgreenness}
 

 
 \var{SPORT greenness data directory:} specifies the location of
 the SPORT greenness files.

 \var{SPORT GVF use realtime mode:} specifies whether to use the
 realtime mode.  When not using realtime mode, LIS reads the
 previous and the next GVF bookends for temporal interpolation.
 When using realtime mode, LIS reads only the next GVF bookend
 for temporal interpolation.
 Acceptable values are:

 \begin{tabular}{ll}
 Value & Description              \\
 0     & Do not use realtime mode \\
 1     & Use realtime mode        \\
 \end{tabular}

 \var{SPORT GVF lower left lat:} specifies the lower left latitude
 of the SPORT GVF domain.
 (cylindrical latitude/longitude projection)

 \var{SPORT GVF lower left lon:} specifies the lower left
 longitude of the SPORT GVF domain.
 (cylindrical latitude/longitude projection)

 \var{SPORT GVF upper right lat:} specifies the upper right latitude
 of the SPORT GVF domain.
 (cylindrical latitude/longitude projection)

 \var{SPORT GVF upper right lon:} specifies the upper right
 longitude of the SPORT GVF domain.
 (cylindrical latitude/longitude projection)

 \var{SPORT GVF resolution (dx):} specifies the resolution of the 
 SPORT GVF domain along the east-west direction.

 \var{SPORT GVF resolution (dy):} specifies the resolution of the 
 SPORT GVF domain along the north-south direction.
 

 \begin{Verbatim}[frame=single]
SPORT greenness data directory:  ./LISDATA/MODISNDVI/GVF_COMBINED_GLOBAL/gvf_SPORT_3KM
SPORT GVF use realtime mode:      1
SPORT GVF lower left lat:        -59.985
SPORT GVF lower left lon:       -179.985
SPORT GVF upper right lat:        89.985
SPORT GVF upper right lon:       179.985
SPORT GVF resolution (dx):         0.03
SPORT GVF resolution (dy):         0.03
 \end{Verbatim}

 
 \subsubsection{VIIRS greenness fraction}
 \label{sssec:viirsgreenness}
 

 
 \var{VIIRS GVF use realtime mode:} specifies whether to use the
 realtime mode.  When not using realtime mode, LIS reads the
 previous and the next GVF bookends for temporal interpolation.
 When using realtime mode, LIS reads only the next GVF bookend
 for temporal interpolation.
 Acceptable values are:

 \begin{tabular}{ll}
 Value & Description              \\
 0     & Do not use realtime mode \\
 1     & Use realtime mode        \\
 \end{tabular}

 \var{VIIRS GVF lower left lat:} specifies the lower left latitude
 of the VIIRS GVF domain.
 (cylindrical latitude/longitude projection)

 \var{VIIRS GVF lower left lon:} specifies the lower left
 longitude of the VIIRS GVF domain.
 (cylindrical latitude/longitude projection)

 \var{VIIRS GVF upper right lat:} specifies the upper right latitude
 of the VIIRS GVF domain.
 (cylindrical latitude/longitude projection)

 \var{VIIRS GVF upper right lon:} specifies the upper right
 longitude of the VIIRS GVF domain.
 (cylindrical latitude/longitude projection)

 \var{VIIRS GVF resolution (dx):} specifies the resolution of the 
 VIIRS GVF domain along the east-west direction.

 \var{VIIRS GVF resolution (dy):} specifies the resolution of the 
 VIIRS GVF domain along the north-south direction.

 \var{VIIRS greenness data directory:} specifies the location of
 the VIIRS greenness files.
 

 \begin{Verbatim}[frame=single]
VIIRS greenness data directory:  ./LISDATA/VIIRSGVF/NESDIS_GVF_LISREAL/gvf_VIIRS_4KM
VIIRS GVF use realtime mode:      1
VIIRS GVF lower left lat:        -89.982
VIIRS GVF lower left lon:       -179.982
VIIRS GVF upper right lat:        89.982
VIIRS GVF upper right lon:       179.982
VIIRS GVF resolution (dx):         0.036
VIIRS GVF resolution (dy):         0.036
 \end{Verbatim}

 
 \subsection{Forcings} \label{ssec:forcings}
 

 
 \subsubsection{GDAS} \label{sssec:forcings_gdas}
 \var{GDAS forcing directory:} specifies the location of the GDAS
 forcing files.
 

 \begin{Verbatim}[frame=single]
GDAS forcing directory:            ./input/FORCING/GDAS/
 \end{Verbatim}

 
 \subsubsection{GEOS} \label{sssec:forcings_geos}
 

 
 \var{GEOS forcing directory:} specifies the location of the GEOS
 forcing files.
 

 \begin{Verbatim}[frame=single]
GEOS forcing directory:            ./input/FORCING/GEOS/BEST_LK/
 \end{Verbatim}

 
 \subsubsection{ECMWF} \label{sssec:forcings_ecmwf}
 

 
 \var{ECMWF forcing directory:} specifies the location of the ECMWF
 forcing files.
 

 \begin{Verbatim}[frame=single]
ECMWF forcing directory:       ./input/FORCING/ECMWF/
 \end{Verbatim}

 
 \subsubsection{ECMWF Reanalysis} \label{sssec:forcings_ecmwf_reanalysis}
 

 
 \var{ECMWF Reanalysis forcing directory:} specifies the location of
 the ECMWF Reanalysis forcing files.

 \var{ECMWF Reanalysis maskfile:} specifies the file containing
 the ECMWF Reanalysis land/sea mask.

 \var{ECMWF Reanalysis domain x-dimension size:} specifies the 
 number of columns of the ECMWF Reanalysis domain.

 \var{ECMWF Reanalysis domain y-dimension size:} specifies the 
 number of rows of the ECMWF Reanalysis domain.
 

 \begin{Verbatim}[frame=single]
ECMWF Reanalysis forcing directory:   ./input/FORCING/ECMWF-REANALYSIS/
ECMWF Reanalysis maskfile:            ./input/FORCING/ECMWF-REANALYSIS/ecmwf_land_sea.05
ECMWF Reanalysis domain x-dimension size:     720
ECMWF Reanalysis domain y-dimension size:     360
 \end{Verbatim}


 
 \subsubsection{PRINCETON} \label{sssec:forcings_princeton}
 

 
 \var{PRINCETON forcing directory:} specifies the location of the
 PRINCETON forcing files.
 

 \begin{Verbatim}[frame=single]
PRINCETON forcing directory:      ./input/FORCING/PRINCETON
 \end{Verbatim}

 
 \subsubsection{Rhone AGG} \label{sssec:forcings_rhone}
 

 
 \var{Rhone AGG forcing directory:} specifies the location of the
 Rhone AGG forcing files.

 \var{Rhone AGG domain x-dimension size:} specifies the number of
 columns of the native domain parameters of the Rhone AGG forcing data.
 The map projection is specified in the driver modules defined for
 the Rhone AGG routines.

 \var{Rhone AGG domain y-dimension size:} specifies the number of
 rows of the native domain parameters of the Rhone AGG forcing data.
 The map projection is specified in the driver modules defined for
 the Rhone AGG routines.
 

 \begin{Verbatim}[frame=single]
Rhone AGG forcing directory:       ./input/FORCING/RHONE
Rhone AGG domain x-dimension size: 5 
Rhone AGG domain y-dimension size: 6
 \end{Verbatim}

 
 \subsubsection{GSWP2} \label{sssec:forcings_gswp2}
 

 
 \var{GSWP2 landmask file:} specifies the GSWP2 landmask file.

 \var{GSWP2 domain x-dimension size:} specifies the number of columns
 of the GSWP2 domain.

 \var{GSWP2 domain y-dimension size:} specifies the number of rows
 of the GSWP2 domain.

 \var{GSWP2 number of forcing variables:} specifies the number of
 GSWP2 forcing variables.

 \var{GSWP2 2m air temperature map:} specifies the GSWP2 2 meter
 air temperature data.

 \var{GSWP2 2m specific humidity map:} specifies the GSWP2 2 meter
 specific humidity data.

 \var{GSWP2 wind map:} specifies the GSWP2 wind data.

 \var{GSWP2 surface pressure map:} specifies the GSWP2 surface
 pressure data.

 \var{GSWP2 convective rainfall rate map:} specifies the GSWP2
 convective rainfall rate data.

 \var{GSWP2 rainfall rate map:} specifies the GSWP2
 rainfall rate data.

 \var{GSWP2 snowfall rate map:} specifies the GSWP2
 snowfall rate data.

 \var{GSWP2 incident shortwave radiation map:} specifies the GSWP2
 incident shortwave radiation data.

 \var{GSWP2 incident longwave radiation map:} specifies the GSWP2
 incident longwave radiation data.
 

 \begin{Verbatim}[frame=single]
GSWP2 landmask file:                    ./input/gswp2data/Fixed/landmask_gswp.nc
GSWP2 domain x-dimension size:          360
GSWP2 domain y-dimension size:          150
GSWP2 number of forcing variables:      10
GSWP2 2m air temperature map:           ./input/gswp2data/Tair_cru/Tair_cru
GSWP2 2m specific humidity map:         ./input/gswp2data/Qair_cru/Qair_cru
GSWP2 wind map:                         ./input/gswp2data/Wind_ncep/Wind_ncep
GSWP2 surface pressure map:             ./input/gswp2data/PSurf_ecor/PSurf_ecor
GSWP2 convective rainfall rate map:     ./input/gswp2data/Rainf_C_gswp/Rainf_C_gswp
GSWP2 rainfall rate map:                ./input/gswp2data/Rainf_gswp/Rainf_gswp
GSWP2 snowfall rate map:                ./input/gswp2data/Snowf_gswp/Snowf_gswp
GSWP2 incident shortwave radiation map: ./input/gswp2data/SWdown_srb/SWdown_srb
GSWP2 incident longwave radiation map:  ./input/gswp2data/LWdown_srb/LWdown_srb
 \end{Verbatim}

 
 \subsubsection{GMAO GLDAS} \label{sssec:forcings_gmaogldas}
 

 
 \var{GLDAS forcing directory:} specifies the location of the
 GMAO GLDAS forcing files.
 

 \begin{Verbatim}[frame=single]
GLDAS forcing directory:         ../FORCING/GLDAS_GMAO/
 \end{Verbatim}

 
 \subsubsection{GFS} \label{sssec:forcings_gfs}
 \var{GFS forcing directory:} specifies the location of the GFS
 forcing files.

 \var{GFS domain x-dimension size:} specifies the number of
 columns of the native domain parameters of the GFS forcing data.
 The map projection is specified in the driver modules defined for
 the GFS routines.

 \var{GFS domain y-dimension size:} specifies the number of
 rows of the native domain parameters of the GFS forcing data.
 The map projection is specified in the driver modules defined for
 the GFS routines.

 \var{GFS number of forcing variables:} specifies the number of
 forcing variables provided by GFS at the model initialization step. 
 

 \begin{Verbatim}[frame=single]
GFS forcing directory:            ./input/FORCING/GFS/
GFS domain x-dimension size:      512
GFS domain y-dimension size:      256
GFS number of forcing variables:  10
 \end{Verbatim}

 
 \subsubsection{MERRA-Land} \label{sssec:forcings_MERRALand}
 \var{MERRA-Land forcing directory:} specifies the location of
 the MERRA-Land forcing files.

 \var{MERRA-Land use lowest model level forcing:} specifies whether
 to use the lowest model level forcing.
 Acceptable values are:

 \begin{tabular}{ll}
 Value & Description                                \\
 0     & Do not use the lowest model level forcing. \\
 1     & Use the lowest model level forcing.        \\
 \end{tabular}
 

 \begin{Verbatim}[frame=single]
MERRA-Land forcing directory:
MERRA-Land use lowest model level forcing:
 \end{Verbatim}

 
 \subsubsection{MERRA2} \label{sssec:forcings_MERRA2}
 \var{MERRA2 forcing directory:} specifies the location of
 the MERRA2 forcing files.

 Please note that MERRA2 forcing data are not currently available
 to external users.  They should become available in July 2015.

 \var{MERRA2 use lowest model level forcing:} specifies whether
 to use the lowest model level forcing.
 Acceptable values are:

 \begin{tabular}{ll}
 Value & Description                                \\
 0     & Do not use the lowest model level forcing. \\
 1     & Use the lowest model level forcing.        \\
 \end{tabular}

 \var{MERRA2 use corrected total precipitation:} specifies whether
 to use the bias corrected total precipitation.
 Acceptable values are:

 \begin{tabular}{ll}
 Value & Description                                        \\
 0     & Do not use the bias corrected total precipitation. \\
 1     & Use the bias corrected total precipitation.        \\
 \end{tabular}
 

 \begin{Verbatim}[frame=single]
MERRA2 forcing directory:
MERRA2 use lowest model level forcing:
MERRA2 use corrected total precipitation:
 \end{Verbatim}

 
 \subsubsection{GSWP1} \label{sssec:forcings_gswp1}
 

 
 \var{GSWP1 forcing directory:} specifies the location of the
 GSWP1 forcing files.

 \var{GSWP1 domain x-dimension size:} specifies the number of
 columns of the native domain parameters of the GSWP1 forcing data.
 The map projection is specified in the driver modules defined for
 the GSWP1 routines.

 \var{GSWP1 domain y-dimension size:} specifies the number of
 rows of the native domain parameters of the GSWP1 forcing data.
 The map projection is specified in the driver modules defined for
 the GSWP1 routines.

 \var{GSWP1 number of forcing variables:} specifies the number of
 forcing variables provided by GSWP1 at the model initialization
 step.
 

 \begin{Verbatim}[frame=single]
GSWP1 forcing directory:       ./input/FORCING/GSWP1
GSWP1 domain x-dimension size: 360
GSWP1 domain y-dimension size: 150
GSWP1 number of forcing variables: 9
 \end{Verbatim}

 
 \subsection{Supplemental forcings} \label{ssec:suppforcings}
 

 
 \subsubsection{AGRMET radiation (latlon)} \label{sssec:supp_agrrad}
 \var{AGRRAD forcing directory:} specifies the directory containing
 AGRMET radiation data.
 

 \begin{Verbatim}[frame=single]
AGRRAD forcing directory:             ./input/FORCING/AGRRAD
 \end{Verbatim}

 
 \subsubsection{AGRMET radiation (polar stereographic)}
 \label{sssec:supp_agrradps}
 

 
 \var{AGRRADPS forcing directory:} specifies the directory containing
 AGRMET polar stereographic radiation data.
 

 \begin{Verbatim}[frame=single]
AGRRADPS forcing directory:             ./input/FORCING/AGRRADPS
 \end{Verbatim}


 
 \subsubsection{CMAP precipitation} \label{sssec:supp_cmap}
 

 
 \var{CMAP forcing directory:} specifies the location of the
 CMAP forcing files.
 

 \begin{Verbatim}[frame=single]
CMAP forcing directory:             ./input/FORCING/CMAP
 \end{Verbatim}

 
 \subsubsection{CEOP station data} \label{sssec:supp_ceop}
 CEOP station forcing --- during EOP1
 

 
 \var{CEOP location index:} specifies the location of the
 CEOP station.

 \var{CEOP forcing directory:} specifies the location of the
 CEOP forcing files.

 \var{CEOP metadata file:} specifies the file containing
 CEOP metadata.
 

 \begin{Verbatim}[frame=single]
CEOP location index:           3 #SGP location
CEOP forcing directory:        ./input/FORCING/CEOP/sgp.cfr
CEOP metadata file:            ./input/FORCING/CEOP/sgp.mdata
 \end{Verbatim}

 
 \subsubsection{SCAN station data} \label{sssec:supp_scan}
 

 
 \var{SCAN forcing directory:} specifies the location of the
 SCAN forcing files.

 \var{SCAN metadata file:} specifies the file containing
 SCAN metadata.
 

 \begin{Verbatim}[frame=single]
SCAN forcing directory:   ./input/FORCING/SCAN
SCAN metadata file:       ./input/FORCING/SCAN/msu_scan.mdata
 \end{Verbatim}

 
 \subsubsection{NLDAS1} \label{sssec:forcings_nldas1}
 

 
 \var{NLDAS1 forcing directory:} specifies the location of the
 NLDAS-1 forcing files.

 \var{NLDAS1 data center source:} specifies the center that produced
 the NLDAS1 files. (This is specified to distinguish the filenames.) 
 Acceptable values are:

 \begin{tabular}{ll}
 Value & Description   \\
 ``GES-DISC'' & NASA GES-DISC \\
 ``NCEP''     & NCEP          \\
 \end{tabular}

 \var{NLDAS1 precipitation field:} specifies the field to be used
 for the precipitation.  ``NLDAS1'' will use the standard gauge-
 based precipitation data from NLDAS-1.  ``EDAS'' will use the
 EDAS model precipitation data instead.  ``STAGEII'' will use the
 unbias-corrected STAGE II radar estimated precipitation, and use
 the EDAS model precipitation for times/locations when the data
 from STAGE II is unavailable.

 \var{NLDAS1 shortwave radiation field:} specifies the field to be
 used for the downward shortwave radiation at the surface.  ``NLDAS1''
 will use GOES SW radiation when/where it is available, and use the
 EDAS SW radiation otherwise.  ``EDAS'' will simply use EDAS radiation
 at all times and locations.

 \var{NLDAS1 apply CONUS mask:} specifies whether to apply the
 CONUS mask.
 Acceptable values are:

 \begin{tabular}{ll}
 Value & Description             \\
 0     & Do not apply CONUS mask \\
 1     & Apply CONUS mask        \\
 \end{tabular}

 \var{NLDAS1 CONUS mask file:} specifies the NLDAS1 CONUS mask file.

 \var{NLDAS1 mask lower left lat:} specifies the lower left latitude
 of the NLDAS1 mask domain. (cylindrical latitude/longitude projection)

 \var{NLDAS1 mask lower left lon:} specifies the lower left longitude
 of the NLDAS1 mask domain. (cylindrical latitude/longitude projection)

 \var{NLDAS1 mask upper right lat:} specifies the upper right latitude
 of the NLDAS1 mask domain. (cylindrical latitude/longitude projection)

 \var{NLDAS1 mask upper right lon:} specifies the upper right longitude
 of the NLDAS1 mask domain. (cylindrical latitude/longitude projection)

 \var{NLDAS1 mask resolution (dx):} specifies the resolution of the 
 NLDAS1 mask domain along the east-west direction.

 \var{NLDAS1 mask resolution (dy):} specifies the resolution of the 
 NLDAS1 mask domain along the north-south direction.
 

 \begin{Verbatim}[frame=single]
NLDAS1 forcing directory:               ./input/FORCING/NLDAS1
NLDAS1 data center source:              "GES-DISC"
NLDAS1 precipitation field:             NLDAS1
NLDAS1 shortwave radiation field:       NLDAS1
NLDAS1 apply CONUS mask:                0
NLDAS1 CONUS mask file:
NLDAS1 mask lower left lat:            25.0625
NLDAS1 mask lower left lon:           -124.9375
NLDAS1 mask upper right lat:           52.9375
NLDAS1 mask upper right lon:          -67.0625
NLDAS1 mask resolution (dx):            0.125
NLDAS1 mask resolution (dy):            0.125
 \end{Verbatim}

 
 \subsubsection{NLDAS2} \label{sssec:forcings_nldas2}
 

 
 \var{NLDAS2 forcing directory:} specifies the location of the NLDAS2
 forcing files.

 \var{NLDAS2 data center source:} specifies the center that produced
 the NLDAS2 files. (This is specified to distinguish the filenames.) 
 Acceptable values are:

 \begin{tabular}{ll}
 Value        & Description   \\
 ``GES-DISC'' & NASA GES-DISC \\
 ``NCEP''     & NCEP          \\
 \end{tabular}

 \var{NLDAS2 use model level data:} specifies whether or not to
 read in the model level data (instead of 2/10m fields) from the
 NLDAS2 forcing dataset (will open up and read ``B'' files).
 This data is at the height of the NARR lowest model level.

 Note that this will read in ``Height of Atmospheric Forcing''
 and ``Surface Exchange Coefficient for Heat''.  You must make
 sure that they are included in your forcing variables list file.
 Acceptable values are:

 \begin{tabular}{ll}
 Value & Description \\
 0     & do not use  \\
 1     & use         \\
 \end{tabular}

 \var{NLDAS2 use model based swdown:} specifies whether or not to
 read in the un-bias corrected model downward shortwave radiation 
 data (in leiu of the bias corrected data) from the NLDAS2 forcing
 dataset (will open up and read ``B'' files).  The data source is
 the NARR shortwave.
 Acceptable values are:

 \begin{tabular}{ll}
 Value & Description \\
 0     & do not use  \\
 1     & use         \\
 \end{tabular}

 \var{NLDAS2 use model based precip:} specifies whether or not
 to read in the model based precipitation data (instead of the 
 observation based precipitation) from the NLDAS2 forcing  
 dataset (will open up and read ``B'' files).  The data source
 is the NARR precipitation.
 Acceptable values are:

 \begin{tabular}{ll}
 Value & Description \\
 0     & do not use  \\
 1     & use         \\
 \end{tabular}

 \var{NLDAS2 use model based pressure:} specifies whether or
 not to read in the model base pressure data (instead of the 
 observation based pressure) from the NLDAS2 forcing dataset
 (will open up and read ``B'' files).  The data source is
 the pressure at the NARR lowest model level.
 Acceptable values are:

 \begin{tabular}{ll}
 Value & Description \\
 0     & do not use  \\
 1     & use         \\
 \end{tabular}
 

 \begin{Verbatim}[frame=single]
NLDAS2 forcing directory:               ./input/FORCING/NLDAS2
NLDAS2 data center source:              "GES-DISC"
NLDAS2 use model level data:            0
NLDAS2 use model based swdown:          0
NLDAS2 use model based precip:          0
NLDAS2 use model based pressure:        0
 \end{Verbatim}

 
 \subsubsection{TRMM 3B42RT precipitation} \label{sssec:supp_3b42rt}
 

 
 \var{TRMM 3B42RT forcing directory:} specifies the location of the
 TRMM 3B42RT forcing files.
 

 \begin{Verbatim}[frame=single]
TRMM 3B42RT forcing directory:     ./input/FORCING/3B42RT/
 \end{Verbatim}

 
 \var{TRMM 3B42RTV7 forcing directory:} specifies the location of the
 TRMM 3B42RT Version 7 forcing files.
 

 \begin{Verbatim}[frame=single]
TRMM 3B42RTV7 forcing directory:  ../MET_FORCING/3B42RT-V7/
 \end{Verbatim}

 
 \subsubsection{TRMM 3B42V6 precipitation} \label{sssec:supp_3b42v6}
 

 
 \var{TRMM 3B42V6 timestep:} specifies the timestep for reading
 the TRMM 3B42V6 data.

 See Section \ref{ssec:timeinterval} for a description
 of how to specify a time interval.

 \var{TRMM 3B42V6 forcing directory:} specifies the location of the
 TRMM 3B42V6 forcing files.
 

 \begin{Verbatim}[frame=single]
TRMM 3B42V6 forcing directory:       ./input/FORCING/3B42V6/
TRMM 3B42V6 timestep:                1hr
 \end{Verbatim}

 
 \subsubsection{TRMM 3B42V7 precipitation} \label{sssec:supp_3b42v7}
 

 
 \var{TRMM 3B42V7 timestep:} specifies the timestep for reading
 the TRMM 3B42V7 data.

 See Section \ref{ssec:timeinterval} for a description
 of how to specify a time interval.

 \var{TRMM 3B42V7 forcing directory:} specifies the location of the
 TRMM 3B42V7 forcing files.
 

 \begin{Verbatim}[frame=single]
TRMM 3B42V7 forcing directory:       ./input/FORCING/3B42V7/
TRMM 3B42V7 timestep:                1hr
 \end{Verbatim}

 
 \subsubsection{CMORPH precipitation} \label{sssec:supp_cmorph}
 

 
 \var{CMORPH forcing directory:} specifies the location of the
 CMORPH precipitation forcing files.
 

 \begin{Verbatim}[frame=single]
CMORPH forcing directory:     ./input/FORCING/CMORPH/
 \end{Verbatim}


 
 \subsubsection{Stage II precipitation} \label{sssec:supp_stageii}
 

 
 \var{STAGE2 forcing directory:} specifies the location of the
 STAGE2 forcing files.
 

 \begin{Verbatim}[frame=single]
STAGE2 forcing directory:       ./input/FORCING/STII
 \end{Verbatim}

 
 \subsubsection{Stage IV precipitation} \label{sssec:supp_stageiv}
 

 
 \var{STAGE4 forcing directory:} specifies the location of the
 STAGE4 forcing files.
 

 \begin{Verbatim}[frame=single]
STAGE4 forcing directory:       ./input/FORCING/STIV
 \end{Verbatim}



 
 \subsubsection{NARR} \label{sssec:supp_narr}
 

 
 \var{NARR forcing directory:} specifies the location of the
 NARR forcing files.

 \var{NARR domain x-dimension size:} specifies the number of
 columns of the native domain parameters of the NARR forcing data.

 \var{NARR domain y-dimension size:} specifies the number of
 rows of the native domain parameters of the NARR forcing data.

 \var{NARR domain y-dimension size:} specifies the number of
 rows of the native domain parameters of the NARR forcing data.

 \var{NARR domain z-dimension size:} specifies the number of
 atmospheric profiles in the NARR forcing data.
 

 \begin{Verbatim}[frame=single]
NARR forcing directory:            ./input/Code/NARR/
NARR domain x-dimension size:      768
NARR domain y-dimension size:      386
NARR domain z-dimension size:      30
 \end{Verbatim}



 
 \subsubsection{RFE2Daily} \label{sssec:supp_rfe2daily}
 

 
 \var{RFE2Daily forcing directory:} specifies the location of the
 RFE2Daily forcing files.

 \var{RFE2Daily time offset:} specifiles the time offset for the
 RFE2Daily forcing data, in hours.  This adjusts when LIS will
 read the RFE2Daily precipitation data.  For general use, the data
 should be read at hour 6z, but for use by GeoWRSI, the data should be
 read at hour 0z.
 

 \begin{Verbatim}[frame=single]
RFE2Daily forcing directory:    ./input/MET_FORCING/RFE2.0_CPC/Africa/
RFE2Daily time offset:          0 # for use by GeoWRSI
 \end{Verbatim}



 
 \subsubsection{PET\_USGS} \label{sssec:supp_petusgs}
 

 
 \var{USGS PET forcing directory:} specifies the location of the
 PET USGS forcing files.

 \var{USGS PET forcing type:} specifies the choice for PET forcing
  data type.
 Acceptable values are:

 \begin{tabular}{ll}
 Value & Description    \\
  current       & Retrospective or current time-based PET files  \\
  climatology   & Climatology-based PET files  \\
 \end{tabular}
 

 \begin{Verbatim}[frame=single]
USGS PET forcing directory:       ./PET_USGS
 \end{Verbatim}

 
 \subsubsection{RFE2 data bias corrected to GDAS} \label{sssec:rfe2gdas}
 

 
 \var{RFE2gdas forcing directory:} specifies the location of the
 RFE2gdas forcing files.
 

 \begin{Verbatim}[frame=single]
RFE2gdas forcing directory:
 \end{Verbatim}

 
 \subsubsection{NAM242} \label{sssec:supp_nam242}
 

 
 \var{NAM242 forcing directory:} specifies the location of the
 ``NAM 242 AWIPS Grid -- Over Alaska'' forcing files
 

 \begin{Verbatim}[frame=single]
NAM242 forcing directory: ./input/MET_FORCING/NAM242
 \end{Verbatim}

 
 \subsubsection{WRFout} \label{sssec:supp_wrfout}
 

 
 \var{WRF output forcing directory:} specifies the location of the
 WRF output data files.

 \var{WRF output domain x-dimension size:} specifies the number of
 columns of the native domain parameters of the WRF output data.

 \var{WRF output domain y-dimension size:} specifies the number of
 rows of the native domain parameters of the WRF output data.

 \var{WRF nest id:} specifies the nest id of the WRF output data files. 
 

 \begin{Verbatim}[frame=single]
WRF output forcing directory: ./input/wrfout/
WRF output domain x-dimension size:   741
WRF output domain y-dimension size:   588
WRF nest id:                          1
 \end{Verbatim}

 
 \subsubsection{GEOS5 Forecast} \label{sssec:geos5forecast}
 

 
 \var{GEOS5 forecast forcing directory:} specifies the location
 of the GEOS5 forecast forcing files.
 

 \begin{Verbatim}[frame=single]
GEOS5 forecast forcing directory:
 \end{Verbatim}

 
 \subsubsection{GDAS for LSWG} \label{sssec:gdaslswg}
 

 
 \var{GDASLSWG forcing file:} specifies the location of the
 GDASLSWG forcing file.

 \var{GDASLSWG domain lower left lat:} specifies the lower left
 latitude of the GDASLSWG domain.
 (cylindrical latitude/longitude projection)

 \var{GDASLSWG domain lower left lon:} specifies the lower left
 longitude of the GDASLSWG domain.
 (cylindrical latitude/longitude projection)

 \var{GDASLSWG domain upper right lat:} specifies the upper right
 latitude of the GDASLSWG domain.
 (cylindrical latitude/longitude projection)

 \var{GDASLSWG domain upper right lon:} specifies the upper right
 longitude of the GDASLSWG domain.
 (cylindrical latitude/longitude projection)

 \var{GDASLSWG domain resolution (dx):} specifies the resolution of the 
 of the GDASLSWG domain along the east-west direction.

 \var{GDASLSWG domain resolution (dy):} specifies the resolution of the 
 of the GDASLSWG domain along the north-south direction.
 

 \begin{Verbatim}[frame=single]
GDASLSWG forcing file:
GDASLSWG domain lower left lat:
GDASLSWG domain lower left lon:
GDASLSWG domain upper right lat:
GDASLSWG domain upper right lon:
GDASLSWG domain resolution (dx):
GDASLSWG domain resolution (dy):
 \end{Verbatim}

 
 \subsubsection{Bondville} \label{sssec:bondville}
 

 
 \var{Bondville forcing file:} specifies the location of the
 Bondville forcing file.
 

 \begin{Verbatim}[frame=single]
Bondville forcing file:
 \end{Verbatim}



 
 \subsubsection{SNOTEL} \label{sssec:snotel}
 

 
 \var{SNOTEL forcing directory:} specifies the location of the
 SNOTEL forcing files.

 \var{SNOTEL metadata file:} specifies the location of the SNOTEL
 metadata file.

 \var{SNOTEL coord file:} specifies the location of the SNOTEL
 coordinates file.
 

 \begin{Verbatim}[frame=single]
SNOTEL forcing directory:
SNOTEL metadata file:
SNOTEL coord file:
 \end{Verbatim}

 
 \subsubsection{COOP} \label{sssec:coop}
 

 
 \var{COOP forcing directory:} specifies the location of the COOP
 forcing files.

 \var{COOP metadata file:} specifies the location of the COOP
 metadata file.

 \var{COOP coord file:} specifies the location of the COOP
 coordinate file.
 

 \begin{Verbatim}[frame=single]
COOP forcing directory:
COOP metadata file:
COOP coord file:
 \end{Verbatim}

 
 \subsubsection{VIC processed forcing} \label{sssec:vicforcing}

 This is used by the LIS development team to support debugging VIC
 within LIS.  One must first run stand-alone VIC, configured to
 output its forcing data.  Then one must grid the output forcing
 data into a format understood by LIS.
 

 
 \var{VIC forcing directory:} specifies the location of the VIC
 processed forcing files.

 \var{VIC forcing interval:} specifies the frequency of the VIC
 processed forcing data, in seconds.

 \var{VIC forcing domain lower left lat:} specifies the lower left
 latitude of the VIC processed forcing data.
 (cylindrical latitude/longitude projection)

 \var{VIC forcing domain lower left lon:} specifies the lower left
 longitude of the VIC processed forcing data.
 (cylindrical latitude/longitude projection)

 \var{VIC forcing domain upper right lat:} specifies the upper
 right latitude of the VIC processed forcing data.
 (cylindrical latitude/longitude projection)

 \var{VIC forcing domain upper right lon:} specifies the upper
 right longitude of the VIC processed forcing data.
 (cylindrical latitude/longitude projection)

 \var{VIC forcing domain resolution (dx):} specifies the resolution
 of the of the VIC processed forcing data along the east-west direction.

 \var{VIC forcing domain resolution (dy):} specifies the resolution
 of the of the VIC processed forcing data along the north-south
 direction.

 \var{VIC NC:} specifies the number of columns of the VIC
 processed forcing data.

 \var{VIC NR:} specifies the number of rows of the VIC processed
 forcing data.
 

 \begin{Verbatim}[frame=single]
VIC forcing directory:
VIC forcing interval:
VIC forcing domain lower left lat:
VIC forcing domain lower left lon:
VIC forcing domain upper right lat:
VIC forcing domain upper right lon:
VIC forcing domain resolution (dx):
VIC forcing domain resolution (dy):
VIC NC:
VIC NR:
 \end{Verbatim}

 
 \subsubsection{PALS station} \label{sssec:pals}
 

 
 \var{PALS met forcing directory:} specifies the location of the
 PALS station forcing files.

 \var{PALS met forcing station name:} specifies the name of the
 PALS station.

 \var{PALS met forcing data start year:} specifies the starting
 year of the PALS station data.

 \var{PALS met forcing data start month:} specifies the starting
 month of the PALS station data.

 \var{PALS met forcing data start day:} specifies the starting
 day of the PALS station data.

 \var{PALS met forcing data start hour:} specifies the starting
 hour of the PALS station data.

 \var{PALS met forcing data start minute:} specifies the starting
 minute of the PALS station data.

 \var{PALS met forcing data start second:} specifies the starting
 second of the PALS station data.
 

 \begin{Verbatim}[frame=single]
PALS met forcing directory:
PALS met forcing station name:
PALS met forcing data start year:
PALS met forcing data start month:
PALS met forcing data start day:
PALS met forcing data start hour:
PALS met forcing data start minute:
PALS met forcing data start second:
 \end{Verbatim}

 
 \subsubsection{PILDAS} \label{sssec:pildas}
 

 
 \var{PILDAS forcing directory:} specifies the location of the
 PILDAS forcing files.

 \var{PILDAS forcing version:} specifies the version of the
 PILDAS forcing data.

 \var{PILDAS forcing use lowest model level fields:} specifies
 whether to use the lowest model level fields.
 Acceptable values are:

 \begin{tabular}{ll}
 Value & Description             \\
  0    & Do not use lowest level \\
  1    & Use lowest level        \\
 \end{tabular}
 

 \begin{Verbatim}[frame=single]
PILDAS forcing directory:
PILDAS forcing version:
PILDAS forcing use lowest model level fields:
 \end{Verbatim}

 
 \subsubsection{RDHM356} \label{sssec:rdhm356forcing}
 

 
 \var{RDHM precipitation forcing directory:} specifies the location
 of the RDHM precipitation forcing files.

 \var{RDHM temperature forcing directory:} specifies the location
 of the RDHM temperature forcing files.

 \var{RDHM precipitation scale factor:} specifies the RDHM
 precipitation scale factor, which is used to scale the integer 
 type XMRG data into real number representing precipitation amount. 

 \var{RDHM precipitation interval:} specifies the frequency of
 the precipitation forcing data, in seconds.

 \var{RDHM temperature interval:} specifies the frequency of the
 temperature forcing data, in seconds.

 \var{RDHM run window lower left hrap y:} lower left HRAP Y coordinate of run domain

 \var{RDHM run window lower left hrap x:} lower left HRAP X coordinate of run domain

 \var{RDHM run window upper right hrap y:} upper right HRAP Y coordinate of run domain

 \var{RDHM run window upper right hrap x:} upper right HRAP X coordinate of run domain 

 \var{RDHM run window hrap resolution:} spatial resolution (in HRAP unit) of run domain 

 \var{RDHM temperature undefined value:} specifies the undefined
 value for the temperature forcing data.

 \var{RDHM precipitation undefined value:} specifies the
 undefined value for the precipitation forcing data.

 \var{RDHM constant wind speed:} Constant wind speed (m/s) for entire run domain  
 

 \begin{Verbatim}[frame=single]
RDHM precipitation forcing directory:  ../testcase/precip
RDHM temperature forcing directory:    ../testcase/tair
RDHM precipitation scale factor:       1.0
RDHM precipitation interval:           3600
RDHM temperature interval:             3600
RDHM run window lower left hrap y:    48
RDHM run window lower left hrap x:    17
RDHM run window upper right hrap y:   821
RDHM run window upper right hrap x:   1059
RDHM run window hrap resolution:      1.0
RDHM temperature undefined value:      -1.0
RDHM precipitation undefined value:    -1.0
RDHM constant wind speed:              4.0
 \end{Verbatim}

 
 \var{Generated metforcing directory:} specifies the location of the
 LDT generated meteorological forcing files.  Files generated in LDT
 are in netCDF format, and they are automatically loaded and handled 
 by the LIS-7 reader.
 
 \begin{Verbatim}[frame=single]
Generated metforcing directory:   ./LDT_OUTPUT/
 \end{Verbatim}

 
 \subsection{Land surface models} \label{ssec:lsm}
 

 
 \subsubsection{Forcing only -- Template} \label{sssec:lsm_template}
 

 
 \var{TEMPLATE model timestep:} specifies the timestep for the run.
 The template LSM is not a model;
 rather, it is a placeholder for a model.  It demonstrates the hooks
 that are needed to add a land surface model into LIS.  This ``LSM''
 is also used to run LIS with the purpose of only processing and writing
 forcing data.

 See Section \ref{ssec:timeinterval} for a description
 of how to specify a time interval.
 

 \begin{Verbatim}[frame=single]
TEMPLATE model timestep: 1hr
 \end{Verbatim}

 
 \subsubsection{NCEP's Noah-2.7.1} \label{sssec:lsm_noah271}
 

 
 \var{Noah.2.7.1 model timestep:} specifies the timestep for the run.

 See Section \ref{ssec:timeinterval} for a description
 of how to specify a time interval.

 For a nested domain, the timesteps for each nest should be specified
 with white spaces as the delimiter. If two domains (one subnest) are
 employed, the first one using 900 seconds and the second one using
 3600 seconds as the timestep, the model timesteps are specified as:

 E.g.: \quad \verb+Noah.2.7.1 model timestep:  15mn 60mn+

 \var{Noah.2.7.1 restart output interval:} defines the restart
 writing interval for Noah-2.7.1. The typical value used in the
 LIS runs is 24 hours (1da).

 See Section \ref{ssec:timeinterval} for a description
 of how to specify a time interval.

 \var{Noah.2.7.1 restart file:} specifies the Noah-2.7.1 active
 restart file.

 \var{Noah.2.7.1 vegetation parameter table:} specifies the
 Noah-2.7.1 static vegetation parameter table file.

 \var{Noah.2.7.1 soil parameter table:} specifies the
 Noah-2.7.1 soil parameter file.

 \var{Noah.2.7.1 use PTF for mapping soil properties:} specifies if
 pedotransfer functions are to be used for mapping soil properties
 (0-do not use, 1-use).

 \var{Noah.2.7.1 number of vegetation parameters:}
 specifies the number of static vegetation
 parameters specified for each veg type.

 \var{Noah.2.7.1 soils scheme:} specifies the soil mapping scheme used.
 Acceptable values are:

 \begin{tabular}{ll}
 Value & Description \\
 1     & Zobler      \\
 2     & STATSGO     \\
 \end{tabular}

 \var{Noah.2.7.1 number of soil classes:} specifies the number of
 soil classes in the above mapping scheme.
 Acceptable values are:

 \begin{tabular}{ll}
 Value & Description \\
  9    & Zobler      \\
 19    & STATSGO     \\
 \end{tabular}

 \var{Noah.2.7.1 number of soil layers:} specifies the number of
 soil layers. The typical value used in Noah-2.7.1 is 4.

 \var{Noah.2.7.1 layer thicknesses:} specifies the thickness (in meters)
 of each of the Noah-2.7.1 soil layers (top layer to bottom layer).

 \var{Noah.2.7.1 initial skin temperature:}
 specifies the initial skin temperature in Kelvin used in the
 cold start runs.

 \var{Noah.2.7.1 initial soil temperatures:}
 specifies the initial soil temperature (for all layers,
 top to bottom) in Kelvin used in the cold start runs.

 \var{Noah.2.7.1 initial total soil moistures:} specifies the
 initial total volumetric soil moistures (for all layers,
 top to bottom) used in the cold start runs.
 (units $\frac{m^3}{m^3}$)

 \var{Noah.2.7.1 initial liquid soil moistures:} specifies the
 initial liquid volumetric soil moistures (for all layers,
 top to bottom) used in the cold start runs.
 (units $\frac{m^3}{m^3}$)

 \var{Noah.2.7.1 initial canopy water:} specifies the initial
 canopy water (m).

 \var{Noah.2.7.1 initial snow depth:} specifies the initial
 snow depth (m).

 \var{Noah.2.7.1 initial snow equivalent:} specifies the initial
 snow water equivalent (m).

 \var{Noah.2.7.1 reference height for forcing T and q:} specifies the
 height in meters of air temperature and specific humidity forcings.

 \var{Noah.2.7.1 reference height for forcing u and v:} specifies the
 height in meters of u and v wind forcings.

 \var{Noah.2.7.1 reinitialize parameters from OPTUE output:} specifies
 whether to reinitialize parameters from OPTUE output.
 Defaults to 0.

 \var{Noah.2.7.1 parameter restart file (from OPTUE):} specifies the
 restart file to use to reinitialize parameters.
 Only used when
 \var{Noah.2.7.1 reinitialize parameters from OPTUE output:}
 is set to 1.
 

 \begin{Verbatim}[frame=single]
Noah.2.7.1 model timestep:                  15mn
Noah.2.7.1 restart output interval:         1mo
Noah.2.7.1 restart file:                    ./LIS.E111.200401210000.d01.Noah271rst
Noah.2.7.1 vegetation parameter table:      ../../noah271_parms/noah.vegparms_UMD.txt
Noah.2.7.1 soil parameter table:            ../../noah271_parms/noah.soilparms_STATSGO-FAO.txt
Noah.2.7.1 use PTF for mapping soil properties: 0
Noah.2.7.1 number of vegetation parameters: 7
Noah.2.7.1 soils scheme:                    2      # 1-Zobler; 2-STATSGO
Noah.2.7.1 number of soil classes:          16     # 9 for Zobler
Noah.2.7.1 number of soil layers:           4
Noah.2.7.1 layer thicknesses:               0.1  0.3  0.6  1.0
Noah.2.7.1 initial skin temperature:        290.0000                                 # Kelvin
Noah.2.7.1 initial soil temperatures:       290.0000  290.0000  290.0000  290.0000   # Kelvin
Noah.2.7.1 initial total soil moistures:    0.2000000 0.2000000 0.2000000 0.2000000  # volumetric (m3 m-3)
Noah.2.7.1 initial liquid soil moistures:   0.2000000 0.2000000 0.2000000 0.2000000  # volumetric (m3 m-3)
Noah.2.7.1 initial canopy water:            0.0                                      # depth (m)
Noah.2.7.1 initial snow depth:              0.0                                      # depth (m)
Noah.2.7.1 initial snow equivalent:         0.0                                      # SWE depth (m)
Noah.2.7.1 reference height for forcing T and q:  20.0
Noah.2.7.1 reference height for forcing u and v:  20.0
 \end{Verbatim}


 
 \subsubsection{NCAR's Noah-3.2} \label{sssec:lsm_noah32}
 

 
 \var{Noah.3.2 model timestep:} specifies the timestep for the run.

 See Section \ref{ssec:timeinterval} for a description
 of how to specify a time interval.

 For a nested domain, the timesteps for each nest should be specified
 with white spaces as the delimiter. If two domains (one subnest) are
 employed, the first one using 900 seconds and the second one using
 3600 seconds as the timestep, the model timesteps are specified as:

 E.g.: \quad \verb+Noah.3.2 model timestep:  15mn 60mn+

 \var{Noah.3.2 restart output interval:} defines the restart
 writing interval for Noah-3.2. The typical value used in the
 LIS runs is 24 hours (1da).

 See Section \ref{ssec:timeinterval} for a description
 of how to specify a time interval.

 \var{Noah.3.2 restart file:} specifies the Noah-3.2 active
 restart file.

 \var{Noah.3.2 vegetation parameter table:} specifies the
 Noah-3.2 static vegetation parameter table file.

 \var{Noah.3.2 soil parameter table:} specifies the
 Noah-3.2 soil parameter file.

 \var{Noah.3.2 general parameter table:} specifies the
 Noah-3.2 general parameter file.

 \var{Noah.3.2 use PTF for mapping soil properties:} specifies if
 pedotransfer functions are to be used for mapping soil properties
 (0-do not use, 1-use).

 \var{Noah.3.2 soils scheme:} specifies the soil mapping scheme used.
 Acceptable values are:

 \begin{tabular}{ll}
 Value & Description \\
 1     & Zobler      \\
 2     & STATSGO     \\
 \end{tabular}

 \var{Noah.3.2 number of soil layers:} specifies the number of
 soil layers. The typical value used in Noah is 4.

 \var{Noah.3.2 layer thicknesses:} specifies the thickness (in meters)
 of each of the Noah-3.2 soil layers (top layer to bottom layer).

 \var{Noah.3.2 use distributed soil depth map:} specifies whether
 to use a distributed soil depth map. Defaults to 0.

 \var{Noah.3.2 use distributed root depth map:} specifies whether
 to use a distributed root depth map. Defaults to 0.

 \var{Noah.3.2 initial skin temperature:}
 specifies the initial skin temperature in Kelvin used in the
 cold start runs.

 \var{Noah.3.2 initial soil temperatures:}
 specifies the initial soil temperature (for all layers,
 top to bottom) in Kelvin used in the cold start runs.

 \var{Noah.3.2 initial total soil moistures:} specifies the
 initial total volumetric soil moistures (for all layers,
 top to bottom) used in the cold start runs.
 (units $\frac{m^3}{m^3}$)

 \var{Noah.3.2 initial liquid soil moistures:} specifies the
 initial liquid volumetric soil moistures (for all layers,
 top to bottom) used in the cold start runs.
 (units $\frac{m^3}{m^3}$)

 \var{Noah.3.2 initial canopy water:} specifies the initial
 canopy water (m).

 \var{Noah.3.2 initial snow depth:} specifies the initial
 snow depth (m).

 \var{Noah.3.2 initial snow equivalent:} specifies the initial
 snow water equivalent (m).

 \var{Noah.3.2 fixed max snow albedo:} specifies a fixed maximum
 snow albedo (fraction, 0.0 to 1.0) for all grid points.  This
 value will only be used if ``fixed'' is chosen for
 \var{Max snow albedo data source}.

 \var{Noah.3.2 fixed deep soil temperature:} specifies a fixed
 deep soil temperature (Kelvin) for all grid points.  Entering
 a value of 0.0 will have the code use the deep soil temperature
 from the LDT-generated \var{LIS domain and parameter data file}.

 \var{Noah.3.2 fixed vegetation type:} specifies a fixed
 vegetation type index for all grid points.  Entering a value
 of 0 will not fix the vegetation types, and the code will use
 the \var{Landcover data source} information instead.

 \var{Noah.3.2 fixed soil type:} specifies a fixed soil
 type index for all grid points.  Entering a value of 0
 will not fix the soil types, and the code will use the
 \var{Soil texture data source} information instead.

 \var{Noah.3.2 fixed slope type:} specifies a fixed slope
 type index for all grid points.  Entering a value of 0 will
 not fix the slope index types, and the code will use the
 \var{Slope data source} information instead.

 \var{Noah.3.2 sfcdif option:} specifies whether to use the updated
 SFCDIF routine in Noah-3.2, or to use the previous SFCDIF routine.
 The typical option is to use the updated SFCDIF routine (option = 1).

 \var{Noah.3.2 z0 veg-type dependence option:} specifies whether
 to use the vegetation type dependent roughness height option
 on the CZIL parameter in the SFCDIF routine.  The typical option
 in Noah-3.2 is not use this dependence (option = 0).

 \var{Noah.3.2 greenness fraction:} specifies a monthly (January
 to December) greenness vegetation fraction for all grid points.
 These values are used only if the \var{Greenness data source}
 option is set to ``none''.

 \var{Noah.3.2 background albedo:} specifies a monthly background
 (snow-free) albedo for all grid points.  These values are only
 used for an initial condition calculation, and only if the
 \var{Albedo data source} option is set to ``none''.  After
 the first timestep, these values are not used.

 \var{Noah.3.2 background roughness length:} specifies a monthly
 background (snow-free) roughness length.  These values are used
 only for an initial condition calculation and are not used after
 the first timestep.

 \var{Noah.3.2 reference height for forcing T and q:} specifies the
 height in meters of air temperature and specific humidity forcings.

 \var{Noah.3.2 reference height for forcing u and v:} specifies the
 height in meters of u and v wind forcings.
 

 \begin{Verbatim}[frame=single]
Noah.3.2 model timestep:                  15mn
Noah.3.2 restart output interval:         1mo
Noah.3.2 restart file:                    LIS.E111.200805140000.d01.Noah32rst
Noah.3.2 vegetation parameter table:      ../../noah32_parms/VEGPARM.TBL
Noah.3.2 soil parameter table:            ../../noah32_parms/SOILPARM.TBL
Noah.3.2 general parameter table:         ../../noah32_parms/GENPARM.TBL
Noah.3.2 use PTF for mapping soil properties: 0
Noah.3.2 soils scheme:                    2      # 1-Zobler; 2-STATSGO
Noah.3.2 number of soil layers:           4
Noah.3.2 layer thicknesses:               0.1  0.3  0.6  1.0
Noah.3.2 use distributed soil depth map:  0      # 0 - do not use; 1 - use map
Noah.3.2 use distributed root depth map:  0      # 0 - do not use; 1 - use map
Noah.3.2 initial skin temperature:        290.0000                                 # Kelvin
Noah.3.2 initial soil temperatures:       290.0000  290.0000  290.0000  290.0000   # Kelvin
Noah.3.2 initial total soil moistures:    0.2000000 0.2000000 0.2000000 0.2000000  # volumetric (m3 m-3)
Noah.3.2 initial liquid soil moistures:   0.2000000 0.2000000 0.2000000 0.2000000  # volumetric (m3 m-3)
Noah.3.2 initial canopy water:            0.0                                      # depth (m)
Noah.3.2 initial snow depth:              0.0                                      # depth (m)
Noah.3.2 initial snow equivalent:         0.0                                      # SWE depth (m)
Noah.3.2 fixed max snow albedo:           0.0    # fraction; 0.0 - do not fix
Noah.3.2 fixed deep soil temperature:     0.0    # Kelvin; 0.0 - do not fix
Noah.3.2 fixed vegetation type:           0      # 0 - do not fix
Noah.3.2 fixed soil type:                 0      # 0 - do not fix
Noah.3.2 fixed slope type:                0      # 0 - do not fix
Noah.3.2 sfcdif option:                   1      # 0 - previous SFCDIF; 1 - updated SFCDIF
Noah.3.2 z0 veg-type dependence option:   0      # 0 - off; 1 - on; dependence of CZIL in SFCDIF
# Green vegetation fraction - by month
#  - used only if "Greenness data source" above is zero
Noah.3.2 greenness fraction:  0.01  0.02  0.07  0.17  0.27  0.58  0.93  0.96  0.65  0.24  0.11  0.02
# Background (i.e., snow-free) albedo - by month
#  - used only for first timestep; subsequent timesteps use
#    the values as computed in the previous call to "SFLX"
Noah.3.2 background albedo:   0.18  0.17  0.16  0.15  0.15  0.15  0.15  0.16  0.16  0.17  0.17  0.18
# Background (i.e., snow-free) roughness length (m) - by month
#  - used only for first timestep; subsequent timesteps use
#    the values as computed in the previous call to "SFLX"
Noah.3.2 background roughness length: 0.020 0.020 0.025 0.030 0.035 0.036 0.035 0.030 0.027 0.025 0.020 0.020
Noah.3.2 reference height for forcing T and q:  20.0      # (m) - negative=use height from forcing data
Noah.3.2 reference height for forcing u and v:  20.0      # (m) - negative=use height from forcing data
 \end{Verbatim}

 
 \subsubsection{NCAR's Noah-3.3} \label{sssec:lsm_noah33}
 

 
 \var{Noah.3.3 model timestep:} specifies the timestep for the run.

 See Section \ref{ssec:timeinterval} for a description
 of how to specify a time interval.

 For a nested domain, the timesteps for each nest should be specified
 with white spaces as the delimiter. If two domains (one subnest) are
 employed, the first one using 900 seconds and the second one using
 3600 seconds as the timestep, the model timesteps are specified as:

 E.g.: \quad \verb+Noah.3.3 model timestep:  15mn 60mn+

 \var{Noah.3.3 restart output interval:} defines the restart
 writing interval for Noah-3.3. The typical value used in the
 LIS runs is 24 hours (1da).

 See Section \ref{ssec:timeinterval} for a description
 of how to specify a time interval.

 \var{Noah.3.3 restart file:} specifies the Noah-3.3 active
 restart file.

 \var{Noah.3.3 vegetation parameter table:} specifies the
 Noah-3.3 static vegetation parameter table file.

 \var{Noah.3.3 soil parameter table:} specifies the
 Noah-3.3 soil parameter file.

 \var{Noah.3.3 general parameter table:} specifies the
 Noah-3.3 general parameter file.

 \var{Noah.3.3 use PTF for mapping soil properties:} specifies if
 pedotransfer functions are to be used for mapping soil properties
 (0-do not use, 1-use).

 \var{Noah.3.3 soils scheme:} specifies the soil mapping scheme used.
 Acceptable values are:

 \begin{tabular}{ll}
 Value & Description \\
 1     & Zobler      \\
 2     & STATSGO     \\
 \end{tabular}

 \var{Noah.3.3 number of soil layers:} specifies the number of
 soil layers. The typical value used in Noah is 4.

 \var{Noah.3.3 layer thicknesses:} specifies the thickness (in meters)
 of each of the Noah-3.3 soil layers (top layer to bottom layer).

 \var{Noah.3.3 use distributed soil depth map:} specifies whether
 to use a distributed soil depth map. Defaults to 0.

 \var{Noah.3.3 use distributed root depth map:} specifies whether
 to use a distributed root depth map. Defaults to 0.

 \var{Noah.3.3 initial skin temperature:}
 specifies the initial skin temperature in Kelvin used in the
 cold start runs.

 \var{Noah.3.3 initial soil temperatures:}
 specifies the initial soil temperature (for all layers,
 top to bottom) in Kelvin used in the cold start runs.

 \var{Noah.3.3 initial total soil moistures:} specifies the
 initial total volumetric soil moistures (for all layers,
 top to bottom) used in the cold start runs.
 (units $\frac{m^3}{m^3}$)

 \var{Noah.3.3 initial liquid soil moistures:} specifies the
 initial liquid volumetric soil moistures (for all layers,
 top to bottom) used in the cold start runs.
 (units $\frac{m^3}{m^3}$)

 \var{Noah.3.3 initial canopy water:} specifies the initial
 canopy water (m).

 \var{Noah.3.3 initial snow depth:} specifies the initial
 snow depth (m).

 \var{Noah.3.3 initial snow equivalent:} specifies the initial
 snow water equivalent (m).

 \var{Noah.3.3 fixed max snow albedo:} specifies a fixed maximum
 snow albedo (fraction, 0.0 to 1.0) for all grid points.  This
 value will only be used if ``fixed'' is chosen for
 \var{Max snow albedo data source}.

 \var{Noah.3.3 fixed deep soil temperature:} specifies a fixed
 deep soil temperature (Kelvin) for all grid points.  Entering
 a value of 0.0 will have the code use the deep soil temperature
 from the LDT-generated \var{LIS domain and parameter data file}.

 \var{Noah.3.3 fixed vegetation type:} specifies a fixed
 vegetation type index for all grid points.  Entering a value
 of 0 will not fix the vegetation types, and the code will use
 the \var{Landcover data source} information instead.

 \var{Noah.3.3 fixed soil type:} specifies a fixed soil
 type index for all grid points.  Entering a value of 0
 will not fix the soil types, and the code will use the
 \var{Soil texture data source} information instead.

 \var{Noah.3.3 fixed slope type:} specifies a fixed slope
 type index for all grid points.  Entering a value of 0 will
 not fix the slope index types, and the code will use the
 \var{Slope data source} information instead.

 \var{Noah.3.3 sfcdif option:} specifies whether to use the updated
 SFCDIF routine in Noah-3.3, or to use the previous SFCDIF routine.
 The typical option is to use the updated SFCDIF routine (option = 1).

 \var{Noah.3.3 z0 veg-type dependence option:} specifies whether
 to use the vegetation type dependent roughness height option
 on the CZIL parameter in the SFCDIF routine.  The typical option
 in Noah-3.3 is not use this dependence (option = 0).

 \var{Noah.3.3 greenness fraction:} specifies a monthly (January
 to December) greenness vegetation fraction for all grid points.
 These values are used only if the \var{Greenness data source}
 option is set to ``none''.

 \var{Noah.3.3 background albedo:} specifies a monthly background
 (snow-free) albedo for all grid points.  These values are only
 used for an initial condition calculation, and only if the
 \var{Albedo data source} option is set to ``none''.  After
 the first timestep, these values are not used.

 \var{Noah.3.3 background roughness length:} specifies a monthly
 background (snow-free) roughness length.  These values are used
 only for an initial condition calculation and are not used after
 the first timestep.

 \var{Noah.3.3 reference height for forcing T and q:} specifies the
 height in meters of air temperature and specific humidity forcings.

 \var{Noah.3.3 reference height for forcing u and v:} specifies the
 height in meters of u and v wind forcings.
 

 \begin{Verbatim}[frame=single]
Noah.3.3 model timestep:                  15mn
Noah.3.3 restart output interval:         1mo
Noah.3.3 restart file:                    LIS.E111.200805140000.d01.Noah33rst
Noah.3.3 vegetation parameter table:      ../../noah33_parms/VEGPARM.TBL
Noah.3.3 soil parameter table:            ../../noah33_parms/SOILPARM.TBL
Noah.3.3 general parameter table:         ../../noah33_parms/GENPARM.TBL
Noah.3.3 use PTF for mapping soil properties: 0
Noah.3.3 soils scheme:                    2      # 1-Zobler; 2-STATSGO
Noah.3.3 number of soil layers:           4
Noah.3.3 layer thicknesses:               0.1  0.3  0.6  1.0
Noah.3.3 use distributed soil depth map:  0      # 0 - do not use; 1 - use map
Noah.3.3 use distributed root depth map:  0      # 0 - do not use; 1 - use map
Noah.3.3 initial skin temperature:        290.0000                                 # Kelvin
Noah.3.3 initial soil temperatures:       290.0000  290.0000  290.0000  290.0000   # Kelvin
Noah.3.3 initial total soil moistures:    0.2000000 0.2000000 0.2000000 0.2000000  # volumetric (m3 m-3)
Noah.3.3 initial liquid soil moistures:   0.2000000 0.2000000 0.2000000 0.2000000  # volumetric (m3 m-3)
Noah.3.3 initial canopy water:            0.0                                      # depth (m)
Noah.3.3 initial snow depth:              0.0                                      # depth (m)
Noah.3.3 initial snow equivalent:         0.0                                      # SWE depth (m)
Noah.3.3 fixed max snow albedo:           0.0    # fraction; 0.0 - do not fix
Noah.3.3 fixed deep soil temperature:     0.0    # Kelvin; 0.0 - do not fix
Noah.3.3 fixed vegetation type:           0      # 0 - do not fix
Noah.3.3 fixed soil type:                 0      # 0 - do not fix
Noah.3.3 fixed slope type:                0      # 0 - do not fix
Noah.3.3 sfcdif option:                   1      # 0 - previous SFCDIF; 1 - updated SFCDIF
Noah.3.3 z0 veg-type dependence option:   0      # 0 - off; 1 - on; dependence of CZIL in SFCDIF
# Green vegetation fraction - by month
#  - used only if "Greenness data source" above is zero
Noah.3.3 greenness fraction:  0.01  0.02  0.07  0.17  0.27  0.58  0.93  0.96  0.65  0.24  0.11  0.02
# Background (i.e., snow-free) albedo - by month
#  - used only for first timestep; subsequent timesteps use
#    the values as computed in the previous call to "SFLX"
Noah.3.3 background albedo:   0.18  0.17  0.16  0.15  0.15  0.15  0.15  0.16  0.16  0.17  0.17  0.18
# Background (i.e., snow-free) roughness length (m) - by month
#  - used only for first timestep; subsequent timesteps use
#    the values as computed in the previous call to "SFLX"
Noah.3.3 background roughness length: 0.020 0.020 0.025 0.030 0.035 0.036 0.035 0.030 0.027 0.025 0.020 0.020
Noah.3.3 reference height for forcing T and q:   2.0      # (m) - negative=use height from forcing data
Noah.3.3 reference height for forcing u and v:  10.0      # (m) - negative=use height from forcing data
 \end{Verbatim}

 
 \subsubsection{NCAR's Noah-3.6} \label{sssec:lsm_noah36}
 

 
 \var{Noah.3.6 model timestep:} specifies the timestep for the run.

 See Section \ref{ssec:timeinterval} for a description
 of how to specify a time interval.

 For a nested domain, the timesteps for each nest should be specified
 with white spaces as the delimiter. If two domains (one subnest) are
 employed, the first one using 900 seconds and the second one using
 3600 seconds as the timestep, the model timesteps are specified as:

 E.g.: \quad \verb+Noah.3.6 model timestep:  15mn 60mn+

 \var{Noah.3.6 restart output interval:} defines the restart
 writing interval for Noah-3.6. The typical value used in the
 LIS runs is 24 hours (1da).

 See Section \ref{ssec:timeinterval} for a description
 of how to specify a time interval.

 \var{Noah.3.6 restart file:} specifies the Noah-3.6 active
 restart file.

 \var{Noah.3.6 vegetation parameter table:} specifies the
 Noah-3.6 static vegetation parameter table file.

 \var{Noah.3.6 soil parameter table:} specifies the
 Noah-3.6 soil parameter file.

 \var{Noah.3.6 general parameter table:} specifies the
 Noah-3.6 general parameter file.

 \var{Noah.3.6 use PTF for mapping soil properties:} specifies if
 pedotransfer functions are to be used for mapping soil properties
 (0-do not use, 1-use).

 \var{Noah.3.6 soils scheme:} specifies the soil mapping scheme used.
 Acceptable values are:

 \begin{tabular}{ll}
 Value & Description \\
 1     & Zobler      \\
 2     & STATSGO     \\
 \end{tabular}

 \var{Noah.3.6 number of soil layers:} specifies the number of
 soil layers. The typical value used in Noah is 4.

 \var{Noah.3.6 layer thicknesses:} specifies the thickness (in meters)
 of each of the Noah-3.6 soil layers (top layer to bottom layer).

 \var{Noah.3.6 use distributed soil depth map:} specifies whether
 to use a distributed soil depth map. Defaults to 0.

 \var{Noah.3.6 use distributed root depth map:} specifies whether
 to use a distributed root depth map. Defaults to 0.

 \var{Noah.3.6 initial skin temperature:}
 specifies the initial skin temperature in Kelvin used in the
 cold start runs.

 \var{Noah.3.6 initial soil temperatures:}
 specifies the initial soil temperature (for all layers,
 top to bottom) in Kelvin used in the cold start runs.

 \var{Noah.3.6 initial total soil moistures:} specifies the
 initial total volumetric soil moistures (for all layers,
 top to bottom) used in the cold start runs.
 (units $\frac{m^3}{m^3}$)

 \var{Noah.3.6 initial liquid soil moistures:} specifies the
 initial liquid volumetric soil moistures (for all layers,
 top to bottom) used in the cold start runs.
 (units $\frac{m^3}{m^3}$)

 \var{Noah.3.6 initial canopy water:} specifies the initial
 canopy water (m).

 \var{Noah.3.6 initial snow depth:} specifies the initial
 snow depth (m).

 \var{Noah.3.6 initial snow equivalent:} specifies the initial
 snow water equivalent (m).

 \var{Noah.3.6 fixed max snow albedo:} specifies a fixed maximum
 snow albedo (fraction, 0.0 to 1.0) for all grid points.  This
 value will only be used if ``fixed'' is chosen for
 \var{Max snow albedo data source}.

 \var{Noah.3.6 fixed deep soil temperature:} specifies a fixed
 deep soil temperature (Kelvin) for all grid points.  Entering
 a value of 0.0 will have the code use the deep soil temperature
 from the LDT-generated \var{LIS domain and parameter data file}.

 \var{Noah.3.6 fixed vegetation type:} specifies a fixed
 vegetation type index for all grid points.  Entering a value
 of 0 will not fix the vegetation types, and the code will use
 the \var{Landcover data source} information instead.

 \var{Noah.3.6 fixed soil type:} specifies a fixed soil
 type index for all grid points.  Entering a value of 0
 will not fix the soil types, and the code will use the
 \var{Soil texture data source} information instead.

 \var{Noah.3.6 fixed slope type:} specifies a fixed slope
 type index for all grid points.  Entering a value of 0 will
 not fix the slope index types, and the code will use the
 \var{Slope data source} information instead.

 \var{Noah.3.6 sfcdif option:} specifies whether to use the updated
 SFCDIF routine in Noah-3.6, or to use the previous SFCDIF routine.
 The typical option is to use the updated SFCDIF routine (option = 1).

 \var{Noah.3.6 z0 veg-type dependence option:} specifies whether
 to use the vegetation type dependent roughness height option
 on the CZIL parameter in the SFCDIF routine.  The typical option
 in Noah-3.6 is not use this dependence (option = 0).

 \var{Noah.3.6 Run UA snow-physics option:} specifies whether
 to run the University of Arizona (UA) snow-physics option.
 Either ``.true.'' or ``.false.'' should be selected.  If
 ``.true.'' is given, then the UA snow-physics will be run.
 If ``.false.'' is given, then the standard Noah snow-physics
 will be run instead.

 \var{Noah.3.6 greenness fraction:} specifies a monthly (January
 to December) greenness vegetation fraction for all grid points.
 These values are used only if the \var{Greenness data source}
 option is set to ``none''.

 \var{Noah.3.6 background albedo:} specifies a monthly background
 (snow-free) albedo for all grid points.  These values are only
 used for an initial condition calculation, and only if the
 \var{Albedo data source} option is set to ``none''.  After
 the first timestep, these values are not used.

 \var{Noah.3.6 background roughness length:} specifies a monthly
 background (snow-free) roughness length.  These values are used
 only for an initial condition calculation and are not used after
 the first timestep.

 \var{Noah.3.6 reference height for forcing T and q:} specifies the
 height in meters of air temperature and specific humidity observations.

 \var{Noah.3.6 reference height for forcing u and v:} specifies the
 height in meters of u and v wind forcings.
 

 \begin{Verbatim}[frame=single]
Noah.3.6 model timestep:                  15mn
Noah.3.6 restart output interval:         1mo
Noah.3.6 restart file:                    LIS.E111.200805140000.d01.Noah36rst
Noah.3.6 vegetation parameter table:      ../../noah36_parms/VEGPARM.TBL
Noah.3.6 soil parameter table:            ../../noah36_parms/SOILPARM.TBL
Noah.3.6 general parameter table:         ../../noah36_parms/GENPARM.TBL
Noah.3.6 use PTF for mapping soil properties: 0
Noah.3.6 soils scheme:                    2      # 1-Zobler; 2-STATSGO
Noah.3.6 number of soil layers:           4
Noah.3.6 layer thicknesses:               0.1  0.3  0.6  1.0
Noah.3.6 use distributed soil depth map:  0      # 0 - do not use; 1 - use map
Noah.3.6 use distributed root depth map:  0      # 0 - do not use; 1 - use map
Noah.3.6 initial skin temperature:        290.0000                                 # Kelvin
Noah.3.6 initial soil temperatures:       290.0000  290.0000  290.0000  290.0000   # Kelvin
Noah.3.6 initial total soil moistures:    0.2000000 0.2000000 0.2000000 0.2000000  # volumetric (m3 m-3)
Noah.3.6 initial liquid soil moistures:   0.2000000 0.2000000 0.2000000 0.2000000  # volumetric (m3 m-3)
Noah.3.6 initial canopy water:            0.0                                      # depth (m)
Noah.3.6 initial snow depth:              0.0                                      # depth (m)
Noah.3.6 initial snow equivalent:         0.0                                      # SWE depth (m)
Noah.3.6 fixed max snow albedo:           0.0    # fraction; 0.0 - do not fix
Noah.3.6 fixed deep soil temperature:     0.0    # Kelvin; 0.0 - do not fix
Noah.3.6 fixed vegetation type:           0      # 0 - do not fix
Noah.3.6 fixed soil type:                 0      # 0 - do not fix
Noah.3.6 fixed slope type:                0      # 0 - do not fix
Noah.3.6 sfcdif option:                   1      # 0 - previous SFCDIF; 1 - updated SFCDIF
Noah.3.6 z0 veg-type dependence option:   0      # 0 - off; 1 - on; dependence of CZIL in SFCDIF
Noah.3.6 Run UA snow-physics option:     .false. # ".true." or ".false"
# Green vegetation fraction - by month
#  - used only if "Greenness data source" above is zero
Noah.3.6 greenness fraction:  0.01  0.02  0.07  0.17  0.27  0.58  0.93  0.96  0.65  0.24  0.11  0.02
# Background (i.e., snow-free) albedo - by month
#  - used only for first timestep; subsequent timesteps use
#    the values as computed in the previous call to "SFLX"
Noah.3.6 background albedo:   0.18  0.17  0.16  0.15  0.15  0.15  0.15  0.16  0.16  0.17  0.17  0.18
# Background (i.e., snow-free) roughness length (m) - by month
#  - used only for first timestep; subsequent timesteps use
#    the values as computed in the previous call to "SFLX"
Noah.3.6 background roughness length: 0.020 0.020 0.025 0.030 0.035 0.036 0.035 0.030 0.027 0.025 0.020 0.020
Noah.3.6 reference height for forcing T and q:   2.0      # (m) - negative=use height from forcing data
Noah.3.6 reference height for forcing u and v:  10.0      # (m) - negative=use height from forcing data
 \end{Verbatim}

 
 \subsubsection{NoahMP 3.6} \label{sssec:lsm_noahmp36}
 

 
 \var{NOAHMP36 model timestep:} specifies the timestep for NoahMP.

 See Section \ref{ssec:timeinterval} for a description
 of how to specify a time interval.

 \var{NOAHMP36 restart output interval:} specifies the restart output
 interval for NoahMP.

 See Section \ref{ssec:timeinterval} for a description
 of how to specify a time interval.

 \var{NOAHMP36 number of soil layers:} specifiles the number of soil layers for NoahMP
 soil moisture/temperature.

 \var{NOAHMP36 number of snow layers:} specifies the number of snow layers for NoahMP
 snow model.

 \var{NOAHMP36 landuse\_tbl\_name:} specifies the file name of the
 NoahMP vegetation parameter table.

 \var{NOAHMP36 soil\_tbl\_name:} specifies the file name of the NoahMP
 soil parameter table.

 \var{NOAHMP36 gen\_tbl\_name:} specifies the file name of the NoahMP
 general parameter table.

 \var{NOAHMP36 noahmp\_tbl\_name:} specifies the file name of the NoahMP
 multi-physics parameter table.

 \var{NOAHMP36 landuse\_scheme\_name:} specifies the NoahMP landuse
 scheme name (e.g. USGS, check the NoahMP land use parameter table).

 \var{NOAHMP36 soil\_scheme\_name:} specifies the NoahMP soil scheme
 name (e.g. STAS, check NoahMP soil parameter table).

 \var{NOAHMP36 dveg\_opt:} specifies the vegetation model.
 Acceptable values are:

 \begin{tabular}{ll}
 Value & Description                         \\
 1     & prescribed [table LAI, shdfac=FVEG] \\
 2     & dynamic                             \\
 3     & table LAI, calculate FVEG           \\
 4     & table LAI, shdfac=maximum           \\
 \end{tabular}

 \var{NOAHMP36 crs\_opt:} specifies the canopy stomatal resistance
 scheme.  Acceptable values are:

 \begin{tabular}{ll}
 Value & Description \\
  1    & Ball-Berry  \\
  2    & Jarvis      \\
 \end{tabular}
 
 \var{NOAHMP36 btr\_opt:} specifies the soil moisture factor for
 stomatal resitance.  Acceptable values are:

 \begin{tabular}{ll}
 Value & Description \\
  1    & Noah        \\
  2    & CLM         \\
  3    & SSiB        \\
 \end{tabular}

 \var{NOAHMP36 run\_opt:} specifies the runoff and groundwater.
 Acceptable values are:

 \begin{tabular}{ll}
 Value & Description \\
  1    & SIMGM       \\
  2    & SIMTOP      \\
  3    & Schaake96   \\
  4    & BATS        \\
 \end{tabular}

 \var{NOAHMP36 sfc\_opt:} specifies surface layer drag coefficient.
 Acceptable values are:

 \begin{tabular}{ll}
 Value & Description \\
  1    & M-O         \\
  2    & Chen97      \\
 \end{tabular}

 \var{NOAHMP36 frz\_opt:} specifies the supercooled liquid water scheme.
 Acceptable values are:

 \begin{tabular}{ll}
 Value & Description \\
  1    & NY06        \\
  2    & Koren99     \\
 \end{tabular}

 \var{NOAHMP36 inf\_opt:} specifies the frozen soil permeability scheme.
 Acceptable values are:

 \begin{tabular}{ll}
 Value & Description \\
  1    & NY06        \\
  2    & Koren99     \\
 \end{tabular}

 \var{NOAHMP36 rad\_opt:} specifies the radiation transfer scheme.
 Acceptable values are:

 \begin{tabular} {ll}
 Value & Description    \\
  1    & gap=F(3D,cosz) \\
  2    & gap=0          \\
  3    & gap=1--Fveg    \\
 \end{tabular}

 \var{NOAHMP36 alb\_opt:} specifies the snow surface albedo scheme.
 Acceptable values are:

 \begin{tabular}{ll}
 Value & Description \\
  1    & BATS        \\
  2    & CLASS       \\
 \end{tabular}

 \var{NOAHMP36 snf\_opt:} specifies the rainfall and snowfall
 determination scheme.  Acceptable values are:

 \begin{tabular} {ll}
 Value & Description \\
  1    & Jordan91    \\
  2    & BATS        \\
  3    & Noah        \\
 \end{tabular}

 \var{NOAHMP36 tbot\_opt:} specifies the lower boundary of soilu
 temperature scheme.  Acceptable values are:

 \begin{tabular} {ll}
 Value & Description \\
  1    & zero-flux   \\
  2    & Noah        \\
 \end{tabular}

 \var{NOAHMP36 stc\_opt:} specifies the snow and soil temperature
 time scheme.  Acceptable values are:

 \begin{tabular}{ll}
 Value & Description    \\
  1    & semi-implicit  \\
  2    & fully implicit \\
 \end{tabular}

 \var{NOAHMP36 soil layer thickness:} specifies the thicknesses of the NoahMP layers.

 \var{NOAHMP36 sc\_idx:} specifies the NoahMP soil color type,
 an integer index from 1 to 8.  Defaults to 4.

 \var{NOAHMP36 CZIL option (iz0tlnd):} specifes whether to use the Chen adjustment
 of Czil.  Defaults to 0.
 Acceptable values are:

 \begin{tabular} {ll}
 Value & Description \\
  0    & do not use  \\
  1    & use         \\
 \end{tabular}

 \var{NOAHMP36 restart file:} specifies the NoahMP restart file.

 \var{NOAHMP36 restart file format:} specifies the NoahMP restart file
 format, default: netcdf

 \var{NOAHMP36 initial albold:} specifies the NoahMP initial albold
 (albedo of previous time step).

 \var{NOAHMP36 initial sneqvo:} specifies the NoahMP initial snow mass
 at the last time step (mm).

 \var{NOAHMP36 initial stc:} specifies the NoahMP initial soil
 temperature.

 \var{NOAHMP36 initial sh2o:} specifies the NoahMP initial soil
 moisture (liquid part).

 \var{NOAHMP36 initial smc:} specifies the NoahMP initial soil
 moisture (total).

 \var{NOAHMP36 initial tah:} specifies the NoahMP initial canopy air
 temperature (K).

 \var{NOAHMP36 initial eah:} specifies the NoahMP initial canopy air
 vapor pressure (Pa).

 \var{NOAHMP36 initial fwet:} specifies the NoahMP initial wetted or
 snowed fraction of canopy.

 \var{NOAHMP36 initial canliq:} specifies the NoahMP intial intercpted
 liquid water (mm).

 \var{NOAHMP36 initial canice:} specifies the NoahMP initial intercepted
 ice mass (mm).

 \var{NOAHMP36 initial tv:} specifies the NoahMP intial vegetation
 temperature (K).

 \var{NOAHMP36 initial tg:} specifies the NoahMP intial ground
 temperature (skin temperature) (K).

 \var{NOAHMP36 initial qsnow:} specifies the NoahMP intial snowfall on
 the ground (mm/s).

 \var{NOAHMP36 initial snowh:} specifies the NoahMP initial snow
 depth (m).

 \var{NOAHMP36 initial sneqv:} specifies the NoahMP initial snow water
 equivalent (mm).

 \var{NOAHMP36 initial zwt:} specifies the NoahMP initial depth to
 water table (m).

 \var{NOAHMP36 initial wa:} specifies the NoahMP initial water storage
 in aquifer (mm).

 \var{NOAHMP36 initial wt:} specifies the NoahMP initial water in
 aquifer and saturated soil (mm).

 \var{NOAHMP36 initial wslake:} specifies the NoahMP intial lake water
 storage(mm).

 \var{NOAHMP36 initial lfmass:} specifies the NoahMP intial leaf
 mass (used only for dveg\_opt=2) (g/m2).

 \var{NOAHMP36 initial rtmass:} specifies the NoahMP intial mass of
 fine roots (g/m2).

 \var{NOAHMP36 initial stmass:} specifies the NoahMP intial stem
 mass (g/m2).

 \var{NOAHMP36 initial wood:} specifies the NoahMP intial mass of
 wood (including woody roots) (g/m2).

 \var{NOAHMP36 initial stblcp:} specifies the NoahMP intial stable
 carbon in deep soil (g/m2).

 \var{NOAHMP36 initial fastcp:} specifies the NoahMP intial short-lived
 carbon in shallow soil (g/m2).

 \var{NOAHMP36 initial lai:} specifies the NoahMP intial leaf area
 index.

 \var{NOAHMP36 initial sai:} specifies the NoahMP intial stem area
 index.

 \var{NOAHMP36 initial cm:} specifies the NoahMP intial momentum drag
 coefficient (s/m).

 \var{NOAHMP36 initial ch:} specifies the NoahMP intial sensible heat
 exchange coefficient (s/m).

 \var{NOAHMP36 initial tauss:} specifies the NoahMP intial snow aging
 term.

 \var{NOAHMP36 initial smcwtd:} specifies the NoahMP intial soil water
 content between bottom of the soil and water table (m3/m3).

 \var{NOAHMP36 initial deeprech:} specifies the NoahMP intial recharge
 to or from the water table when deep (m).

 \var{NOAHMP36 initial rech:} specifies the NoahMP intial recharge to
 or from the water table when shallow (m).

 \var{NOAHMP36 initial zlvl:} specifies the NoahMP intial reference
 height of temperature and humidity (m).
 

 \begin{Verbatim}[frame=single]
NOAHMP36 model timestep: "15mn"
NOAHMP36 restart output interval: 1da 
NOAHMP36 number of soil layers:      4   
NOAHMP36 number of snow layers:      3   
NOAHMP36 landuse parameter table: "./input/noahmp_params/VEGPARM.TBL"    
NOAHMP36 soil parameter table:  "./input/noahmp_params/SOILPARM.TBL"    
NOAHMP36 general parameter table:  "./input/noahmp_params/GENPARM.TBL"    
NOAHMP36 MP parameter table: "./input/noahmp_params/MPTABLE.TBL"    
NOAHMP36 option of vegetation model:  4    
NOAHMP36 option of canopy stomatal resistance:  1    
NOAHMP36 option of soil moisture factor for stomatal resistance:  1    
NOAHMP36 option of runoff and groundwater:  1    
NOAHMP36 option of surface layer drag coefficient:  2    
NOAHMP36 option of supercooled liquid water:  1    
NOAHMP36 option of frozen soil permeability:  1    
NOAHMP36 option of radiation transfer:  1    
NOAHMP36 option of snow surface albedo:  2    
NOAHMP36 option of rainfall and snowfall:  1    
NOAHMP36 option of lower boundary of soil temperature:  1    
NOAHMP36 option of snow and soil temperature time scheme:  1    
NOAHMP36 soil layer thickness:   0.1 0.3 0.6 1.0    
NOAHMP36 soil color index:   4    
NOAHMP36 CZIL option (iz0tlnd):  0    
NOAHMP36 zlvl:     6.0   
NOAHMP36 zlvl_wind: 6.0    
NOAHMP36 restart file: ./OUTPUT/opt_dveg_4/SURFACEMODEL/201212/LIS_RST_NOAHMP36_201212312100.d01.nc
NOAHMP36 restart file format: "netcdf"
NOAHMP36 initial albold:  0.1899999976    
NOAHMP36 initial sneqvo:  0.0    
NOAHMP36 initial stc:   266.09950  274.0445000  276.8954000  279.915200    
NOAHMP36 initial sh2o:   0.2981597   0.2940254    0.2713114    0.3070948    
NOAHMP36 initial smc:   0.2981597    0.2940254    0.2713114    0.3070948    
NOAHMP36 initial tah:   263.94998    
NOAHMP36 initial eah:   261.68518    
NOAHMP36 initial fwet:  0.0    
NOAHMP36 initial canliq: 0.0003935303    
NOAHMP36 initial canice: 0.0000000000    
NOAHMP36 initial tv: 263.6908874512    
NOAHMP36 initial tg: 263.6908874512    
NOAHMP36 initial qsnow:  0.0    
NOAHMP36 initial snowh:  0.0010600531    
NOAHMP36 initial sneqv:  0.0002095700    
NOAHMP36 initial zwt: 2.5000000000    
NOAHMP36 initial wa: 4900.0000000000    
NOAHMP36 initial wt: 4900.0000000000    
NOAHMP36 initial wslake: 0.0000000000    
NOAHMP36 initial lfmass: 9.0000000000    
NOAHMP36 initial rtmass: 500.0000000000    
NOAHMP36 initial stmass: 3.3299999237    
NOAHMP36 initial wood: 500.0000000000    
NOAHMP36 initial stblcp: 1000.0000000000    
NOAHMP36 initial fastcp: 1000.0000000000    
NOAHMP36 initial lai: 0.5000000000    
NOAHMP36 initial sai: 0.1000000015    
NOAHMP36 initial cm: 0.0000000000    
NOAHMP36 initial ch: 0.0000000000    
NOAHMP36 initial tauss: 0.0    
NOAHMP36 initial smcwtd: 0.0 
NOAHMP36 initial deeprech: 0.0    
NOAHMP36 initial rech: 0.0    
NOAHMP36 initial zlvl: 6.0  
 \end{Verbatim}





 
 \subsubsection{CLM 2.0} \label{sssec:lsm_clm2}
 

 
 \var{CLM model timestep:} specifies the timestep for the run.

 See Section \ref{ssec:timeinterval} for a description
 of how to specify a time interval.

 \var{CLM restart output interval:} defines the restart
 writing interval for CLM. The typical value used in the
 LIS runs is 24 hours (1da).

 See Section \ref{ssec:timeinterval} for a description
 of how to specify a time interval.

 \var{CLM restart file:} specifies the CLM active restart file.

 \var{CLM vegetation parameter file:} specifies vegetation type
 parameters look-up table.

 \var{CLM canopy height table:} specifies the canopy top and
 bottom heights (for each vegetation type) look-up table.

 \var{CLM initial soil moisture:} specifies the initial volumetric
 soil moisture wetness used in the cold start runs.

 \var{CLM initial soil temperature:} specifies the initial soil
 temperature in Kelvin used in the cold start runs.

 \var{CLM initial snow mass:} specifies the initial snow mass used
 in the cold start runs.
 

 \begin{Verbatim}[frame=single]
CLM model timestep:                   15mn
CLM restart output interval:          1da
CLM restart file:                     ./clm.rst
CLM vegetation parameter table:       ./input/clm_parms/umdvegparam.txt
CLM canopy height table:              ./input/clm_parms/clm2_ptcanhts.txt
CLM initial soil moisture:            0.45
CLM initial soil temperature:         290.0
CLM initial snow mass:                0.0
 \end{Verbatim}

 
 \subsubsection{VIC 4.1.1} \label{sssec:lsm_vic411}
 

 
 \var{VIC411 model timestep:} specifies the timestep for the run.

 See Section \ref{ssec:timeinterval} for a description
 of how to specify a time interval.

 \var{VIC411 model step interval:} defines the model step interval
 for VIC, in seconds.

 VIC uses two timestep variables to control its execution.
 \var{VIC411 model step interval:} corresponds to VIC's
 \var{TIME\_STEP} variable.  VIC's \var{VIC411 model timestep:}
 corresponds to VIC's \var{SNOW\_STEP} variable.

 For water balance mode, \var{VIC411 model step interval:}
 must be set to 86400.

 For energy balance mode, \var{VIC411 model step interval:}
 must be set to VIC's \var{VIC411 model timestep:}.

 Note that for both energy balance mode and water balance mode,
 VIC's \var{VIC411 model timestep:}, in seconds, must be both a
 multiple of 3600 and a factor of 86400.
 Simply stated VIC's \var{VIC411 model timestep:} must
 correspond to 1, 2, 3, 4, 6, 12, or 24 hours.

 \var{VIC411 restart output interval:} defines the restart
 writing interval for VIC. The typical value used in the
 LIS runs is 24 hours (1da).

 See Section \ref{ssec:timeinterval} for a description
 of how to specify a time interval.

 \var{VIC411 veg tiling scheme:} specifies whether VIC or LIS
 will perform vegetation-based sub-grid tiling.

 For LIS sub-grid tiling, tiling is based on vegetation fractions
 from the \var{landcover file:} file.

 For VIC sub-grid tiling, tiling is based on vegetation fractions
 from the \var{VEGPARAM} file.
 Acceptable values are:

 \begin{tabular}{ll}
 Value & Description \\
 0     & VIC tiling  \\
 1     & LIS tiling  \\
 \end{tabular}

 \var{VIC411 global parameter file:} This is VIC's
 configuration file.  Please see VIC's documentation at: \\
 \hyperref{http://www.hydro.washington.edu/Lettenmaier/Models/VIC/index.shtml}{}{}{http://www.hydro.washington.edu/Lettenmaier/Models/VIC/index.shtml} \\
 for more information.

 \var{VIC411 total number of veg types:} specifies the
 number of vegetation classes in VIC's landcover dataset
 (\var{VEGPARAM}).

 \var{VIC411 convert units:} Used for testing; set this to 1.
 

 \begin{Verbatim}[frame=single]
VIC411 model timestep:             1hr
VIC411 model step interval:        3600
VIC411 restart output interval:    1da
VIC411 veg tiling scheme:          1
VIC411 global parameter file:      ./input/vic411_global_file_nldas2_testcase
VIC411 total number of veg types:  13
VIC411 convert units:              1
 \end{Verbatim}

 
 \subsubsection{VIC 4.1.2} \label{sssec:lsm_vic412}
 

 
 \var{VIC412 model timestep:} specifies the timestep for the run.

 See Section \ref{ssec:timeinterval} for a description
 of how to specify a time interval.

 \var{VIC412 model step interval:} defines the model step
 interval for VIC, in seconds.

 VIC uses two timestep variables to control its execution.
 \var{VIC412 model step interval:} corresponds to VIC's
 \var{TIME\_STEP} variable.  VIC's \var{VIC412 model timestep:}
 corresponds to VIC's \var{SNOW\_STEP} variable.

 For water balance mode, \var{VIC412 model step interval:}
 must be set to 86400.

 For energy balance mode, \var{VIC412 model step interval:}
 must be set to VIC's \var{VIC412 model timestep:}.

 Note that for both energy balance mode and water balance mode,
 VIC's \var{VIC412 model timestep:}, in seconds, must be both a
 multiple of 3600 and a factor of 86400.
 Simply stated VIC's \var{VIC412 model timestep:} must
 correspond to 1, 2, 3, 4, 6, 12, or 24 hours.

 \var{VIC412 restart output interval:} defines the restart
 writing interval for VIC. The typical value used in the
 LIS runs is 24 hours (1da).

 See Section \ref{ssec:timeinterval} for a description
 of how to specify a time interval.

 \var{VIC412 restart file:} specifies the VIC 4.1.2 active
 restart file.

 \var{VIC412 restart file format:} specifies the format for the
 VIC 4.1.2 restart file.
 Acceptable values are:
 \begin{tabular}{ll}
 Value  & Description   \\
 binary & binary format \\
 netcdf & netCDF format \\
 \end{tabular}

 \var{VIC412 veg tiling scheme:} specifies whether VIC or LIS
 will perform vegetation-based sub-grid tiling.

 For LIS sub-grid tiling, tiling is based on vegetation fractions
 from the \var{landcover file:} file.

 For VIC sub-grid tiling, tiling is based on vegetation fractions
 from the \var{VEGPARAM} file.
 Acceptable values are:

 \begin{tabular}{ll}
 Value & Description \\
 0     & VIC tiling  \\
 1     & LIS tiling  \\
 \end{tabular}

 \var{VIC412 total number of veg types:} specifies the
 number of vegetation classes in VIC's landcover dataset
 (\var{VEGPARAM}).

 \var{VIC412 convert units:} Used for testing; set this to 1.

 The VIC global parameter file is no longer needed. All configuration
 settings are in \file{lis.config} for VIC. Specifications are the same
 as the global parameter file of standalone VIC except option names
 come with a prefix ``VIC412\_'', in which 412 is the version number
 of the VIC model. For example, the number of VIC soil layers is
 specified as the following:

 \var{VIC412\_NLAYER: 3}

 See VIC's documentation at: \\
 \hyperref{http://www.hydro.washington.edu/Lettenmaier/Models/VIC/index.shtml}{}{}{http://www.hydro.washington.edu/Lettenmaier/Models/VIC/index.shtml} \\
 for more information about configuring VIC.
 

 \begin{Verbatim}[frame=single]
VIC412 model timestep:             1hr
VIC412 model step interval:        3600
VIC412 restart file:               ./vic412.rst
VIC412 restart file format:        "binary"
VIC412 restart output interval:    1da
VIC412 veg tiling scheme:          1
VIC412 total number of veg types:  13
VIC412 convert units:              1
 \end{Verbatim}




 
 \subsubsection{Mosaic} \label{sssec:lsm_mosaic}
 

 
 \var{Mosaic model timestep:} specifies the timestep for the run.

 See Section \ref{ssec:timeinterval} for a description
 of how to specify a time interval.

 \var{Mosaic restart output interval:} defines the restart
 writing interval for Mosaic.  The typical value used in the
 LIS runs is 24 hours (1da).

 See Section \ref{ssec:timeinterval} for a description
 of how to specify a time interval.

 \var{Mosaic restart file:} specifies the Mosaic active restart file.

 \var{Mosaic vegetation parameter table:} specifies the vegetation
 parameters look-up table.

 \var{Mosaic monthly vegetation parameter table:} specifies the
 monthly vegetation parameters look-up table.

 \var{Mosaic soil parameter table:} specifies the soil
 parameters look-up table.

 \var{Mosaic number of soil classes:} specifies the number of soil
 classes.
 Acceptable values are:

 \begin{tabular}{ll}
 Value & Description \\
  11   & FAO         \\
 \end{tabular}

 \var{Mosaic Depth of Layer 1 (m):} specifies the depth in meters
 of layer 1.

 \var{Mosaic Depth of Layer 2 (m):} specifies the depth in meters
 of layer 2.

 \var{Mosaic Depth of Layer 3 (m):} specifies the depth in meters
 of layer 3.

 \var{Mosaic initial soil moisture:} specifies the initial soil
 moisture.

 \var{Mosaic initial soil temperature:} specifies the initial soil
 temperature in Kelvin.

 \var{Mosaic use forcing data observation height:} specifies whether
 to use observation height from the forcing dataset.

 Acceptable values are:

 \begin{tabular}{ll}
 Value & Description                                             \\
 0     & Do not use observation height from forcing              \\
 1     & Use observation height from forcing                     \\
 \end{tabular}

 \var{Mosaic use forcing data aerodynamic conductance:} specifies
 whether to use aerodynamic conductance field from the forcing
 dataset.

 Acceptable values are:

 \begin{tabular}{ll}
 Value & Description                                            \\
 0     & Do not use aerodynamic conductance from forcing data   \\
 1     & Use aerodynamic conductance from forcing dataset       \\
 \end{tabular}

 \var{Mosaic use distributed soil depth map:} specifies
 whether to use a distributed soil depth map.

 Acceptable values are:

 \begin{tabular}{ll}
 Value & Description                           \\
 0     & Do not use distributed soil depth map \\
 1     & Use distributed soil depth map        \\
 \end{tabular}
 

 \begin{Verbatim}[frame=single]
Mosaic model timestep:                     15mn
Mosaic restart output interval:            1da
Mosaic restart file:                       ./mosaic.rst
Mosaic vegetation parameter table:         ./input/mos_parms/mosaic_vegparms_umd.txt
Mosaic monthly vegetation parameter table: ./input/mos_parms/mosaic_monthlyvegparms_umd.txt
Mosaic soil parameter table:               ./input/mos_parms/mosaic_soilparms_fao.txt
Mosaic number of soil classes:             11
Mosaic Depth of Layer 1 (m):               0.02
Mosaic Depth of Layer 2 (m):               1.48
Mosaic Depth of Layer 3 (m):               2.00
Mosaic initial soil moisture:              0.3
Mosaic initial soil temperature:           290
Mosaic use forcing data observation height:       0
Mosaic use forcing data aerodynamic conductance:  0
Mosaic use distributed soil depth map:            0
 \end{Verbatim}

 
 \subsubsection{HySSiB} \label{sssec:lsm_hyssib}
 

 
 \var{HYSSIB model timestep:} specifies the timestep for the run.

 See Section \ref{ssec:timeinterval} for a description
 of how to specify a time interval.

 \var{HYSSIB restart output interval:} defines the restart
 writing interval for HySSiB. The typical value used in the
 LIS runs is 24 hours (1da).

 See Section \ref{ssec:timeinterval} for a description
 of how to specify a time interval.

 \var{HYSSIB restart file:} specifies the HySSiB active restart file.

 \var{HYSSIB vegetation parameter table:} specifies the HySSiB static
 vegetation parameter table file.

 \var{HYSSIB albedo parameter table:} specifies the HySSiB static
 albedo parameter table file.

 \var{HYSSIB topography stand dev file:} specifies the HySSiB topography
 standard deviation file.

 \var{HYSSIB number of vegetation parameters:} specifies the
 number of vegetation parameters.

 \var{HYSSIB number of vegetation parameters:} specifies the
 number of monthly vegetation parameters.

 \var{HYSSIB reference height for forcing T and q:} specifies the
 height of the forcing T and q variables used from the forcing;
 specifying a negative value will use the height from the forcing
 data, provided it is available.

 \var{HYSSIB reference height for forcing u and v:} specifies the
 height of the forcing u and v variables used from the forcing;
 specifying a negative value will use the height from the forcing
 data, provided it is available.

 \var{HYSSIB initial soil moisture:} specifies the
 initial soil moisture.

 \var{HYSSIB initial soil temperature:} specifies the
 initial soil temperature in Kelvin.
 

 \begin{Verbatim}[frame=single]
HYSSIB model timestep:                        15mn
HYSSIB restart output interval:               1mo
HYSSIB restart file:                          ./hyssib.rst
HYSSIB vegetation parameter table:            ./input/hyssib_parms/hyssib_vegparms.bin
HYSSIB albedo parameter table:                ./input/hyssib_parms/hyssib_albedo.bin
HYSSIB topography stand dev file:             ./input/UMD-25KM/topo_std.1gd4r
HYSSIB number of vegetation parameters:       20
HYSSIB number of monthly veg parameters:      11
HYSSIB reference height for forcing T and q:  -1.0      # (m) - negative=use height from forcing data
HYSSIB reference height for forcing u and v:  -1.0      # (m) - negative=use height from forcing data
HYSSIB initial soil moisture:                 0.30
HYSSIB initial soil temperature:              290.0
 \end{Verbatim}


 
 \subsubsection{Catchment Fortuna-2\_5} \label{sssec:lsm_clsmf25}
 

 
 \var{CLSM F2.5 model timestep:} specifies the timestep for the run.

 See Section \ref{ssec:timeinterval} for a description
 of how to specify a time interval.

 \var{CLSM F2.5 restart output interval:} defines the restart
 writing  interval for Catchment Fortuna-2\_5. The typical
 value used in the LIS runs is 24 hours (1da).

 See Section \ref{ssec:timeinterval} for a description
 of how to specify a time interval.

 \var{CLSM F2.5 restart file:} specifies the Catchment active
 restart file.

 \var{CLSM F2.5 top soil layer depth:} specifies the top soil
 layer depth.

 \var{CLSM F2.5 initial soil moisture:} specifies the
 initial volumetric soil moisture. (units $\frac{m^3}{m^3}$)

 \var{CLSM F2.5 initial soil temperature:} specifies the
 initial soil temperature in Kelvin.

 \var{CLSM F2.5 fixed reference height:} specifies the fixed
 reference height.  The default value used for this height by
 the GMAO is 10.0 meters.  This fixed value will only be used
 if a forcing height field is not used in LIS.  If a forcing
 height field is not used, and the height at which the wind
 is observed is known, then the wind height should be used
 for this value.  There is not a separate term available
 for the height of the temperature or humidity forcing.

 \var{CLSM F2.5 turbulence scheme:} specifies the
 turbulence scheme.

 \var{CLSM F2.5 use MODIS albedo flag:} specifies
 whether to use the MODIS scale factor albedo.
 Acceptable values are:

 \begin{tabular}{ll}
 Value & Description                 \\
 0     & Do not use the MODIS albedo \\
 1     & Use the MODIS albedo        \\
 \end{tabular}

 \begin{Verbatim}[frame=single]
CLSM F2.5 model timestep:             30mn
CLSM F2.5 restart output interval:    1da
CLSM F2.5 restart file:               ./clmsf25.rst
CLSM F2.5 top soil layer depth:       0.02
CLSM F2.5 initial soil moisture:      0.30
CLSM F2.5 initial soil temperature:   290.0
CLSM F2.5 fixed reference height:     10.0
CLSM F2.5 turbulence scheme:          0
CLSM F2.5 use MODIS albedo flag:      1
 \end{Verbatim}

 
 \subsubsection{GeoWRSI 2.0} \label{sssec:geowrsi.2}
 

 
 \var{WRSI CalcSOS model run mode:} specifies which model run
 mode to run the model in, either ``SOS'' or ``WRSI''.

 \var{WRSI user input settings file:} specifies the path for the
 WRSI model file to select user-specific WRSI and SOS settings.

 \var{WRSI crop parameter directory:} specifies the path for the
 crop-type parameter files.

 \var{WRSI initial dekad of season:} The crop growing season
 initial timestep (in dekads).

 \var{WRSI final dekad of season:} The crop growing season final
 timestep (in dekads).

 \var{WRSI initial growing season year:} Initial year of the first
 growing season for the LIS-GeoWRSI model run.  For now, should
 match the first year in the lis.config file \var{Starting year:}.

 \var{WRSI final growing season year:} Final year of the last
 growing season for the LIS-GeoWRSI model run.  For now, should
 match the final year in the lis.config file \var{Ending year:}.

 \var{WRSI number of growing seasons:} Set the number of growing
 seasons to have GeoWRSI run over (default value is 1).

 \var{WRSI model timestep:} specifies the timestep for the run.

 See Section \ref{ssec:timeinterval} for a description
 of how to specify a time interval.

 \var{WRSI restart output interval:} defines the restart
 writing  interval for WRSI. The typical value used in the
 LIS-WRSI runs is 10-day, or 1-dekad.

 See Section \ref{ssec:timeinterval} for a description
 of how to specify a time interval.

 \var{WRSI restart file:} specifies the WRSI active restart file.
 

 \begin{Verbatim}[frame=single]
WRSI CalcSOS model run mode:        SOS
WRSI user input settings file:      ./wrsi_inputs/EA_Oct2Feb/GeoWRSI_userSettings.txt
WRSI crop parameter directory:      ./wrsi_inputs/crops
WRSI initial dekad of season:       25
WRSI final dekad of season:         6
WRSI initial growing season year:   2009
WRSI final growing season year:     2010
WRSI number of growing seasons:     1
WRSI model timestep:                "1da"
WRSI restart output interval:       "1da"
WRSI restart file:                  "none"
 \end{Verbatim}




 
 \subsubsection{RDHM 3.5.6} \label{sssec:lsm_rdhm356}
 

 
 \var{RDHM356 model timestep:} specifies the timestep for the run.

 See Section \ref{ssec:timeinterval} for a description
 of how to specify a time interval.

 \var{RDHM356 restart output interval:} defines the restart
 writing interval for RDHM356.

 See Section \ref{ssec:timeinterval} for a description
 of how to specify a time interval.

 \var{RDHM356 TempHeight:} specifies the observation height of
 the temperature and humidity fields, in meters.

 \var{RDHM356 WindHeight:} specifies the observation height of
 the wind field, in meters.

 \var{RDHM356 DT\_SAC\_SNOW17:} specifies the timestep
 of SacHTET and Snow-17 in seconds.
 \highlight{This must be \var{RDHM356 model timestep:}
 specified in seconds.}

 \var{RDHM356 DT\_FRZ:} specifies the timestep of the frozen
 soil model, in seconds.

 \var{RDHM356 FRZ\_VER\_OPT:} specifies the version number of
 the frozen soil model.
 Acceptable values are:

 \begin{tabular}{ll}
 Value & Description \\
 1     & Old version \\
 2     & New version \\
 \end{tabular}

 Note, if set to 1, zero snow depth causes problems.

 \var{RDHM356 SNOW17\_OPT:} SNOW-17 option
 Acceptable values are:

 \begin{tabular}{ll}
 Value & Description        \\
 1     & Use Snow-17        \\
 else  & Do not use Snow-17 \\
 \end{tabular}

 \var{RDHM356 SACHTET\_OPT:} SAC-HTET option
 Acceptable values are:

 \begin{tabular}{ll}
 Value & Description         \\
 1     & Use Sac-HTET        \\
 else  & Do not use Sac-HTET \\
 \end{tabular}

 \var{RDHM356 NSTYP:} specifies the number of soil types.

 \var{RDHM356 NVTYP:} specifies the number of vegetation types.

 \var{RDHM356 NDINTW:} specifies the number of desired soil
 layers for total and liquid soil moisture.

 \var{RDHM356 NDSINT:} specifies the number of desired soil
 layers for soil temperature.

 \var{RDHM356 NORMALIZE:} specifies whether to normalize total
 and liquid soil moisture output.
 Acceptable values are:

 \begin{tabular}{ll}
 Value & Description      \\
 0     & Do not normalize \\
 1     & Normalize        \\
 \end{tabular}

 \var{RDHM356 DSINTW:} specifies the thickness of the desired
 soil layers for liquid and total soil moisture.

 \var{RDHM356 DSINT:} specifies the thickness of the desired
 soil layers for soil temperature.

 \var{RDHM356 PETADJ\_MON:} specifies the adjustment of
 potential evapotranspiration for 12 months.

 \var{RDHM356 CZIL:} specifies the Zilitinkevich parameter,
 range: [0.0, 1.0].

 \var{RDHM356 FXEXP:} specifies the bare soil evaporation
 exponential non-linear parameter.

 \var{RDHM356 vegRCMAX:} specifies the maximum stomatal
 resistance, in s/m.

 \var{RDHM356 PC:} specifies the plant coefficient.

 \var{RDHM356 PET\_OPT:} specifies the potential
 evapotranspiration scheme.
 Acceptable values are:

 \begin{tabular}{ll}
 Value & Description                   \\
 $<0$  & Use energy-based Penman       \\
 $0$   & Use non-Penman-based ETP      \\
 $>0$  & Use empirical Penman equation \\
 \end{tabular}

 \var{RDHM356 TOPT:} specifies the optimum air temperature,
 in Kelvin.

 \var{RDHM356 RDST:} specifies the tension water redistribution
 scheme.
 Acceptable values are:

 \begin{tabular}{ll}
 Value & Description             \\
 0     & Use OHD version of SRT (uses reference gradient instead
         of actual)              \\
 1     & Use Noah version of SRT \\
 \end{tabular}

 \var{RDHM356 thresholdRCMIN:} constant for alternating
 RCMIN (0.5) (s/m).

 \var{RDHM356 SFCREF:} specifies the reference wind speed
 for potential evapotranspiration adjustment, in m/s.

 \var{RDHM356 BAREADJ:} specifies the Ek-Chen evaporation
 threshold switch.

 \var{RDHM356 SNOW17\_SWITCH:} specifies liquid water freezing
 version.
 Acceptable values are:

 \begin{tabular}{ll}
 Value & Description      \\
 0     & Victor's version \\
 1     & Eric's version   \\
 \end{tabular}

 \var{RDHM356 restart file:} specifies the RDHM 3.5.6 active
 restart file.

 \var{RDHM356 restart file format:} specifies the format of
 the RDHM 3.5.6 restart file.
 Acceptable values are:

 \begin{tabular}{ll}
 Value  & Description                     \\
 binary & read/write binary restart files \\
 netcdf & read/write netCDF restart files \\
 \end{tabular}

 \var{RDHM356 tmxmn file:} NetCDF file of daily maximum and
 minimum temperature (F).

 \var{RDHM356 initial UZTWC (ratio):} specifies the initial
 upper zone tension water storage content.

 \var{RDHM356 initial UZFWC (ratio):} specifies the initial
 upper zone free water storage content.

 \var{RDHM356 initial LZTWC (ratio):} specifies the initial
 lower zone tension water storage content.

 \var{RDHM356 initial LZFSC (ratio):} specifies the initial
 lower zone supplemental free water storage content.

 \var{RDHM356 initial LZFPC (ratio):} specifies the initial
 lower zone primary free water storage content.

 \var{RDHM356 initial ADIMC (ratio):} specifies the initial
 additional impervious area content.

 \var{RDHM356 initial TS0:} specifies the initial first soil
 layer temperature, in Celsius.

 \var{RDHM356 initial UZTWH (ratio):} specifies the initial
 unfrozen upper zone tension water.

 \var{RDHM356 initial UZFWH (ratio):} specifies the initial
 unfrozen upper zone free water.

 \var{RDHM356 initial LZTWH (ratio):} specifies the initial
 unfrozen lower zone tension water.

 \var{RDHM356 initial LZFSH (ratio):} specifies the initial
 unfrozen lower zone supplemental free water.

 \var{RDHM356 initial LZFPH (ratio):} specifies the initial
 unfrozen lower zone primary free water.

 \var{RDHM356 initial SMC:} specifies the initial volumetric
 content of total soil moisture for each layer.

 \var{RDHM356 initial SH2O:} specifies the initial volumetric
 content of liquid soil moisture for each layer.

 \var{RDHM356 initial WE:} specifies the initial snow water
 equivalent without liquid water, in mm.

 \var{RDHM356 initial LIQW:} specifies the initial liquid water
 in snow.

 \var{RDHM356 initial NEGHS:} specifies the initial negative
 snow heat, in mm.

 \var{RDHM356 initial TINDEX:} specifies the initial antecedent
 temperature index.

 \var{RDHM356 initial ACCMAX:} specifies the initial accumulated
 snow water temperature, including liquid, in Celsius.

 \var{RDHM356 initial SNDPT:} specifies the initial snow depth,
 in cm.

 \var{RDHM356 initial SNTMP:} specifies the initial average
 snow temperature.

 \var{RDHM356 initial SB:} specifies the last highest snow
 water equivalent before any snow fall, in mm.

 \var{RDHM356 initial SBAESC:} specifies the initial extent
 of snow cover during melt and new snow fall.

 \var{RDHM356 initial SBWS:} specifies the initial snow water
 storage during melt and new snow fall, in mm.

 \var{RDHM356 initial STORAGE:} specifies the initial snow
 liquid water attenuation storage, in mm.

 \var{RDHM356 initial AEADJ:} specifies the initial adjusted
 areal snow cover fraction [0, 1].

 \var{RDHM356 initial EXLAG:} specifies the initial array of
 lagged liquid water values.

 \var{RDHM356 initial NEXLAG:} specifies the number of
 coordinates in the lagged liquid water array.

 \var{RDHM356 initial TA\_PREV:} specifies the air temperature
 of previous timestep, in Celsius.
 

 \begin{Verbatim}[frame=single]
RDHM356 model timestep:          "1hr"
RDHM356 restart output interval: "1hr"
RDHM356 TempHeight:               2.0
RDHM356 WindHeight:              10.0
RDHM356 DT_SAC_SNOW17:           3600
RDHM356 DT_FRZ:                  1800
RDHM356 FRZ_VER_OPT:             1
RDHM356 SNOW17_OPT:              1
RDHM356 SACHTET_OPT:             1
RDHM356 NSTYP:                   12
RDHM356 NVTYP:                   14
RDHM356 NDINTW:                  5
RDHM356 NDSINT:                  5
RDHM356 NORMALIZE:               1
RDHM356 DSINTW:                  0.05 0.25 0.60 0.75 1.00
RDHM356 DSINT:                   0.05 0.25 0.60 0.75 1.00
RDHM356 PETADJ_MON:              1.0 1.0 1.0 1.0 1.0 1.0 1.0 1.0 1.0 1.0 1.0 1.0
RDHM356 CZIL:                    0.12
RDHM356 FXEXP:                   2.0
RDHM356 vegRCMAX:                5000
RDHM356 PC:                      -1
RDHM356 PET_OPT:                 -1
RDHM356 TOPT:                    298
RDHM356 RDST:                    1
RDHM356 thresholdRCMIN:          0.5
RDHM356 SFCREF:                  4.0
RDHM356 BAREADJ:                 0.230000004
RDHM356 SNOW17_SWITCH:           1
RDHM356 restart file:            "none"
RDHM356 restart file format:     "netcdf"
RDHM356 tmxmn file:              "none"
RDHM356 initial UZTWC (ratio):   0.55
RDHM356 initial UZFWC (ratio):   0.14
RDHM356 initial LZTWC (ratio):   0.56
RDHM356 initial LZFSC (ratio):   0.11
RDHM356 initial LZFPC (ratio):   0.46
RDHM356 initial ADIMC (ratio):   1.0
RDHM356 initial TS0:             4.0
RDHM356 initial UZTWH (ratio):   0.1
RDHM356 initial UZFWH (ratio):   0.1
RDHM356 initial LZTWH (ratio):   0.1
RDHM356 initial LZFSH (ratio):   0.1
RDHM356 initial LZFPH (ratio):   0.1
RDHM356 initial SMC:             0.35 0.35 0.35 0.35 0.35 0.35
RDHM356 initial SH2O:            0.35 0.35 0.35 0.35 0.35 0.35
RDHM356 initial WE:              0
RDHM356 initial LIQW:            0
RDHM356 initial NEGHS:           0
RDHM356 initial TINDEX:          0
RDHM356 initial ACCMAX:          0
RDHM356 initial SNDPT:           0
RDHM356 initial SNTMP:           0
RDHM356 initial SB:              0
RDHM356 initial SBAESC:          0
RDHM356 initial SBWS:            0
RDHM356 initial STORAGE:         0
RDHM356 initial AEADJ:           0
RDHM356 initial EXLAG:           0 0 0 0 0 0 0
RDHM356 initial NEXLAG:          7
RDHM356 initial TA_PREV:         0
 \end{Verbatim}




 
 \subsection{Irrigation} \label{ssec:irrigation}
 

 
 \var{Irrigation scheme:} specifies the name of the irrigation
 scheme to use.
 Acceptable values are:

 \begin{tabular}{ll}
 Value     & Description             \\
 none      & No irrigation scheme    \\
 Sprinkler & Demand sprinkler scheme \\
 Flood     & Demand flood scheme \\
 \end{tabular}

 \var{Irrigation output interval:} defines the output writing
 interval for irrigation.

 See Section \ref{ssec:timeinterval} for a description
 of how to specify a time interval.

 \var{Irrigation threshold:} defines the irrigation trigger
 threshold for the flood and sprinkler irrigation schemes.

 \var{Sprinkler irrigation max root depth file:} specifies the
 location of the max root depth file for sprinkler irrigation.

 \var{Flood irrigation max root depth file:} specifies the

 \var{Drip irrigation max root depth file:} specifies the
 location of the max root depth file for flood irrigation.
 

 \begin{Verbatim}[frame=single]
Irrigation scheme:            "none"
Irrigation output interval:   "1da"
Irrigation threshold:          0.50
Sprinkler irrigation max root depth file:
Flood irrigation max root depth file:
Drip irrigation max root depth file:
 \end{Verbatim}	

 
 \subsection{Routing} \label{ssec:routing}
 

 
 \subsubsection{HYMAP routing} \label{sssec:hymaprouting}
 

 
 \var{HYMAP routing model time step:} specifies the timestep for
 the run.

 See Section \ref{ssec:timeinterval} for a description
 of how to specify a time interval.

 \var{HYMAP routing model output interval:} defines the output
 writing interval for the HYMAP router.

 See Section \ref{ssec:timeinterval} for a description
 of how to specify a time interval.

 \var{HYMAP run in ensemble mode:} specifies whether to run in
 ensemble mode.
 Acceptable values are:

 \begin{tabular}{ll}
 Value & Description                 \\
 0     & Do not run in ensemble mode \\
 1     & Run in ensemble mode        \\
 \end{tabular}

 \var{HYMAP routing method:} specifies the HYMAP routing method
 to use.
 Acceptable values are:

 \begin{tabular}{ll}
 Value     & Description          \\
 kinematic & use kinematic method \\
 diffusive & use diffusive method \\
 \end{tabular}

 \var{HYMAP routing model linear reservoir flag:} specifies whether to
 use model linear reservoir.
 Acceptable values are:

 \begin{tabular}{ll}
 Value & Description \\
 1     & Use         \\
 2     & Do not use  \\
 \end{tabular}

 \var{HYMAP routing model evaporation option:} specifies whether
 to compute evaporation in flood plains.
 Acceptable values are:

 \begin{tabular}{ll}
 Value & Description                                \\
 1     & Compute evaporation in flood plains        \\
 2     & Do not compute evaporation in flood plains \\
 \end{tabular}

 \var{HYMAP routing model start mode:} specifies if a restart mode is
 being used. 
 Acceptable values are:

 \begin{tabular}{ll}
 Value     & Description                                           \\
 restart   & A restart mode is being used                          \\
 coldstart & A cold start mode is being used, no restart file read \\
 \end{tabular}

 \var{HYMAP routing model restart interval:} defines the restart writing 
 interval for the HYMAP router.

 See Section \ref{ssec:timeinterval} for a description
 of how to specify a time interval.

 \var{HYMAP routing model restart file:} specifies the HYMAP router
 active restart file.

 \var{HYMAP routing LIS output directory:} specifies the name of the
 top-level output directory for HYMAP router.
 Acceptable values are any 40 character string.
 

 \begin{Verbatim}[frame=single]
HYMAP routing model time step:
HYMAP routing model output interval:
HYMAP run in ensemble mode:
HYMAP routing method:
HYMAP routing model linear reservoir flag:
HYMAP routing model evaporation option:
HYMAP routing model start mode:
HYMAP routing model restart interval:
HYMAP routing model restart file:
HYMAP routing LIS output directory:
 \end{Verbatim}

 
 \subsubsection{NLDAS routing} \label{sssec:nldasrouting}
 

 
 \var{NLDAS routing model output interval:} defines the output
 writing interval for the NLDAS router.

 See Section \ref{ssec:timeinterval} for a description
 of how to specify a time interval.

 \var{NLDAS routing model restart interval:} defines the restart
 writing interval for the NLDAS router.

 See Section \ref{ssec:timeinterval} for a description
 of how to specify a time interval.

 \var{NLDAS routing internal unit hydrograph file:} specifies the
 internal unit hydrograph file.

 \var{NLDAS routing transport unit hydrograph file:} specifies
 the transport unit hydrograph file.

 \var{NLDAS routing coordinates order file:} specifies the
 coordinates order file.

 \var{NLDAS routing initial condition for runoff:} specifies the
 initial condition for runoff file.

 \var{NLDAS routing initial condition for transport:} specifies
 the initial condition for transport file.

 \var{NLDAS routing model start mode:} specifies if a restart mode is
 being used. 
 Acceptable values are:

 \begin{tabular}{ll}
 Value     & Description                                           \\
 restart   & A restart mode is being used                          \\
 coldstart & A cold start mode is being used, no restart file read \\
 \end{tabular}

 \var{NLDAS routing model restart file:} specifies the NLDAS router
 active restart file.
 

 \begin{Verbatim}[frame=single]
NLDAS routing model output interval:
NLDAS routing model restart interval:
NLDAS routing internal unit hydrograph file:
NLDAS routing transport unit hydrograph file:
NLDAS routing coordinates order file:
NLDAS routing initial condition for runoff:
NLDAS routing initial condition for transport:
NLDAS routing model start mode:
NLDAS routing model restart file:
 \end{Verbatim}

 
 \subsection{Model output configuration} \label{ssec:outputconfig}
 

 
 The output start time is used to define when to begin writing
 model output.  Any value not defined will default to the corresponding
 LIS start time.  The output start time does not affect restart writing.
 Restart files are written according to the LIS start time and the model
 restart output interval value.

 The output start time is specified in the following format: 

 \begin{tabular}{lll}
 Variable & Value & Description                                \\
 \var{Output start year:}    & integer 2001 -- present & 
                               specifying output start year     \\
 \var{Output start month:}   & integer 1 -- 12 & 
                               specifying output start month    \\
 \var{Output start day:}     & integer 1 -- 31 & 
                               specifying output start day      \\
 \var{Output start hour:}    & integer 0 -- 23 &
                               specifying output start hour     \\
 \var{Output start minutes:} & integer 0 -- 59 &
                               specifying output start minute   \\
 \var{Output start seconds:} & integer 0 -- 59 &
                               specifying output start second   \\
 \end{tabular}

 Writing output may be restricted to a specified time with
 respect to any year.  To restrict output to a specified time,
 you must set
 \var{Output at specific time only:} to 1, and then you must specify the
 specific output writing time.  If you choose not to restrict output
 writing to a specified time, then you do not have to set the
 specific output writing time variables.

 \var{Output at specific time only:} specifies whether to write
 output only at a specified time.  Defaults to 0.
 Acceptable values are:

 \begin{tabular}{ll}
 Value & Description                                \\
 0     & Do not restrict output to a specified time \\
 1     & Restrict output to a specified time        \\
 \end{tabular}

 The specific output writing time is specified in the following format: 

 \begin{tabular}{lll}
 Variable & Value & Description                                   \\
 \var{Specific output writing time (month):}  & integer 1 -- 12 & 
                                    specifying output month       \\
 \var{Specific output writing time (day):}    & integer 1 -- 31 & 
                                    specifying output day         \\
 \var{Specific output writing time (hour):}   & integer 0 -- 23 &
                                    specifying output hour        \\
 \var{Specific output writing time (minute):} & integer 0 -- 59 &
                                    specifying output minute      \\
 \var{Specific output writing time (second):} & integer 0 -- 59 &
                                    specifying output second      \\
 \end{tabular}
 

 \begin{Verbatim}[frame=single]
Output start year:
Output start month:
Output start day:
Output start hour:
Output start minutes:
Output start seconds:
Output at specific time only:
Specific output writing time (month):
Specific output writing time (day):
Specific output writing time (hour):
Specific output writing time (minute):
Specific output writing time (second):
 \end{Verbatim}

 
 \var{Model output attributes file:} specifies the attributes to be 
 used for a customizable model output. Please refer to the 
 sample MODEL\_OUTPUT\_LIST.TBL file for the complete specification. 
 

 \begin{Verbatim}[frame=single]
Model output attributes file: './MODEL_OUTPUT_LIST.TBL'
 \end{Verbatim}

 
 \subsection{Defining a time interval} \label{ssec:timeinterval}
 Time interval values must be entered in a format where the timestep 
 value is followed by 2 character string indicating the 
 time units.

 Examples include : 60ss, 30mn, 2hr, 0.5da

 Acceptable values for the timestep units are: 

 \begin{tabular}{ll}
 Value & Description \\
 ss    & seconds     \\
 mn    & minutes     \\
 hr    & hours       \\
 da    & days        \\
 mo    & months      \\
 yr    & years       \\
 \end{tabular}

 Units of months assumes a 30-day month.

 Units of years assumes a 365-day year.
 


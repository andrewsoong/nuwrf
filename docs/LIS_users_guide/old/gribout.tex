\section{GRIB Output Information } \label{sec:writegrib_appendix}

\textbf{Introduction}: \\
This is a package of subroutines to write GRIB data in LIS.  As land surface
models become more sophisticated, more variable metadata will need to be
added. The package currently supports all of the mandatory ALMA output variables for
LSMs.  However, the GRIB output module can easily be extended to support additional
variables.  This new LIS grib interface was adapted
from a module similair to one used in the AFWA AGRMET model.  Supported
LIS projections include Lat/Lon, Lambert Conformal, Polar Stereographic, and
Mercator.  The setup of the GRIB grid description section (GDS section) is handled
automatically by LIS within the domain initialization module. \\
\\
Charles J. Alonge \\
SAIC/NASA GSFC        \\
Winter 2006/2007, and continuing \\
\\
The user interfaces are:  \\
\textbf{SUBROUTINE GRIB1\_SETUP }
 - Initializes variable independent information in the GRIB product definition
 section (PDS). \\
\textbf{SUBROUTINE GRIB1\_FINALIZE }
 - Finalizes the product definition section of the GRIB record by encoding
 variable specific metadata into the output grib record.\\
\textbf{SUBROUTINE DRV\_WRTIEVAR\_GRIB }
 - Writes the grib record (and stats) by gathering the variable from the
  indiviudual MPI tasks (if applicable) and calls the lower level grib
  output routines

\subsection{ SUBROUTINE GRIB1\_SETUP }
- Call: GRIB1\_SETUP(SECT1, INFO, BITMAP, DATE)   \\
- Initialize variable indpendent information in the GRIB product definition
  record (e.g. center, subcenter, time valid, and bitmap flag)  \\
- NOTE!: BITMAP MUST ALWAYS BE TRUE as the code will handle output over all
  gridpoints (including water) or just LIS output over land.\\
\begin{itemize}
 \item[Input: ]
  SECT1 (integer(36)): GRIB PDS array.  Contains variable specific GRIB metadata. \\
  INFO (integer(5)): GRIB packing descriptor, describes the length of each
     section of grib header for low-level grib encoding. \\
  BITMAP (logical) : Determines if a bitmap is used in the grib encoding
     (ALWAYS leave defined as true!!) \\
  DATE (character(10)): Character string containing the date of the grib
     record ( format: YYYYMMDDHH )  \\
 \item[Side Effects:]
  NONE - Do not set Bitmap to false or grib packing will fail.
\end{itemize}

\subsection{ SUBROUTINE GRIB1\_FINALIZE }
- Call: SUBROUTINE GRIB1\_FINALIZE(GRIB\_INDEX, SECT1, TIME\_UNIT, TIME\_1,
  TIME\_2, TIME\_RANGE) \\
- Finalizes the PDS section of a grib record by endcoding variable specific
  metadata into the output grib record. \\
\begin{itemize}
 \item[ Input:]
    GRIB\_INDEX (integer): Index into the GRIB pds values array corresponding
    to the variable being processed for output (See tables below for
    enumerated types of this value for the different output variables)  \\
    SECT1 (integer (36)): Parameters for Section 1 (PDS) of the GRIB record \\
    TIME\_UNIT (integer): Units in time for output contained in GRIB record.
    These are encoded as follows:
    \begin{description}
      \item[0] - Minute
      \item[1] - Hour
      \item[2] - Day
      \item[3] - Month
      \item[4] - Year
      \item[254] - Second
    \end{description}
    TIME\_1(integer): Time1 for GRIB Record Descriptor.  Used only when
    averaging or accumulating variables over a period of time (See TIME\_RANGE
    description). \\
    TIME\_2(integer): Time2 for GRIB Record Descriptor.  For general LIS usage
    this should always be set to zero. \\
    TIME\_RANGE: Time Range Indicator.  Describes relationship between TIME\_1
    and TIME\_2.  These are encoded as follows (for a more detailed description
    please refer to the WMO Grib 1 Manual - Table 5):
    \begin{description}
      \item[0] - Forecast product value for reference TIME\_1 (TIME\_2 ignored)
      \item[1] - Analysis product for reference TIME\_1 (TIME\_2 ignored)
      \item[2] - Product wth a valid time ranging between +TIME\_1 and +TIME\_2
      \item[3] - Average (reference time +TIME\_1 to reference time +TIME\_2)
      \item[4] - Accumulation (reference time +TIME\_1 to reference time
        +TIME\_2) product considered valid at TIME\_2
      \item[5] - Difference (reference time +TIME\_2 minus reference time
        +TIME\_1) product considered valid at TIME\_2
      \item[6] - Average (reference time -TIME\_1 to reference time -TIME\_2)
      \item[7] - Average (reference time -TIME\_1 to reference time +TIME\_2)
    \end{description}
  \item[Side Effects: ]
    Do not use negative values when defining the three time descriptor
    variables as this will cause an error in the low-level grib packing
    routines.  Instead use a different time range indicator.
\end{itemize}

\subsection{ SUBROUTINE DRV\_WRITEVAR\_GRIB }
- Call: DRV\_WRITEVAR\_GRIB(FTN, FTN\_STATS, N, FLAG, VAR, MVAR,
  FORM, TOPLEV, BOTLEV, KDIM)   \\
- Writes the grib record by gathering the variable from the indiviudual MPI
  tasks and calling the lower level grib output routines
\begin{itemize}
 \item[Input: ]
  FTN (integer): Unit number of grib output file. \\
  FTN\_STATS (integer): Unit number of grib output file. \\
  N (integer): Index of the of LIS domain or nest. \\
  FLAG (integer): Unit number of grib output file. \\
  VAR (real(lis\%nch(n)): Variable output data. \\
  MVAR (character(len=*)): Name of variable being written. \\
  FORM (integer): Format to be used in stats file (1-decimal,2-scientific) \\
  TOPLEV (real(KDIM)): Format to be used in stats file (1-decimal,2-scientific) \\
  BOTLEV (real(KDIM)): Format to be used in stats file (1-decimal,2-scientific) \\
  KDIM (integer): Grid dimension Format to be used in stats file (1-decimal,2-scientific) \\
  INFO (integer(5)): GRIB packing descriptor, describes the length of each
     section of grib header for low-level grib encoding. \\
  BITMAP (logical) : Determines if a bitmap is used in the grib encoding
     (ALWAYS leave defined as true!!) \\
  DATE (character(10)): Character string containing the date of the grib
     record ( format: YYYYMMDDHH )  \\
 \item[Side Effects:]
  NONE - Do not set Bitmap to false or grib packing will fail.
\end{itemize}

\subsection{ Output GRIB Variables }
 \setlongtables
 \begin{longtable}{|l|c|c|}
  \hline
  GRIB ID & Table Number & Description \\
  \hline
  \multicolumn{3}{|c|}{ALMA ENERGY BALANCE COMPONENTS} \\
  \hline
  GRIB\_SWNET      & 1 & Net Shortwave Radiation Flux \\ 
  GRIB\_LWNET      & 2 & Net Longwave Radiation Flux \\ 
  GRIB\_QLE        & 3 & Latent Heat Flux \\
  GRIB\_QH         & 4 & Sensible Heat Flux\\
  GRIB\_QG         & 5 & Ground Heat Flux \\
  GRIB\_QF         & 6 & Energy of Fusion \\ 
  GRIB\_QV         & 7 & Energy of Sublimation \\
  GRIB\_QTAU       & 8 & Momentum Flux \\
  GRIB\_QA         & 9 & Advective Energy\\
  GRIB\_DELSRFHEAT & 10 & Change in Surface Heat Storage \\
  GRIB\_DELCLDCNT  & 11 & Change in Snow Cold Content \\
  \hline
  \multicolumn{3}{|c|}{ALMA WATER BALANCE COMPONENTS} \\
  \hline
  GRIB\_SNOWF      & 12 & Snowfall Rate \\
  GRIB\_RAINF      & 13 & Rainfall Rate \\
  GRIB\_EVAP       & 14 & Total Evaporation \\
  GRIB\_QS         & 15 & Surface Runoff \\
  GRIB\_QREC       & 16 & Recharge \\ 
  GRIB\_QSB        & 17 & Subsurface Runoff \\
  GRIB\_QSM        & 18 & Snowmelt \\
  GRIB\_QFZ        & 19 & Refreezing of Water in the Snow \\
  GRIB\_QST        & 20 & Snow Throughfall \\
  GRIB\_DELSM      & 21 & Change in Soil Moisture \\
  GRIB\_DELSWE     & 22 & Change in Snow Water Equivalent \\
  GRIB\_DELSRFSTR  & 23 & Change in Surface Water Storage \\
  GRIB\_DELINTCPT  & 24 & Change in Interception Storage \\
  \hline
  \multicolumn{3}{|c|}{ALMA SURFACE STATE VARIABLES} \\
  \hline
  GRIB\_SNOWT      & 25 & Snow Surface Temperature \\
  GRIB\_VEGT       & 26 & Vegetation Canopy Temperature \\
  GRIB\_BARESOILT  & 27 & Temperature of Bare Soil \\
  GRIB\_AVGSURFT   & 28 & Average Surface Temperature \\
  GRIB\_RADT       & 29 & Surface Radiative Temperature \\
  GRIB\_ALBEDO     & 30 & Surface Albedo \\
  GRIB\_SWE        & 31 & Snow Water Equivalent (SWE) \\
  GRIB\_SWEVEG     & 32 & SWE intercepted by Vegetation \\ 
  GRIB\_SURFSTOR   & 33 & Surface Water Storage \\
  \hline  
  \multicolumn{3}{|c|}{ALMA SUBSURFACE STATE VARIABLES} \\
  \hline
  GRIB\_SOILMOIST  & 34 & Soil Moisture \\
  GRIB\_SOILTEMP   & 35 & Soil Temperature \\ 
  GRIB\_LSOILMOIST & 36 & Avg. layer fraction of Liquid Moisture\\
  GRIB\_FSOILMOIST & 37 & Avg. layer fraction of Frozen Moisture\\
  GRIB\_SOILWET    & 38 & Total Soil Wetness \\
  \hline
  \multicolumn{3}{|c|}{ALMA EVAPORATION COMPONENTS} \\
  \hline
  GRIB\_POTEVAP    & 39 & Potential Evaporation \\
  GRIB\_ECANOP     & 40 & Interception Evaporation \\
  GRIB\_TVEG       & 41 & Vegetation Transpiration \\
  GRIB\_ESOIL      & 42 & Bare Soil Evaporation \\
  GRIB\_EWATER     & 43 & Open Water Evaporation \\ 
  GRIB\_ROOTMOIST  & 44 & Root Zone Soil Moisture \\
  GRIB\_CANOPINT   & 45 & Total Canopy Water Storage \\
  GRIB\_ESNOW      & 46 & Snow Evaporation \\
  GRIB\_SUBSNOW    & 47 & Snow Sublimation \\
  GRIB\_SUBSURF    & 48 & Sublimation of Snow Free Area\\
  GRIB\_ACOND      & 49 & Aerodynamic Conductance \\
  \hline
  \multicolumn{3}{|c|}{FORCING VARIABLES} \\
  \hline
  GRIB\_WINDFORC   & 50 & Wind Speed \\
  GRIB\_RAINFFORC  & 51 & Rainfall Forcing \\
  GRIB\_SNOWFFORC  & 52 & Snowfall Forcing \\
  GRIB\_TAIRFORC   & 53 & Air Temperature \\
  GRIB\_QAIRFORC   & 54 & Specific Humidity \\
  GRIB\_PSURFFORC  & 55 & Surface Pressure \\
  GRIB\_SWDOWNFORC & 56 & Downwelling Shortwave Radiation Flux \\
  GRIB\_LWDOWNFORC & 57 & Downwelling Longwave Radiation Flux \\
  \hline
  \multicolumn{3}{|c|}{PARAMETER OUTPUT - EXPERIMENTAL} \\
  \hline
  GRIB\_LANDMASK   & 58 & Land Mask \\
  GRIB\_LANDCOVER  & 59 & Vegetation Type - Landcover\\
  GRIB\_SOILTYPE   & 60 & Soil Type \\
  GRIB\_SOILCOLOR  & 61 & Soil Color \\
  GRIB\_TOPOGRAPHY & 62 & Topography of Land Surface \\
  GRIB\_LAI        & 63 & Leaf Area Index \\
  GRIB\_SAI        & 64 & Stem Area Index \\
  GRIB\_SNFRALBEDO & 65 & Snow-free Albedo \\
  GRIB\_MXSNALBEDO & 66 & Maximum Snow Albedo \\
  GRIB\_GREENNESS  & 67 & Greenness Fraction \\
  GRIB\_TEMP\_BOT   & 68 & Bottom Temperature \\
  \hline
 \end{longtable}

\subsection{ Example of Fortran code that calls GRIB output routines }

\begin{verbatim}

! Setup of GRIB Metadata Section

! toplev is the depth of the top of each soil layer
! botlev is the depth of the bottom of each soil layer
toplev(1) = 0.0
botlev(1) = noah_struc(n)%lyrthk(1)

! determine bounding levels for each soil moisture layer
do i = 2, noah_struc(n)%nslay
   toplev(i) = toplev(i-1) + noah_struc(n)%lyrthk(i-1)
   botlev(i) = botlev(i-1) + noah_struc(n)%lyrthk(i)
enddo

! Convert to centimeters -- the depths for layers below the land
! surface (surface = 112) are expected in centimeters,
! per GRIB specifications.
toplev = toplev * 100.0
botlev = botlev * 100.0

! Set values for non layered fields (Fluxes, Sfc Fields, etc.)
toplev0 = 0
botlev0 = 0

! Set Date String to Pass to GRiB Module
hr1=lis%hr
da1=lis%da
mo1=lis%mo
yr1=lis%yr
write(unit=date,fmt='(i4.4,i2.2,i2.2,i2.2)') yr1,mo1,da1,hr1

! Setup common information to go into the PDS section
! of the GRIB file (Variable Independent Metadata)

call grib1_setup(gribobj(n)%sect1, gribobj(n)%info, &
     gribobj(n)%bitmap, date)

! Set time units of output for GRIB file
! Is output interval in days,hours,minutes,seconds?)
! Also, determine output time range in that unit (time_past)

if(noah_struc(n)%outInterval .GT. 0) then
   time_unit = 254     ! seconds
   time_past = (noah_struc(n)%outInterval / 1)
endif
if(noah_struc(n)%outInterval .GE. 60) then
   time_unit = 0      ! minutes
   time_past = (noah_struc(n)%outInterval / 60)
endif
if(noah_struc(n)%outInterval .GE. 3600) then
   time_unit = 1    ! hours
   time_past = (noah_struc(n)%outInterval / 3600)
endif
if(noah_struc(n)%outInterval .GE. 86400) then
   time_unit = 2   ! days
   time_past = (noah_struc(n)%outInterval / 86400)
endif

! End of GRIB metadata section

! Sample Outputs.....

! Output Net Shortwave Radiation

! Finalize PDS section of this variable, in this case SWNET is a time
! averaged variable => time range indicator equals 7
call grib1_finalize(GRIB\_SWNET,gribobj(n)%sect1,time_unit,time_past,0,7)

noah_struc(n)%noah%swnet = noah_struc(n)%noah%swnet/float(noah_struc(n)%count)

! Call low level grib routines (single level data, hence 1 as final arg)
call drv_writevar_grib(ftn,ftn_stats,n,metadata_output%swnet,    &
                       noah_struc(n)%noah%swnet,"Swnet(W/m2)",1, &
                       toplev0, botlev0, 1)


! Output Soil Moisture

! Store soil moisture in a 2D array (LSM points x Num. Layers)
temp_nslay = 0.0
do i=1,noah_struc(n)%nslay
   do t=1,lis%nch(n)
      temp_nslay(t,i) = noah_struc(n)%noah(t)%soilmoist(i)
   enddo
enddo

! Finalize PDS section of this variable, in this case SOILMOIST is an
! instantaneous variable => time range indicator equals 1 - analysis
call grib1_finalize(GRIB\_SOILMOIST,gribobj(n)%sect1,time_unit,0,0,1)

! Call low level grib routines (output all levels, number of layers
! is used as the last argument)
call drv_writevar_grib(ftn,ftn_stats,n,metadata_output%soilmoist,&
                          temp_nslay,"SoilMoist(kg/m2)",2,          &
                          toplev, botlev,  noah_struc(n)%nslay)

\end{verbatim}

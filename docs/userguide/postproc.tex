\section{Post-Processors}
\label{sec:PostProc}

In Section~\ref{sec:FrontWorkflows}, sample workflows are presented to 
initialize and run WRF in multiple configurations. In the present section 
we address the question of post-processing the WRF (and in some cases, LIS) 
output. All of the post-processors address the task of evaluation, either 
subjective or objective. Several tools are available to prepare visualizations
of model fields, while others allow for calculating verification metrics 
against observations or gridded analyses. 

It should be noted that other generic tools exist that can be used to
evaluate NU-WRF netCDF4 output. These include:

\begin{itemize}
  \item IDL (\url{http://www.harrisgeospatial.com/ProductsandSolutions/GeospatialProducts/IDL.aspx});
  \item IDV (\url{http://www.unidata.ucar.edu/software/idv/});
  \item MATLAB (\url{http://www.mathworks.com/products/matlab/index.html?s_tid=gn_loc_drop});
  \item NCL (\url{http://www.ncl.ucar.edu});
  \item Ncview (\url{http://meteora.ucsd.edu/~pierce/ncview\_home\_page.html});
  \item Python (\url{http://www.python.org}) with:
    \begin{itemize}
    \item Matplotlib (\url{http://matplotlib.org}) or;
    \item PyNGL and PyNIO (\url{https://www.pyngl.ucar.edu});
    \end{itemize}
  \item R (\url{http://www.r-project.org}) with:
    \begin{itemize}
    \item ncdf4 (\url{https://cran.r-project.org/web/packages/ncdf4/index.html}); or
    \item RNetCDF (\url{https://cran.r-project.org/web/packages/RNetCDF/index.html}); 
    \end{itemize}
  \item VAPOR (\url{https://www.vapor.ucar.edu}).
\end{itemize}

See also a list of netCDF compatible software maintained at 
\url{http://www.unidata.ucar.edu/software/netcdf/software.html}.

\subsection{G-SDSU}

The Goddard Satellite Data Simulator Unit [G-SDSU;~\cite{ref:MatsuiEtAl2014}] 
is a program developed by NASA for use with cloud-resolving model data. 
The program can simulate multiple microwave, radar, visible and 
infrared, lidar, and broadband satellite products from the input model 
fields. These simulations can be used for detailed verification against actual
satellite observations~\citep{ref:MatsuiEtAl2009}, for assimilation of 
satellite radiances, or for exploring future satellite missions. The software 
is compiled by typing \texttt{./build.sh gsdsu}. Instructions on running the 
program are available in \cite{ref:GsdsuUserGuide}.

\subsection{RIP4}

The community Read/Interpolate/Plot (RIP) Version 4 software package is 
capable of processing WRF netCDF files, deriving new variables (e.g., air 
temperature, relative humidity, CAPE), interpolating to isobaric, isentropic, 
or constant height levels as well has vertical cross-sections, and plotting
the fields in NCAR Graphics \texttt{gmeta} format. Advanced options also
exist, including calculating and plotting trajectories, interpolating
between coarse and fine grid resolutions, and writing data in a format
readable by the Vis5D visualization package (see 
\url{http://vis5d.sourceforge.net}).

The RIP software is compiled by typing \texttt{./build.sh rip}. The two most 
important executables in RIP4 are:

\begin{itemize}
\item \textbf{RIPDP\_WRFARW}. This program will read WRF netCDF files and
transform the data to an internal binary data format. The user will have
the option of processing either a basic set of variables or all the
variables in the files.
\item \textbf{RIP}. This program will process the output of RIPDP\_WRFARW
based on the user's settings in a provided input file.
\end{itemize}

Users are referred to \cite{ref:RipUserGuide} for detailed instructions on
using RIP. Sample namelist files are included in the NU-WRF package in
\texttt{scripts/rip/}.

\subsection{ARWPOST}

The ARWPOST program is a post-processor developed by NCAR for converting
WRF-ARW netCDF data into GrADS format. Analogous to RIP, ARWPOST supports
derivation of certain variables from the model output, and interpolation
of those fields to isobaric or constant height levels. GrADS
(see \url{http://www.iges.org/grads}) can then be used to visualize the data.
The program is compiled by typing \texttt{./build.sh arw}. Instructions for
running ARWPOST are given in Chapter 9 of \cite{ref:ArwUserGuide}.

\subsection{UPP}
\label{subsec:UPP}

The UPP program is the ``Unified Post-Processor'' developed by NOAA/NCEP
for processing all NCEP model data. As with RIP and ARWPOST,
UPP can read WRF netCDF output files, derive a number of meteorological fields
from the provided model data, and interpolate to user specified levels.
In the case of UPP, the data are output in GRIB format. The program is
compiled by typing \texttt{./build.sh upp}. Instructions for running UPP are 
given \cite{ref:UppUserGuide}.

The NU-WRF version of UPP includes several modifications provided by 
NASA SPoRT. These are experimental severe weather diagnostics: 
\begin{itemize}
\item \textbf{Instantaneous Lighting Threat 1}. Based on grid-resolved
graupel flux at -15C. Specified as ``LIGHTNING THREAT 1'' in 
\texttt{parm/wrf\_cntrl.parm} file.
\item \textbf{Instantaneous Lightning Threat 2}. Based on vertically
integrated ice. Specified as ``LIGHTNING THREAT 2'' in 
\texttt{parm/wrf\_cntrl.parm} file.
\item \textbf{Instantaneous Lightning Threat 3}. Based on Threat 1 and 2 
products. Specified as ``LIGHTNING THREAT 3'' in 
\texttt{parm/wrf\_cntrl.parm} file.
\item \textbf{Interval Maximum Lighting Threat 1}. Based on grid-resolved
graupel flux at -15C. Specified as ``MAX LTG THREAT 1'' in 
\texttt{parm/wrf\_cntrl.parm} file.
\item \textbf{Interval Maximum Lightning Threat 2}. Based on vertically
integrated ice. Specified as ``MAX LTG THREAT 2'' in 
\texttt{parm/wrf\_cntrl.parm} file.
\item \textbf{Interval Maximum Lightning Threat 3}. Based on Threat 1 and 2 
products. Specified as ``MAX LGT THREAT 3'' in  
\texttt{parm/wrf\_cntrl.parm} file.
\end{itemize}

In addition, UPP has been modified to read both graupel and hail mixing
ratios, which are provided by the 4ICE microphysics scheme.  Since the
NCEP GRIB2 table does not list ``hail'' as a variable, hail is added to
graupel for output and for use in visibility or radiance calculations.  Also,
UPP will read in radar reflectivity from the WRF output (variable
REFL\_10CM) if the 3ICE or 4ICE microphysics scheme was used, and this
field will be used for any requested reflectivity product (in contrast,
UPP calculates reflectivity for other microphysics schemes).

\subsection{MET}

The MET software is a community meteorological verification toolkit developed
by the DTC. This is a generic tool for comparing gridded model forecasts
and analyses against numerous observations -- METARs, Mesonets, rawinsondes,
MODIS satellite data, and Air Force cloud analysis data. MET expects the
model data to be in GRIB format, a requirement that forces the user to 
run UPP on the WRF output first (see Section \ref{subsec:UPP}). Observation
data formats include PREPBUFR and MADIS. With this input data, MET can be
used for a number of different meteorological verifications, including 
point-to-point verification, object-oriented verification, and wavelet
verification. Numerous statistical measures can be calculated with
confidence intervals, and plotting capabilities are available. 

The MET software is compiled by typing \texttt{./build.sh met}. Thorough 
instructions on running the software are provided in \cite{ref:MetUserGuide}.

\subsection{LVT}

LVT is a NASA developed land surface verification toolkit. It is designed
to compare LIS output files against numerous in-situ, remotely sensed,
and reanalysis products. Fields that can be evaluated include surface
fluxes, soil moisture, snow, and radiation. Multiple verification metrics
can be calculated, and advanced features include data masking, time series,
temporal averaging, and analysis of data assimilation impacts. The software
is compiled by typing \texttt{./build.sh lvt}.  Detailed instructions on 
running LVT can be found in \cite{ref:LvtUserGuide}.


\section{Porting NU-WRF}
\label{sec:Porting}

Currently NU-WRF is only fully supported for Intel compilers, for SGI MPT (on 
Discover and Pleiades), and for Intel MPI (on Discover).  The current development 
branch also works with GNU compilers and OpenMPI or MVAPICH2 on Discover. 
The underlying  software should, however, run on other systems as long as the 
appropriate tools (compilers, MPI implementation, make, Perl, csh, bash, etc.) are 
available.  Users who wish to port NU-WRF will need to take the following 
steps:

\begin{itemize}
\item Libraries:
  \begin{itemize}

  \item Compile the libraries listed in section~\ref{subsec:Libraries}.

  \item Determine the paths to the \texttt{yacc} binary and the \texttt{flex} 
    library. Make sure \texttt{yacc} and \texttt{flex} are in your 
    \texttt{PATH} environment variable.

  \item Copy \texttt{scripts/python/build/nu-wrf.cfg} to a new top level build configuration file 
    (here called \texttt{newconfig.cfg}, but any unique name can be used).

  \item Edit \texttt{newconfig.cfg} and update the library paths.

  \item Edit \texttt{scripts/other/set\_module\_env.bash}. Remove hardwired checks for 
   pleiades and discover that will about execution of the script. Next, set LIBDIR\_TAG as described
    in section~\ref{subsec:libdir} and change modules for system binaries and 
    libraries (compilers, MPI, etc). If the Environmental Modules package 
    (\url{http://modules.sourceforge.net/}) is not installed on your system, 
    comment out the module commands and explicitly edit the \texttt{PATH} and 
    \texttt{LD\_LIBRARY\_PATH} environment variables.  
  \end{itemize}

\item ARWPOST:

  \begin{itemize}

  \item Inspect and edit \texttt{ARWpost/arch/configure.defaults} to ensure a 
    block exists for the desired operating system, hardware, and 
    compilers. Note that ARWPOST is serial only (no MPI support).

  \item Run \texttt{ARWpost/configure} at the command line to identify the 
    integer value of the appropriate build selection.

  \item Edit \texttt{newconfig.cfg} to enter the configure option as
    environment variable ARWPOST\_CONFIGURE\_OPT.

  \item In the top NU-WRF directory, run 
    \texttt{./build.sh -\--config \mbox{newconfig.cfg} arwpost} to test the 
    build.

  \end{itemize}

\item GSDSU:
  \begin{itemize}

  \item Create a new makefile template in directory \texttt{GSDSU/SRC/} 
    specifying the appropriate compilers, compiler flags, and MPI library 
    if applicable.

  \item Edit \texttt{newconfig.cfg} to enter the new makefile template
    name as environment variable SDSU\_MAKEFILE.

  \item In the top NU-WRF directory, run 
    \texttt{./build.sh -\--config \mbox{newconfig.cfg} gsdsu} to test the 
    build.

  \end{itemize}

\item LDT:
  \begin{itemize}

  \item Edit \texttt{ldt/arch/Config.pl} to specify the compiler flags. 
    Compilers are specified using the LDT\_ARCH environment flag (e.g.,
    `linux\_ifc' indicates Intel compilers on Linux).

  \item Edit \texttt{newconfig.cfg} to specify the correct LDT\_ARCH value.
    If necessary, also edit the LDT\_FC and LDT\_CC variables to specify the
    correct MPI compiler wrappers.

  \item In the top NU-WRF directory, run 
    \texttt{./build.sh -\--config \mbox{newconfig.cfg} ldt} to test the 
    build.

  \end{itemize}

\item LVT:
  \begin{itemize}

  \item Edit \texttt{lvt/arch/Config.pl} to specify the compiler flags. 
    Compilers are specified using the LVT\_ARCH environment flag (e.g.,
    'linux\_ifc' indicates Intel compilers on Linux).

  \item Edit \texttt{newconfig.cfg} to specify the correct LVT\_ARCH value.
    If necessary, also edit the LVT\_FC and LVT\_CC variables to specify the
    correct MPI compiler wrappers.

  \item In the top NU-WRF directory, run 
    \texttt{./build.sh -\--config \mbox{newconfig.cfg} ldt} to test the build.

  \end{itemize}

\item MET:
  \begin{itemize}

  \item Edit \texttt{newconfig.cfg} to set the CXX, CC, and F77 environment
    variables, which defines the names of the C++, C, and Fortran compilers,
    respectively.

  \item In the top NU-WRF directory, run 
    \texttt{./build.sh -\--config \mbox{newconfig.cfg} met} to test the 
    build.

  \end{itemize}

\item RIP4:
  \begin{itemize}

  \item Inspect and edit \texttt{RIP4/arch/configure.defaults} to ensure a 
    block exists for the desired operating system, hardware, and 
    compilers. Note that RIP4 is serial only (no MPI support).

  \item Run \texttt{RIP4/configure} at the command line to identify the 
    integer value of the appropriate build selection.

  \item Edit \texttt{newconfig.cfg} to enter the configure option as
    environmental variable RIP\_CONFIGURE\_OPT.

  \item In the top NU-WRF directory, run 
    \texttt{./build.sh -\--config \mbox{newconfig.cfg} rip} to test the 
    build.

  \end{itemize}

\item UPP:
  \begin{itemize}

  \item \emph{NOTE: Make sure WRF is ported first.}

  \item Inspect and edit \texttt{UPP/arch/configure.defaults} to ensure a 
    block exists for the desired operating system, hardware, compilers, 
    and MPI implementation. 

  \item Run \texttt{UPP/configure} at the command line to identify the integer
    value of the appropriate build selections.

  \item Edit \texttt{newconfig.cfg} to enter the configure option as
    environment variable UPP\_CONFIGURE\_MPI\_OPT. 

  \item In the top NU-WRF directory, run 
    \texttt{./build.sh -\--config \mbox{newconfig.cfg} upp} to test the build.

  \end{itemize}

\item WPS:
  \begin{itemize}

  \item \emph{NOTE: Make sure WRF is ported first.}

  \item Inspect and edit \texttt{WPS/arch/configure.defaults} to ensure a 
    block exists for the desired operating system, hardware, compilers, 
    and MPI implementation.

  \item Run \texttt{WPS/configure} at the command line to identify the integer
    value of the appropriate build selection.

  \item Edit \texttt{newconfig.cfg} to enter the configure option as
    environmental variable WPS\_CONFIGURE\_MPI\_OPT.  If necessary, also edit
    the WPS\_DEBUG\_CFLAGS, WPS\_DEBUG\_FFLAGS, and WPS\_DEBUG\_F77FLAGS
    variables to specify appropriate debugging flags for your compilers.

  \item In the top NU-WRF directory, run 
    \texttt{./build.sh -\--config \mbox{newconfig.cfg} wps} to test the 
    build.

  \end{itemize}

\item WRF and LIS:
  \begin{itemize}

  \item Inspect and edit \texttt{WRF/arch/configure\_new.defaults} to ensure
    a block exists for the desired operating system, hardware, compilers, 
    and MPI implementation.

  \item Run \texttt{WRF/configure} to identify the integer value of the 
    appropriate build selection.

  \item Create a new configure.lis makefile template in 
    \texttt{LISF/lis/arch/} with appropriate compiler selection. NOTE:
    This approach is used instead of running the LIS \texttt{configure} 
    script because LDFLAGS must be absent from the configure.lis
    file if the LIS code is compiled for coupling; also, it is easier
    to pass consistent debugging compiler flags to WRF and LIS by 
    having the NU-WRF build system do it on the fly.

  \item Edit the \texttt{newconfig.cfg} to enter the configure option as
    environmental variable WRF\_CONFIGURE\_MPI\_OPT.
    Also list the makefile template with environmental variable 
    WRF\_CONFIGURE\_LIS\_MPI.  Also, set environmental 
    variable LIS\_ARCH to the value appropriate for your operating system
    and compiler [see~\cite{ref:LisUserGuide} for options].  If necessary,
    also edit the WRF\_DEBUG\_CFLAGS\_LOCAL and WRF\_DEBUG\_FCOPTIM
    variables to specify appropriate debugging flags for your compilers.

  \item In the top NU-WRF directory, run 
    \texttt{./build.sh -\--config \mbox{newconfig.cfg} wrf} to test the 
    build. Also test with the chem and kpp targets.

  \end{itemize}

\end{itemize}

Porting the programs under \texttt{utils/} (GEOS2WRF, CASA2WRF, SST2WRF, etc.) only
requires that the file path of the external NU-WRF libraries, LIBRDIR\_TAG, described at the end of 
section~\ref{subsec:libdir} be correctly specified. 

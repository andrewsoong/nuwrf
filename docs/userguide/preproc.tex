\section{Preprocessors}
\label{sec:Preprocessors}

\subsection{WPS}

The WRF Preprocessing System is part of the official WRF modeling system, 
and includes a terrestrial data processor (GEOGRID), a GRIB data reader
(UNGRIB), and a preprocessor for horizontally interpolating meteorological 
data to the WRF grid (METGRID). These programs are described in Chapter 3 
of~\cite{ref:ArwUserGuide}. Only a brief summary of NU-WRF specific changes 
will be provided here.

The NU-WRF build system compiles WPS with GRIB 2 support (linking against the
JasPer, PNG, and zlib libraries) and with full netCDF4 support (with HDF5
compression). The former is an optional configuration in the community version,
while the latter is not possible without editing lower level Perl scripts
to add recognition of HDF5 and related compression libraries. We have made
both features mandatory in NU-WRF for simplicity, generality (GRIB2 has been
the standard WMO gridded data format for a number of years, and adding support
does not remove the ability to process older GRIB1 files), and file system
efficiency (NetCDF4/HDF5 compression can reduce file sizes by 50\%). 

In NU-WRF, a new \textit{GEOGRID.TBL.ARW\_CHEM\_NUWRF} look-up table has been
added to list new erodible soil related data sets:
\begin{itemize}
\item \textit{EROD\_MDB\_SAND} New seasonal erodible sand based on MODIS-Deep
Blue data.
\item \textit{EROD\_MDB\_SILT} New seasonal erodible silt based on MODIS-Deep
Blue data.
\item \textit{EROD\_MDB\_CLAY} New seasonal erodible clay based on MODIS-Deep
Blue data.
\item \textit{BARENESS\_DYN\_CLIMO} New monthly bareness derived from MODIS 
monthly greenness, MODIS and USGS landuse, and soil type.
\item \textit{TOPODEP} New topographic depression dataset derived from standard
WRF GTOPO 30 arc second terrain.
\end{itemize}

Also, the community \textit{EROD} option is renamed to \textit{EROD\_STATIC} to
better discriminate from the NU-WRF datasets. Finally, the entry for 
\textit{GREENFRAC} has been modified to allow use of daily NASA SPoRT
GVF products in GEOGRID binary format. 

As with the other variants, users who wish to use this file should copy to 
\textit{GEOGRID.TBL} before running GEOGRID.

\subsection{GEOS2WRF}
\subsection{SST2WRF}
\subsection{LISCONFIG}
\subsection{LDT}

The Land Data Toolkit is a new preprocessor developed for use with LIS 7. This
program serves three purposes:

\begin{itemize}
\item Interpolate all required LIS terrestrial data (land use, soil type, 
terrain, etc) to the LIS grids;
\item Preprocess and quality control observations for use by the
LIS data assimilation package; and
\item Combine LIS dynamic output fields with the input terrestrial fields
for use by the WRF REAL preprocessor.
\end{itemize}

The tool is described in~\cite{ref:LdtUserGuide}. Note that in the present
version, LDT can only be run in serial mode, and the NU-WRF build system 
will compile it accordingly.

Several useful tips when running LDT as part of NU-WRF:
\begin{itemize}
\item Run GEOGRID first, and then use the LISCONFIG utility to customize the
ldt.config file to use the WRF grid. 
\item Select Noah.3.3 as the land surface model. Note that work is planned to 
allow NU-WRF to run coupled with LIS with Noah-MP, CLM, and a newer version of
Noah.
\item Set ``Include water points:'' to ``.false.'' to ensure only land points
are considered by LIS. This will save computational time.
\item Set ``Bottom temperature topographic downscaling:'' to ``lapse-rate''.
This will apply the standard lapse rate to the climatological skin temperature
map (assumed to be valid at mean sea level) and cool the temperatures in
higher terrain. This should result in similar bottom temperatures as those
normally processed by REAL for WRF.
\item When running to provide input to LIS, set the 
``Processed LSM parameter filename:'' to use ``lis\_input'' as part of the 
output file name. Otherwise, when processing the LIS output for use by REAL,
set the ``Processed NUWRF file for input to real:'' to use ``lis\_output'' as
part of the output file name. 
\end{itemize}


\subsection{REAL}

Described in Chapter 4 of~\cite{ref:ArwUserGuide}.

\subsection{PREP\_CHEM\_SOURCES}

Described by~\cite{ref:FreitasEtAl2011}.

\subsection{PLOT\_CHEM}
\subsection{CONVERT\_EMISS}

Described in Chapters 2 and 3 of~\cite{ref:WrfChemUserGuide}.

\subsection{GOCART2WRF}

\subsection{CASA2WRF}



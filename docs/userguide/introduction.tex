
\section{Introduction}
\label{sec:Introduction}

\subsection{This Document}
\label{subsec:ThisDocument}
This is the NASA Unified-Weather Research and Forecasting (NU-WRF) Version 10 
User's Guide. This document provides an overview of NU-WRF; describes how to 
download, compile, and run the NU-WRF software; and provides guidance on 
porting the software to new platforms.

This document consists of six sections and three appendices:

\begin{itemize}
\item Section~\ref{sec:Introduction} is the present introductory section.

\item Section~\ref{sec:System} provides general information about the 
NU-WRF project and its components.

\item Section~\ref{sec:Obtaining} provides information on obtaining a
software usage agreement and the NU-WRF source code.

\item Section~\ref{sec:Building} describes how to compile the NU-WRF software;

\item Section~\ref{sec:FrontWorkflows} describes several front-end workflows 
that can be employed with the NU-WRF modeling system, ranging from basic 
weather simulation to advanced aerosol-microphysics-radiation coupling to
alternate initialization methods.  This includes information on new 
pre-processors and NASA changes to the community WRF model.

\item Section~\ref{sec:PostProc} describes several post-processors for 
visualization and/or verification.

\item Finally, Appendix~\ref{sec:FAQ} answers Frequently Asked Questions about
NU-WRF, Appendix~\ref{sec:Porting} provides guidance on porting NU-WRF
to new platforms and Appendix~\ref{sec:revhist} lists the NU-WRF revision history.

\end{itemize}

\subsection{Acknowledgments}
The development of NU-WRF has been funded by the NASA Modeling, Analysis, and 
Prediction Program. The Goddard microphysics, radiation, aerosol coupling 
modules, and G-SDSU are developed and maintained by the Mesoscale 
Atmospheric Processes Laboratory at NASA Goddard Space Flight Center (GSFC). 
The GOCART and NASA dust aerosol emission modules are developed and maintained
by the GSFC Atmospheric Chemistry and Dynamics Laboratory. The LIS, LDT, and 
LVT components are developed and maintained by the GSFC Hydrological Sciences 
Laboratory. The CASA2WRF, GEOS2WRF, GOCART2WRF, LISWRFDOMAIN, LIS4SCM, 
NEVS, NDVIBARENESS4WRF, and SST2WRF components are maintained by the GSFC 
Computational and Information Sciences and Technology Office. SST2WRF includes
binary reader source code developed by Remote Sensing Systems.

Past and present contributors affiliated with the NU-WRF effort include:
Kristi Arsenault, Clay Blankenship, Scott Braun, Rob Burns, Jon Case, Mian 
Chin, Tom Clune, David Considine, Carlos Cruz, John David, Craig Ferguson, Jim Geiger, 
Mei Han, Ken Harrison, Arthur Hou, Takamichi Iguchi, Jossy Jacob, Randy Kawa, 
Eric Kemp, Dongchul Kim, Kyu-Myong Kim, Jules Kouatchou, Anil Kumar, Sujay 
Kumar, William Lau, Tricia Lawston, David Liu, Yuqiong Liu, Toshi Matsui, 
Hamid Oloso, Christa Peters-Lidard, Chris Redder, Scott Rheingrover, Joe 
Santanello, Roger Shi, David Starr, Rahman Syed, Qian Tan, Wei-Kuo Tao, 
Zhining Tao, Eduardo Valente, Bruce Van Aartsen, Gary Wojcik, Di Wu, 
Benjamin Zaitchik, Brad Zavodsky, Sara Zhang, Shujia Zhou, and Milija Zupanski.

The upstream community WRF, WPS, and ARWPOST components are developed and 
supported by the National Center for Atmospheric Research (NCAR), which is 
operated by the University Corporation for Atmospheric Research and sponsored
by the National Science Foundation (NSF). The Kinetic Pre-Processor included 
with WRF-Chem was primarily developed by Dr.~Adrian Sandu of the Virginia 
Polytechnic Institute and State University. The community RIP4 is maintained 
by NCAR and was developed primarily by Dr.~Mark Stoelinga, formerly of the 
University of Washington. The UPP is developed by the National Centers 
for Environmental Prediction; the community version is maintained by the 
Developmental Testbed Center (DTC), which is sponsored by the National Oceanic
and Atmospheric Administration, the United States Air Force, and the NSF. DTC 
also develops and maintains the community MET package.  The community 
PREP\_CHEM\_SOURCES is primarily developed by the Centro de Previs\~{a}o de 
Tempo e Estudos Clim\'{a}ticos, part of the Instituto Nacional de Pesquisas 
Espaciais, Brazil.

\subsection{Transition to GitHub}

NU-WRF version 10 marks the transition of NU-WRF into GitHub\footnote{Specifically, NASA GitHub, which requires a NASA account. Non-NASA employees will be able to access the software from a non-NASA cloud-based server provided they submit a \textbf{Software Usage Agreement}.}.  Version 10 is a 
major  revision, not only because NU-WRF is now hosted on GitHub, but because it has been significantly streamlined.  Specifically, ARWPOST, MET, RIP4 and UPP are not distributed with the 
source code but instead are hosted on the DISCOVER system and used by 
NU-WRF as needed. Also, The LIS software, including LDT and LVT, have been partially
decoupled from WRF and are synchronized with their parent repositories on GitHub. 
The plan is to do the same for all software that is GitHub-based. Finally, in keeping with
tradition, NU-WRF version 10 will be henceforth known as the "Dalton" release in honor of John Dalton.  
Dalton was an English chemist, physicist, and meteorologist. He is best known for introducing the atomic theory into chemistry. In meteorology he also developed a theory of atmospheric circulation and discovered the law of partial pressures - known as Dalton's Law.


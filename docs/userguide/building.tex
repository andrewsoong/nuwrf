\section{Building Software}
\label{sec:Building}

\subsection{Compilers}

The NU-WRF source code requires Fortran 90/2003, C, and C++ compilers. 
The current release officially supports Intel compilers.
(\texttt{ifort}, \texttt{icc}, and \texttt{icpc}) on Discover (version 18.0.3.222 ) and Pleiades (version 15.0.3.187). GCC (version 6.3 on Discover, 5.3 on Pleiades) is also required for library compatibility for \texttt{icpc}. The \emph{master} branch also works with GNU compilers and OpenMPI on Discover. 

\subsection{External Libraries and Tools}
\label{subsec:Libraries}

A large number of third party libraries must be installed before building 
NU-WRF. Except as noted, the libraries must be compiled using the same 
compilers as NU-WRF.  Included with  the NU-WRF source code are scripts (see scripts/other/baselibs.*) that facilitate the installation of all software packages on Discover and Pleiades. The scripts can be adapted to other systems with modifications. The list of libraries is as follows:

\begin{itemize}
\item An MPI library. (By default, NU-WRF uses Intel MPI on Discover and
the latest SGI MPT on Pleiades. )
\item BUFRLIB 11.0.
\item CAIRO 1.14.6.
\item ESMF 6.3.0rp1 compiled with MPI support.
\item FREETYPE 2.6.5.
\item FLEX (can use precompiled system binary on Linux).
\item G2CLIB 1.6.0.
\item GHOSTSCRIPT 8.11.
\item GRIB\_API 1.17.0.
\item GSL 2.2.1.
\item HDF4 4.2.12.
\item HDF5 1.8.17.
\item HDF-EOS 2.19v1.00.
\item JASPER 1.900.1.
\item JPEG 6b.
\item LIBPNG 1.2.56. (The newer 1.5.* versions \emph{cannot} be used due to 
  incompatibilities with WPS.)
\item NCAR Graphics 6.0.0. (Newer versions of NCAR Graphics require compilation
and linking to the CAIRO library, which is currently only supported by MET.)
\item NETCDF 4.4.1 (C library), NETCDF-Fortran 4.4.4, and NETCDF-CXX 4.2 
  (C++ library), built with HDF5 compression.
\item NETCDF 4.4.1.1 (C library) and NETCDF-CXX 4.3.0.1
  (C++ library) used for MET.
\item PIXMAN 0.34.0.
\item YACC (can use preinstalled version on Linux).
\item ZLIB 1.2.8.
\end{itemize}

In addition to the above libraries, NU-WRF requires \texttt{perl}, 
\texttt{python}, \texttt{bash}, \texttt{tcsh}, \texttt{gmake}, \texttt{sed}, 
\texttt{awk}, \texttt{m4}, and the UNIX \texttt{uname} command to be available
on the computer. 

\subsection{Build System}
\label{subsec:BuildSystem}
Each component of the NU-WRF modeling system has a unique compilation 
mechanism, ranging from simple Makefiles to sophisticated Perl and shell 
scripts. To make it easier for the user to create desired executables and to 
more easily resolve dependencies between components, NU-WRF includes a 
build-layer that glues together the entire framework. The build layer consists of a
Python3-based script system (that also works with Python2) configured by a text file 
with a particular structure readable by Python and used to define the build environment, 
library paths and  the build options. 

The build system is shipped with the code and is found in the scripts/python/build subfolder. 
It includes a configuration file \texttt{nu-wrf.cfg} that contains settings for Discover and Pleiades but can be easily configurable for other systems. The settings in the configuration file allows users to specify 
\texttt{configure} options for the ARWPOST, RIP4, UPP, WPS, and WRF components, as well as template \texttt{Makefile} names for the  GSDSU, LVT, and MET components. Not all components 
need a configuration (e.g. utils) but those that do need it are specified in separate sections. 
The provided configuration file should aid in porting NU-WRF to new 
compilers, MPI implementations, and/or operating systems, a process discussed 
more fully in Appendix~\ref{sec:Porting}. Finally, note that there is an additional file - \texttt{nu-wrf\_python2.cfg} - used in systems running Python2.

Invoking the build system is done in the top level NU-WRF directory by executing the \texttt{build.sh} script.
The \texttt{build.sh} script represents a thin layer over the python build system that generally acts like the \texttt{build.sh} script of previous NU-WRF versions - with some differences. \texttt{build.sh} requires at least one argument plus a handful of optional arguments. Running \texttt{build.sh} without arguments will print a usage page with the following information:

\begin{itemize}

 \item \textit{Target}. A required argument that defines the NU-WRF component to build.

 \item \textit{Installation}. The \texttt{-p, --prefix} flag followed by a path 
    will install all built executables in the specified path. If the flag is skipped,
    \texttt{build.sh} will install the executables in the model directory.

 \item \textit{Configuration}. The \texttt{-c, --config} flag followed by the name
  of a configuration file specifying critical environment variables (e.g., the
  path to the netCDF library). Users may develop their own configuration 
  file to customize settings (see Appendix~\ref{sec:Porting}). If the 
  config flag is skipped, \texttt{build.sh} will default to 
   \texttt{nu-wrf.cfg} which works on Discover and Pleiades systems.
   The software will exit if it is on an unrecognized computer. 

\item \textit{Options}. The \texttt{-o, --options} flag to specify the target options. 
Valid options are \texttt{cleanfirst}, 
  \texttt{debug}, \texttt{rebuild}, and/or \texttt{nest=n} where \texttt{n} is an integer 
  ranging from 1 to 3. Multiple options must be comma-separated,
   e.g. cleanfirst,debug,nest=2. Note that only one nest option can be specified at a time.
   
  \begin{itemize} 
  
    \item The \texttt{cleanfirst} option will cause the build system to
      ``clean'' a target (delete object files and static libraries) before 
      starting compilation. 

    \item The \texttt{debug} option forces some build subsystems 
    (\texttt{WRF, WPS, ldt, lvt, utils}) to use alternative compilation flags set 
      in the configuration file (e.g., for disabling optimization and turning 
      on run-time array bounds checking). This option is currently ignored by 
      other (\texttt{ARWpost, GSDSU, MET, RIP4, UPP})  NU-WRF components.

    \item The \texttt{rebuild} option is used to re-build the WRF executable - when only 
    WRF code is modified - while skipping LIS re-compilation. It can thus speed up the 
    regeneration of the WRF executable. It has no effect on other components.
               
    \item The \texttt{nest=n} option specifies compiling WRF with basic
      nesting (\texttt{n=1}), preset-moves nesting (\texttt{n=2}), or 
      vortex-tracking nesting (\texttt{n=3}). Basic nesting is assumed by 
      default. Note that WRF cannot be run coupled to LIS if preset-moves
      or vortex-tracking nesting is used. Similarly, WRF-Chem only runs
      with basic nesting.
            
  \end{itemize}

  \item The \texttt{-e} flag generates a file, nu.wrf.envs, that specifies
   all the NU-WRF configuration variables in a bash-script format.
   This can be useful for debugging or porting to other systems.
 
\item \texttt{Targets}. At least one target is required and multiple targets must be 
  comma-separated. The recognized targets are (note that all* and ideal* targets are 
  mutually exclusive):

  \begin{itemize}

  \item The \texttt{all} target compiles all executables without WRF-Chem.

  \item The \texttt{allchem} target compiles all executables with WRF-Chem but
    without the Kinetic Pre-Processor.

  \item The \texttt{allclean} target deletes all executables, object files,
    and static libraries.

  \item The \texttt{allkpp} target compiles all executables with WRF-Chem and
    including the Kinetic Pre-Processor.

  \item The \texttt{arwpost} target compiles executables in the 
    \texttt{ARWpost/} directory.

  \item The \texttt{casa2wrf} target compiles executables in the
    \texttt{utils/casa2wrf/} directory.

  \item The \texttt{chem} target compiles executables in the 
    \texttt{WRF/} directory with WRF-Chem support but 
    without the Kinetic Pre-Processor. The compiled executables include 
    CONVERT\_EMISS.

  \item The \texttt{doc} target builds user guide and tutorial PDF documentation.

  \item The \texttt{gocart2wrf} target compiles executables in the
    \texttt{utils/gocart2wrf/} directory. 

  \item The \texttt{gsdsu} target compiles executables in the \texttt{GSDSU/}
    directory.

  \item The \texttt{ideal\_*} targets compile ideal preprocessors in the \texttt{WRF/}
    directory.   These include:
  \begin{itemize}  
     \item \texttt{ideal\_scm\_lis\_xy}: Produces inputs for running WRF-LIS in Single Column Mode.
     \item \texttt{ideal\_b\_wave} Baroclinically unstable jet u(y,z) on an f-plane.
     \item \texttt{ideal\_convrad}: Idealized 3d convective-radiative equilibrium test with constant SST
   and full physics at cloud-resolving 1 km grid size.
     \item \texttt{ideal\_heldsuarez}: Held-Suarez coarse-resolution global forecast model.
     \item \texttt{ideal\_les}: An idealized large-eddy simulation (LES).
     \item \texttt{ideal\_quarter\_ss}: 3D quarter-circle shear supercell simulation.
     \item \texttt{ideal\_scm\_xy}: Single column model.
     \item \texttt{ideal\_tropical\_cyclone}: Idealized tropical cyclone on an f-plane with constant SST in a
   specified environment.
  \end{itemize}

  \item The \texttt{kpp} target compiles executables in the \texttt{WRF/}
    directory with WRF-Chem and Kinetic Pre-Processor
    support. The compiled executables include CONVERT\_EMISS.
  
  \item The \texttt{ldt} target compiles executables in the \texttt{LISF/ldt/} 
    directory.

  \item The \texttt{lis} target compiles LIS in uncoupled mode in the 
    \texttt{LISF/lis/make/} directory.

  \item The \texttt{lisWrfDomain} target compiles executables in the 
    \texttt{utils/lisWrfDomain/} directory.
    
  \item The \texttt{lis4scm} target compiles executables in the 
  \texttt{utils/lis4scm} directory. 

  \item The \texttt{lvt} target compiles executables in the \texttt{LISF/lvt/} 
    directory.

  \item The \texttt{geos2wrf} target compiles executables in the 
    \texttt{utils/geos2wrf/} directory (both GEOS2WPS and MERRA2WRF).

  \item The \texttt{met} target compiles executables in the \texttt{MET/} 
    directory.

  \item The \texttt{ndviBareness4Wrf} target compiles executables in the\\
    \texttt{utils/ndviBareness4Wrf/} directory.

  \item The \texttt{plot\_chem} target compiles executables in the
    \texttt{utils/plot\_chem/} directory.

  \item The \texttt{prep\_chem\_sources} target compiles executables in the\\
    \texttt{utils/prep\_chem\_sources/} directory.

  \item The \texttt{rip} target compiles executables in the \texttt{RIP4/}
    directory.

  \item The \texttt{sst2wrf} target compiles executables in the
    \texttt{utils/sst2wrf/} directory.

  \item The \texttt{upp} target compiles executables in the \texttt{UPP/}
    directory.

  \item The \texttt{utils} target compiles all the executables in the \texttt{utils/}
    directory.

  \item The \texttt{wps} target compiles executables in the \texttt{WPS/}
    directory.

  \item The \texttt{wrf} target compiles executables in the \texttt{WRF/}
    directory -- except for LIS and CONVERT\_EMISS -- without WRF-Chem support.

  \end{itemize}

\end{itemize}

One complication addressed by the build system is that the WPS and
UPP components are dependent on libraries and object files in the
\texttt{WRF/} directory. To account for this, the \texttt{wrf} target
will be automatically built  \textit{if necessary} for WPS and/or UPP, 
even if the \texttt{wrf} target is not listed on the command line.

A second complication is that the coupling of LIS to WRF requires linking 
the \texttt{WRF/} code to the ESMF and ZLIB libraries. As a result, the 
\texttt{configure.wrf} file [see Chapter 2 of \cite{ref:ArwUserGuide}] is 
modified to link against these libraries. A similar modification occurs for 
UPP. (No modification is needed for WPS as long as WPS is compiled with GRIB2 
support.)

Finally, if there are dependent components (WRF, WPS, UPP, LIS) that have been 
built with different  compilation flags, compiler versions or with/without chemistry, that will 
result in cleaning the associated directories.

The most straight-forward way to compile the full NU-WRF system on Discover or
Pleiades is to type \texttt{./build.sh all} in the top level directory. If
chemistry is required, the command is \texttt{./build.sh~allchem} 
(\texttt{./build.sh~allkpp} if KPP-enabled chemistry is needed). To fully
clean the entire system, run \texttt{./build.sh~allclean}. 

The user can selectively build components by listing specific targets. For
example, to build the WRF model without chemistry along with WPS and UPP,
type \texttt{./build.sh wrf,wps,upp}.

Finally, on Discover or a system with a proper \emph{pdflatex} installation one can run 
\texttt{./build.sh~doc} to generate the tutorial and user's guide documentation. The generated pdf files will be under docs/tutorial and docs/userguide respectively.

\subsection{LIBDIR\_TAG}
\label{subsec:libdir}
We discussed earlier (section \ref{subsec:Libraries}) that a large number of external 
libraries, or \emph{baselibs},  must be installed before NU-WRF can be built. Also, the 
\emph{baselibs} must be compiled using the same  compilers as NU-WRF.  The build 
systems parses the configuration file, nu-wrf.cfg, and, if on Discover or Pleiades systems, 
it uses pre-installed \emph{baselibs} determined by the value of the LIBDIR\_TAG environment 
variable set in scripts/other/set\_module\_env.bash. However, when porting the code to a 
different system and/or using a newer compiler the environment variable must be set by the user 
in the configuration file. 

For more information see Appendix~\ref{sec:Porting}. 




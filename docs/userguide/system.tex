\section{NU-WRF System}
\label{sec:System}

NU-WRF has been developed at Goddard Space Flight Center (GSFC) as an 
observation-driven integrated modeling system representing aerosol, cloud, 
precipitation, and land processes at satellite-resolved scales
\citep{ref:PetersLidardEtAl2014}. NU-WRF is intended as a superset of the 
standard NCAR Advanced Research WRF [WRF-ARW; \cite{ref:SkamarockEtAl2008}] 
and incorporates:

\begin{itemize} 
\item The GSFC Land 
Information System [LIS; see \cite{ref:KumarEtAl2006} and 
\cite{ref:PetersLidardEtAl2007}], coupled to WRF and also available as a 
stand-alone executable;
\item The WRF-Chem enabled version of the Goddard 
Chemistry Aerosols Radiation Transport model [GOCART; \cite{ref:ChinEtAl2002}];
\item GSFC radiation and microphysics schemes including revised couplings to 
the aerosols [\cite{ref:ShiEtAl2014}; \cite{ref:LangEtAl2014}]; and 
\item The Goddard Satellite Data Simulator 
Unit [G-SDSU; see \cite{ref:MatsuiEtAl2014}].
\end{itemize}
 
In addition, multiple pre- and post-processors from the community and from 
GSFC have been collected with WRF and LIS. Taken together, the NU-WRF modeling
system provides a framework for simulating aerosol and land use effects on 
such phenomena as floods, tropical cyclones, mesoscale convective systems, 
droughts, and monsoons~\citep{ref:PetersLidardEtAl2014}. Support also exists 
for treating CO$_2$ as a tracer, with plans to further refine into source 
components (anthropogenic versus biogenic). Finally, the software has been 
modified to use netCDF4 (\url{http://www.unidata.ucar.edu/software/netcdf/}) 
with HDF5 compression (\url{https://www.hdfgroup.org/HDF5/}), reducing netCDF 
file sizes by up to 50\%.

Recently NU-WRF was incorporated in a separate, Maximum Likelihood Ensemble 
Filter-based atmospheric data assimilation system~\citep{ref:GEDASUserGuide}, 
with the capability of assimilating cloud and precipitation affected radiances
\citep{ref:ZhangEtAl2015}. In addition, some secondary, rarely used elements 
of the community WRF modeling system that are not yet included with NU-WRF 
will be added in the future.

\subsection{Components}
\label{subsec:Components}WRF

The NU-WRF package contains the following components and utilities:

\begin{itemize}
\item The \texttt{WRF/} component contains a modified copy of the core WRF 
  version 3.9.1 modeling system [see Chapter 5 of \cite{ref:ArwUserGuide}], 
  the WRF-Fire wildfire library [see Appendix A of \cite{ref:ArwUserGuide}], 
  the WRF-Chem atmospheric chemistry library [see \cite{ref:WrfChemUserGuide}
  and \cite{ref:WrfChemEmissionsGuide}], and several preprocessors (REAL, 
  CONVERT\_EMISS, TC, and NDOWN). These codes have been modified to add
  the following physics packages:
  \begin{itemize}
  \item the 2017 Goddard radiation package
  \item the 2014 Goddard radiation package~\citep{ref:GRADUserGuide}
  \item the new Goddard 4ICE microphysics scheme~\citep{ref:LangEtAl2014}
  \item the new Goddard 3ICE  microphysics scheme~\citep{ref:ShiEtAl2014}
  \item couplings between these schemes and GOCART
  \item new dust emission options
  \item a new CO$_2$ tracer parameterization,
  \end{itemize}
  and several new diagnostic products. Recently, there have been contributions
  from other groups that add more physics options:
  \begin{itemize}
  \item DOE CRM: utilizes WRF as a CRM/LES using doubly periodic lateral boundaries
  \item NTU (National Taiwan University) microphysics: A multi-moment four-ice (pristine ice, aggregate, graupel, and hail) bulk microphysical scheme
  \item WRF electrification scheme: This scheme requires environment variable WRF\_ELEC=1 which can be set in the build configuration files.
  \end{itemize}
  The legacy community WRF versions of  the Goddard microphysics and radiation 
  schemes are separate options from the
  latest versions developed by GSFC. \texttt{WRF/} also includes LIS \footnote{Based on the 5/17/19 snapshot of the LISF repository.}
  (in \texttt{LISF/lis/}) with code modifications to both LIS and WRF-ARW to 
  facilitate on-line coupling between the atmosphere and Noah land surface
  models, as well as land data assimilation~\citep{ref:LisUserGuide}.   

\item The \texttt{WPS/} component contains a modified copy of the WRF 
  Preprocessing System version 3.9.1 [see Chapter 3 
  of~\cite{ref:ArwUserGuide}]. This includes the GEOGRID, UNGRIB, and METGRID 
  programs used to set up a WRF domain, to interpolate static and 
  climatological terrestrial data, to extract fields from GRIB and GRIB2 
  files, and to horizontally interpolate the fields to the WRF grid. This also
  includes the optional AVG\_TSFC preprocessor, which calculates time-average
  surface temperatures for use in initializing small in-land lake temperatures.

\item The \texttt{ldt/} component contains versions of the NASA Land 
  Data Toolkit (LDT) software~\citep{ref:LdtUserGuide}.  This acts both as a 
  preprocessor for LIS (including interpolation of terrestrial data to the LIS
  grid and separate preprocessing for data assimilation) and a postprocessor 
  for LIS (merging dynamic fields from a LIS off-line ``spin-up'' simulation 
  with static data for eventual input to WRF in LIS coupled mode).

\item The \texttt{RIP4/} component contains a modified copy of the NCAR 
  Graphics-based Read/Interpolate/Plot \citep{ref:RipUserGuide} graphical 
  postprocessing software, version 4.6.7. Modifications include support
  for UMD land use data (from LIS) and WRF-Chem output.

\item The \texttt{ARWpost/} component contains version 3.1 of the 
  GrADS-compatible ARWpost program for visualization of output [see Chapter 9 
  of \cite{ref:ArwUserGuide}].

\item The \texttt{UPP/} component contains a modified copy of the NCEP Unified 
  Post Processor version 3.1.1. This software can derive fields from WRF netCDF 
  output and write in GRIB format (see \cite{ref:UppUserGuide}).

\item The \texttt{MET/} component contains version 6.1 of the Meteorological 
  Evaluation Tools \citep{ref:MetUserGuide} software produced by the 
  Developmental Testbed Center. This can be used to evaluate WRF atmospheric 
  fields converted to GRIB via UPP against observations and gridded analyses.

\item The \texttt{lvt/} component contains version 7.2 of the NASA Land 
  Verification Toolkit \citep{ref:LvtUserGuide} for verifying LIS land and 
  near-surface fields against observations and gridded analyses.

\item The \texttt{GSDSU/} component contains version 4.0 of the Goddard 
  Satellite Data Simulation Unit \citep{ref:GsdsuUserGuide}, which can be used
  to simulate satellite imagery, radar, and lidar data for comparison against 
  actual remote-sensing observations.

\item The \texttt{utils/lisWrfDomain/} utility contains the NASA LISWRFDOMAIN 
  software for customizing LDT and LIS ASCII input files so their domain(s) 
  (grid size, resolution, and map projection) match that of WRF.  It uses 
  output from the WPS GEOGRID program to determine the reference latitude and 
  longitude.

\item The \texttt{utils/geos2wrf/} utility contains version 2 of the NASA
  GEOS2WRF software, which extracts and/or derives atmospheric data from the
  Goddard Earth Observing System Model Version 5 [\cite{ref:RieneckerEtAl2008};
  \url{http://gmao.gsfc.nasa.gov/GMAO\_products}] for input into WRF. It also 
  contains MERRA2WRF, which can preprocess atmospheric fields from the 
  Modern-Era Retrospective Analysis for Research and Applications [MERRA; 
  \cite{ref:RieneckerEtAl2011}] hosted by the Goddard Earth Sciences Data 
  Information Services Center (GES DISC; \url{http://daac.gsfc.nasa.gov}), as 
  well as MERRA-2 reanalyses \citep{ref:BosilovichEtAl2015} also hosted by 
  GES DISC. These programs essentially take the place of UNGRIB in WPS, as 
  UNGRIB cannot read the netCDF, HDF4, or HDFEOS formats used with GEOS-5, 
  MERRA, and MERRA-2.

\item The \texttt{utils/sst2wrf/} utility contains the NASA SST2WRF 
  preprocessor, which reads sea surface temperature (SST) analyses produced by
  Remote Sensing Systems (\url{http://www.remss.com}) and converts into a 
  format readable by the WPS program METGRID. This essentially takes the place
  of UNGRIB as the SST data are in a non-GRIB binary format.

\item The \texttt{utils/ndviBareness4Wrf/} utility contains the NASA\\ 
  NDVIBARENESS4WRF preprocessor, which reads gridded Normalized Difference
  Vegetation Index (NDVI) data and calculates a ``surface bareness'' field
  for use by the NU-WRF dynamic dust emissions scheme.  Currently the
  software reads NDVI produced from the NASA Global Inventory Modeling and
  Mapping Studies (GIMMS) group and by the NASA Short-term Prediction 
  Research and Transition Center (SPoRT).  Both products are based on
  observations from the Moderate Resolution Imaging Spectroradiometer (MODIS),
  which is flown on the NASA \textit{Terra} and \textit{Aqua} satellites.  The
  preprocessor outputs both the NDVI and the derived bareness fields in a 
  format readable by the WPS program METGRID.

\item The \texttt{utils/prep\_chem\_sources/} utility contains a modified
  copy version of the community PREP\_CHEM\_SOURCES version 1.5 
  preprocessor. This program prepares anthropogenic, biogenic, wildfire, and 
  volcanic emissions data for further preprocessing by the WRF-Chem 
  preprocessor CONVERT\_EMISS.  The NU-WRF version of PREP\_CHEM\_SOURCES
  uses the WPS map projection software to ensure consistency in interpolation.
  It also adds support for GFEDv4 biomass burning emissions [see~\citep{ref:RandersonEtAl2015}], 
  NASA QFED wildfire emissions~\citep{ref:QfedDocumentation}, support for new 72-level 
  GOCART background fields, improved interpolation of the GOCART background 
  fields when the WRF grid is at a relatively finer resolution, and output of 
  data for plotting with the NASA PLOT\_CHEM program.

\item The \texttt{utils/plot\_chem/} utility stores a simple NCAR Graphics
  based PLOT\_CHEM program for visualizing the output from PREP\_CHEM\_SOURCES.
  This program is only intended for manual review and sanity checking, not
  for publication quality plots.

\item The \texttt{utils/gocart2wrf/} utility stores version 2 of 
  GOCART2WRF, a NASA program for reading GOCART aerosol data from an offline
  GOCART run~\citep{ref:ChinEtAl2002}, online GOCART with GEOS-5, 
  MERRAero~\citep{ref:KishchaEtAl2014}, or MERRA-2
  \citep{ref:BosilovichEtAl2015} files; interpolate the data to the WRF grid; 
  and then add the data to netCDF4 initial condition and lateral boundary 
  condition (IC/LBC) files for WRF. This includes a script for downloading and
  processing MERRA-2 files from the GES DISC web page.

\item The \texttt{utils/casa2wrf/} utility contains the NASA CASA2WRF 
  preprocessor and related software to read CO$_2$ emissions and 
  concentrations from the CASA biosphere model, and interpolate and append 
  the data to the WRF netCDF4 IC/LBC files~\citep{ref:Casa2WrfUserGuide}.

\item The \texttt{utils/lis4scm/} utility contains the NASA LIS4SCM
  preprocessor to generate horizontally homogeneous conditions in a LIS 
  parameter file (produced by LDT), a LDT output file normally generated for 
  the WRF REAL preprocessor, and a LIS restart file.  These updated files are 
  part of the initial conditions for NU-WRF (WRF and LIS coupled) in idealized 
  Single Column Mode (see section \ref{subsec:ScmLisCoupling}).

\item The \texttt{utils/nevs/} directory contains the NU-WRF Ensemble Validation System (NEVS) 
scripts. These use a Jupyter Notebook interface that run the scripts to simplify the process of 
comparing the results of an ensemble of NU-WRF forecasts with the actual observations from 
the same period. It relies on NOAA/UCAR Model Evaluation Tools (MET) to do a statistical 
comparison of the forecast and observation datasets. The output of NEVS includes several 
types of plots, including RMSE, Bias, Pearson Correlation, and 3 types of forecast Skill Scores.

\subsubsection{Note}
As mentioned in section \ref{sec:Introduction}, \texttt{ARWpost/}, \texttt{MET/}, \texttt{RIP4/} and \texttt{UPP/}  are not distributed with NU-WRF and therefore their respective directories will not be available with the GiHub code base. However, on NASA systems, the NU-WRF build mechanism will use the code stored in the NU-WRF file system and will build and/or install the component if the user chooses to do so.  At that point, the component directory will be available but will not be under version control.

\end{itemize}

In addition, NU-WRF includes a unified build system written in the python 
scripting language to ease compilation of the different NU-WRF components, 
and to automatically resolve several dependencies between the components 
(e.g., WPS requires WRF to be compiled first). The build system is discussed
in section \ref{subsec:BuildSystem}.

\subsection{Other Features}

While NU-WRF aims to be a superset of the official WRF modeling system,
there are some WRF components that are currently not supported. These are:

\begin{itemize}
\item WRF-Hydro, a hydrologic library for WRF (see 
\url{https://www.ral.ucar.edu/projects/wrf_hydro}). The source code for 
WRF-Hydro is located in \texttt{WRF/hydro/}.  

\item WRFDA, the data assimilation library for WRF [see 
\cite{ref:BarkerEtAl2012}, Chapter 6 of \cite{ref:ArwUserGuide}, and
\url{http://www2.mmm.ucar.edu/wrf/users/wrfda/}]. The source code for WRFDA is
not included in NU-WRF Bjerknes or later.

\item OBSGRID, an objective analysis program for WRF [see Chapter 7 of 
\cite{ref:ArwUserGuide}]. The source code for this program is not included
with NU-WRF.

\item PROC\_OML, a preprocessor for interpolating HYCOM
[see \cite{ref:HycomUserGuide} and \url{https://hycom.org}] 3D ocean model 
temperature data, and for writing for further processing by the WPS METGRID 
program. See Chapter 10 of \cite{ref:ArwUserGuide}. The source code for this 
program is not included with NU-WRF.

\item WRF-NMM, an alternative dynamic core developed by NOAA/NCEP. The NMM 
source code is located in \texttt{WRF/dyn\_nmm/} but there is no support to 
compile it with the NU-WRF build system. While it may be possible to run 
WRF-NMM with the physics packages and coupling added by NASA, it has never 
been tested and is not recommended.  (Note that as of May 2016, WRF-NMM is no 
longer supported by the Developmental Testbed Center.)

\item HWRF, the Hurricane WRF modeling system developed by NOAA/NCEP with 
on-line coupling between WRF-NMM and the Princeton Ocean Model for Tropical 
Cyclones (see \url{http://www.dtcenter.org/HurrWRF/users/index.php}). This is 
a standalone software package provided by the Developmental Testbed Center, 
and is not included in NU-WRF.

\item READ\_WRF\_NC, a utility similar to the netCDF NCDUMP program
for examining WRF netCDF output [see Chapter 10 of \cite{ref:ArwUserGuide}]. 
The source code is not included in NU-WRF.

\item IOWRF, a utility for manipulating WRF-ARW netCDF data [see Chapter 10 
of~\cite{ref:ArwUserGuide}]. The source code is not included in NU-WRF.

\item WRF\_INTERP, a new utility to interpolate WRF data to isobaric, 
constant height (AGL or MSL), isentropic, or constant equivalent potential
temperature levels.  The source code is not included in NU-WRF.

\item P\_INTERP, a utility to interpolate WRF netCDF output to user-specified
isobaric levels [see Chapter 10 of~\cite{ref:ArwUserGuide}]. The source code 
is not included in NU-WRF.

\item V\_INTERP, a utility to add vertical levels to a WRF-ARW netCDF file 
[see Chapter 10 of~\cite{ref:ArwUserGuide}]. The source code is not included
in NU-WRF.

\end{itemize}

In addition, the NU-WRF build system (discussed in Section 
\ref{subsec:BuildSystem}) does not currently support compilation of the 2-D
``ideal'' data cases nor the 3-D WRF-Fire data case [see Chapter 4 of 
\cite{ref:ArwUserGuide}].  In addition, NU-WRF does not support 
compilation with OpenMP or hybrid OpenMP-MPI due to a lack of OpenMP support 
in some of the physics packages.


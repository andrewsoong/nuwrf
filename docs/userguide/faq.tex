\section{Frequently Asked Questions}
\label{sec:FAQ}

\paragraph{Q:} What modules should I use on Discover and Pleiades when running
NU-WRF?

\paragraph{A:} Modules are automatically loaded if you used the build.sh script provided with NU-WRF. The current defaults on Discover are (assuming Bash shell):

\begin{quote}
\texttt{source /usr/share/modules/init/bash} \\
\texttt{module purge} \\
\texttt{unset LD\_LIBRARY\_PATH} \\
\texttt{module load comp/intel-18.0.3.222} \\
\texttt{module load mpi/impi-18.0.3.222} \\
\texttt{module load lib/mkl-18.0.3.222} \\
\texttt{module load other/comp/gcc-7.3} \# For Intel C++ compiler \\
\texttt{export LD\_LIBRARY\_PATH=\$LD\_LIBRARY\_PATH:/usr/lib64} \\
\end{quote}

And on Pleiades:

\begin{quote}
\texttt{source /usr/share/modules/init/bash} \\
\texttt{module purge} \\
\texttt{unset LD\_LIBRARY\_PATH} \\
\texttt{module load comp-intel/2015.3.187} \\
\texttt{module load gcc/6.2} \# For Intel C++ compiler \\
\texttt{module load mpi-sgi/mpt} \# Per NAS recommendation \\
\texttt{export LD\_LIBRARY\_PATH=\$LD\_LIBRARY\_PATH:/usr/lib64} \\
\end{quote}

If you must change the modules (perhaps to test a different compiler version) or port to a new platform make sure that you modify the settings in scripts/other/set\_module\_env.bash. This includes specifying the location of third-party libraries used by NU-WRF.

The module settings should be done in your shell if you are running a program 
at the command line. If instead you are launching a batch job to SLURM or PBS,
the above settings should be in the batch script, so the commands are run on 
the allocated compute node.

\paragraph{Q:} What other environment settings should I use?

\paragraph{A:} Make sure to set stack size to unlimited. This can help 
prevent memory allocation errors for automatic arrays. If using bash:

\begin{quote}
\texttt{ulimit -s unlimited}
\end{quote}

If using csh:

\begin{quote}
\texttt{limit stacksize unlimited}
\end{quote}

\paragraph{Q:} What namelist settings should I use with WRF?

\paragraph{A:} It is not possible to provide a single configuration optimal for
all types of simulations (LES, regional climate, cloud system resolving, 
chemical transport, etc), but we have recommendations that provide a
reasonable first-guess.
\\

\begin{tabular}{|l|l|l|l|} \hline
Category     & Selection     & Namelist           & Comments             \\ \hline
Microphysics & NU-WRF Goddard& mp\_physics=56     & Latest stable        \\
             & 4ICE          &                    & NASA option.         \\ \hline

Radiation    & NU-WRF Goddard& ra\_lw\_physics=57 & Latest stable NASA   \\
             & 2017 Radiation& ra\_sw\_physics=57 & option.              \\ \hline
Aerosol      & GOCART        & chem\_opt=300       & Simple aerosol, \\
Coupling     & simple        & gsfcgce\_gocart\_coupling=1 & coupled with    \\
             & aerosol       & gsfcrad\_gocart\_coupling=1 & radiation and   \\
             &               & vertmix\_onoff=1            & microphysics,   \\
             &               & chem\_conv\_tr=1            & no gas          \\
             &               & dust\_opt=1                 & chemistry.      \\
             &               & seas\_opt=1                 & \\
             &               & dmsemis\_opt=1              & \\ \hline
LSM          & LIS Noah 3.6  & sf\_surface\_physics=55     & Spin-up LIS on  \\
             &               & num\_soil\_layers=4         & WRF grid for    \\
             &               &                             & detailed initial\\
             &               &                             & fields; cannot  \\
             &               &                             & use moving      \\
             &               &                             & nests; must have\\
             &               &                             & lis.config file \\ \hline
PBL          & MYNN2         & bl\_pbl\_physics=5          & Replaces MYJ PBL\\
             &               & sf\_sfclay\_physics=5       & scheme; used in \\
             &               &                             & RAPv2 and HRRR; \\
             &               &                             & reportedly gives\\
             &               &                             & unbiased PBL    \\
             &               &                             & depth, moisture,\\
             &               &                             & and temperature.\\ \hline
Cumulus      & G3            & cu\_physics=5               & Third-gen Grell\\
             &               & ishallow=1                  & scheme; tackles\\
             &               &                             & ``grey zone''; \\
             &               &                             & compatible with\\
             &               &                             & used in RAPv2; \\
             &               &                             & handles shallow\\
             &               &                             & cumulus.\\ \hline
\end{tabular}\\

\begin{tabular}{|l|l|l|l|} \hline
Category     & Selection     & Namelist           & Explanation          \\ \hline
Diffusion    & 2nd Order on  & diff\_opt=1        & Appropriate for real \\
             & coordinate    & km\_opt=4          & data cases           \\
             & surfaces; eddy&                    & (dx $\geq$ 1 km)     \\
             & coefficient   &                    & \\
             & based on      &                    & \\
             & deformation   &                    & \\ \hline
6th-order    & Monotonic     & diff\_6th\_opt=2   & Removed 2*dx noise in  \\
horizontal   &               & diff\_6th\_factor=0.12 & light winds; can be\\
diffusion    &               &                        & tuned. \\ \hline
Advection    & 5th-order,    & moist\_adv\_opt=1   & Used in RAPv2 and   \\
             & positive-     & scalar\_adv\_opt=1  & HRRR; positive-     \\
             & definite      & tke\_adv\_opt=1     & definite prevents   \\
             &               & chem\_adv\_opt=1    & negative mixing     \\
             &               & momentum\_adv\_opt=1& ratios from original\\
             &               & h\_mom\_adv\_order=5& non-negative values.\\
             &               & h\_sca\_adv\_order=5& \\
             &               & v\_mom\_adv\_order=5& \\
             &               & v\_sca\_adv\_order=5& \\ \hline
Rayleigh     & Implicit      & damp\_opt=3         & Prevents gravity    \\
Damping      &               & zdamp=5000.         & waves from          \\
             &               & dampcoef=0.2        & reflecting off model\\
             &               &                     & top; designed for   \\
             &               &                     & real-data cases;    \\
             &               &                     & used in RAPv2 and   \\
             &               &                     & HRRR.\\ \hline
Vertical     & Activated     & w\_damping=1        & Damps updrafts      \\
Velocity     &               &                     & approaching CFL     \\
Damping      &               &                     & limit; best used for\\
             &               &                     & long or quasi-      \\
             &               &                     & operational runs. \\ \hline
Time Off-    & Tuning factor & epssm=0.1           & Controls vertically-\\
Centering    &               &                     & propagating sound   \\
             &               &                     & waves; set to max   \\
             &               &                     & slope of model      \\
             &               &                     & terrain. \\ \hline
Nesting      & 1-Way         & feedback=0          & 2-way nesting does  \\
             &               &                     & not work with LIS,  \\
             &               &                     & and can lead to     \\
             &               &                     & strange results with\\
             &               &                     & high-res nesting.   \\ \hline
\end{tabular}\\
\\

The above recommendations have some caveats: 

\begin{itemize}

\item The PBL and cumulus settings are the most debatable physics selections.
Our recommendations are mostly because of their use in the NCEP RAPv2 and 
HRRR models which speak to their robustness. In addition, WRF-Chem requires a 
Grell-family cumulus scheme for most simulations, placing these schemes at an 
advantage over popular alternatives such as Kain-Fritsch~\citep{ref:Kain2004}
and Betts-Miller-Janji\'{c}~\citep{ref:Janjic1994}. However, the G3 scheme has
a pronounced wet bias when shallow cumulus is turned on, and the new 
Grell-Freitas scheme~\citep{ref:GrellFreitas2014} changes answers with varying
 CPU counts (the cause is under investigation).  As for the PBL setting, the 
YSU scheme~\citep{ref:HongEtAl2006} is a popular alternative and includes 
options for subgrid orographic effects [e.g., \cite{ref:JimenezDudhia2012}].  
Users are encouraged to experiment.

\item 6th-order diffusion was added to WRF because the normal diffusion scheme
is tied to the wind speed, and can insufficiently smooth the fields in light
winds~\citep{ref:KnievelEtAl2007}. Users running a short case may wish to turn
off the 6th-order scheme to see if $2\Delta{}x$ features develop in the 
vertical velocity and divergence fields without it.

\item A popular alternative to the positive-definite advection filter is the
monotonic filter~\citep{ref:WangEtAl2009}, which damps both positive and 
negative spikes in the advected fields (the positive-definite only damps 
negative spikes). Unfortunately monotonic advection may lead to excessive 
smoothing when 6th-order diffusion is also turned on. The user may wish to 
experiment with monotonic advection and turning off 6th-order diffusion, 
particularly with short chemistry runs where winds are not too light.

\item Vertical velocity damping is artificial and is recommended mostly for
situations where CFL violations are particularly unwelcome (e.g., quasi-
operational runs).

\item 2-way nesting cannot be used with WRF-LIS coupling because the feedback
routine may change the land/sea mask for a parent WRF grid to better match
the child WRF grid.

\end{itemize}

\paragraph{Q:} What settings should I use with GEOGRID?

\paragraph{A:} For GEOGRID, we recommend these settings:

\begin{itemize}
\item Use MODIS land use data instead of USGS. This requires changing the
geog\_data\_res variable in \texttt{namelist.wps} to something like:

\begin{quote}
 geog\_data\_res = 'modis\_30s+10m','modis\_30s+2m','modis\_30s+30s',
\end{quote}

or, if using lake temperature initialization:

\begin{quote}
 geog\_data\_res = 'modis\_lakes+10m','modis\_lakes+2m','modis\_lakes+30s',
\end{quote}

GEOGRID will check the \texttt{GEOGRID.TBL} settings to relate the selections
to each dataset (terrain, soil type, etc). The 'modis\_30s' ('modis\_lakes')
will only match with the land-use data and will force processing of MODIS 
data; the remaining data types will fall back on '10m', '2m', or '30s' for 
the respective WRF grid.  See Chapter 3 of~\cite{ref:ArwUserGuide} for more 
information.

\item Process all EROD data with GEOGRID. This requires use of the new
\texttt{GEOGRID.TBL.ARW\_CHEM\_NUWRF} table which lists entries for four 
different EROD datasets. The user does not need to decide which EROD option
to use until running REAL.

\end{itemize}


\paragraph{Q:} Why does WRF or REAL fail with this error about NUM\_LAND\_CAT?
\begin{quote}
----------------- ERROR -------------------

 namelist    : NUM\_LAND\_CAT =         20

 input files : NUM\_LAND\_CAT =         24 (from geogrid selections).

 -------------- FATAL CALLED ---------------

 FATAL CALLED FROM FILE:  $\langle$stdin$\rangle$  LINE:     709

 Mismatch between namelist and wrf input files for dimension NUM\_LAND\_CAT
\end{quote}


\paragraph{A:} There are two possible causes.

First, you are running WRF without LIS coupling, your land use data is coming
from GEOGRID, and your namelist settings for land use are inconsistent between
GEOGRID, REAL, and WRF. Normally these programs expect USGS data (with 24 
categories). If you configure GEOGRID to process MODIS instead, you must set 
NUM\_LAND\_CAT in namelist.input to 20 for consistency.

Second, you are trying to run WRF coupled with LIS, REAL is replacing the 
landuse data from GEOGRID with that from LDT, and your namelist.input is not
consistent. In this case, NUM\_LAND\_CAT should match the value from GEOGRID
when you run REAL, but should match the value from LDT when you run WRF. 
(The reference to geogrid in the error message from WRF is incorrect in this
case, and stems from the community WRF not knowing anything about LDT.) Note
that LDT can provide USGS (24 categories), MODIS (20 categories), or UMD
(14 categories).

\paragraph{Q:} How do I convert between netCDF3 and netCDF4?

\paragraph{A:} NetCDF comes with a NCCOPY utility that allows you to 
convert files between variants of netCDF.  This can be useful for sharing
NU-WRF netCDF4 data with outside users that prefer netCDF3, or for compressing
large legacy netCDF3 files with netCDF4/HDF5. For best results, make sure you 
use a version of NCCOPY that was compiled with netCDF4/HDF5 compression 
support.

To convert to 64-bit netCDF3 (large file support):

\begin{quote}
\texttt{nccopy -k 2 infile outfile}
\end{quote}

And to convert to netCDF4:

\begin{quote}
\texttt{nccopy -k 3 infile output}
\end{quote}

It is also possible to remove compression from a netCDF4 file, which may make
it readable to netCDF3 users:

\begin{quote}
\texttt{nccopy -d 0 infile outfile}
\end{quote}

Run \texttt{man nccopy} for more information, including options for tuning 
netCDF4/HDF5 compression. Note that writing to netCDF3 classic format 
(\texttt{-k 1}) is \emph{not} recommended due to file size limitations. 

\paragraph{Q:} To be asked.

\paragraph{A:} To be answered.


